\subsection{Governance}\label{sec:governance}
 \paragraph{Relay Chain Council Referenda}
 \paragraph{Parachain Allocation}
We use auctions to have a fair and transparent parachain allocation procedure.
Since implementing seal-bid auctions are difficult and to avoid bid sniping we adopt an Candle auction \cite{Fuellbrunn:2012:CandleAuction} with a retroactively determined close as follows.

Once the auction has started within a fixed window bidders can post bids for the auction.
Bids go into the block as transactions.
Bidders are allowed to submit multiple bids.
Bids that a bidder is submitting either should intersect with all winning bids by same bidder or be contiguous with winning bids by the same bidder.
If an incoming bid is not changing the winners it is ignored.

For 4 lease periods we have 10 possible ranges.
We store the winner for each one of the 10 ranges in a designated data structure.
We need to make sure that a new bid does not have a gap with a winning bid on another interval from the same bidder.
This means that once a bidder has won a bid for a given range, say for example, lease periods 1-2 then he cannot bid on lease period 4 unless someone overbids him for lease periods 1-2.

For any incoming bid the new winner is calculated by choosing the combination of bids where the average deposit for overall all 4 lease periods is most.
Once a bid is added to the block, the amount of their bid gets reserved.

Once a fixed number of blocks have been produced for the auction a random numbers decides which one of the previous blocks was the closing block and we return the winners and their corresponding ranges for that closing block.
The reserved funds of losers are going to be released once the ending time of the auction is determined and the final winners are decided.

For example, let us assume we have three bidders that want to submit bids for a parachain slot.
Bidder $B_1$ submits the bid (1-4,75 DOT), bidder $B_2$ submits (3-4, 90 DOTs), and bidder $B_3$ submits (1-2, 30).
In this example bidder $B_1$ wins because if bidder $B_2$ and bidder $B_3$ win each unit would only be locked for an average of 60 DOTs or something else equivalent to 240 DOT-intervals, while of bidder $B_1$ wins each unit is locked for 75 DOTs.

 \paragraph{Treasury}

 The system needs to continually raise funds, which we call the treasury.
 These funds are used to pay for developers that provide software updates, apply any changes decided by referenda, adjust parameters, and generally keep the system running smoothly.

 Funds for treasury are raised in two ways:

 1.   by minting new tokens, leading to inflation, and
 2.   by channelling the tokens set for burning from transaction fees and slashing.

 Notice that these methods to raise funds mimic the traditional ways that governements raise funds: by minting coins which leads to controlled inflation, and by collecting taxes and fines.

 We could raise funds solely from minting new tokens, but we argue that it makes sense to redirect into treasure the tokens from transaction fees and slashing that would otherwise be burned:

 - By doing so we reduce the amount of actual stake burning, and this gives us better control over the inflation rate (notice that stake burning leads to deflation, and we can’t control or predict the events that lead to burning).

 - Following an event that produced heavy stake slashing, we might often have to reimburse the slashed stake, if there is evidence of no wrongdoing. Thus it makes sense to have the dots availabe in treasury, instead of burning and then minting.

 - Suppose that there is a period in which there is an unusually high amount of stake burning, due to either misconducts or transaction fees. This fact is a symptom that there is something wrong with the system, that needs fixing. Hence, this will be precisely a period when we need to have more funds available in treasury to afford the development costs to fix the problem.
