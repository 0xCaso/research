\subsection{Relay Chain State Machine}\label{sec:relaychain}

Formally, Polkadot is a replicated sharded state machine where shards are the parachains and Polkadot relay chain is part of the protocol ensuring global consensus among all the parachains. Therefore the Polkadot relay chain protocol, can itself be considered as a replicated state machine on its own. In this sense, this section describes the relay chain protocol by specifying the state machine governing the relay chain. To that end, we describe the relay chain state and the detail of state transition govern by transactions grouped the relay chain blocks.

\paragraph{State}
Polkadot relay chain state is represented similar to of the Ethereum. In the sense that the state is represented using an <i>associative array</i> data structure composed of a collection of $(key, value)$ pairs where each key is a unique. There is no assumption on the format of the key or the value stored under it beside the fact that they are finite byte arrays.

A <i>Markle radix-16 trie</i> keeps the Merkle hashes corresponding to the $(key, value)$ pairs stored in the relay chain state enable identifying current state using its root hash and providing efficient proof of inclusion of a specific pair.

To keep state size in control, the relay chain state is solely used to facilitate relay chain operation such as staking and identifying Validators. It is not suppose to store any information regarding the internal operation of the parachains.

\paragraph{Transaction Inclusion}

\paragraph{Block Building}\label{sec:relaychainblockproduction}
 
