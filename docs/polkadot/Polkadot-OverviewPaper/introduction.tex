\section{Introduction}\label{sec:intro}
The Internet was originally designed for and built upon decentralised protocols such as TCP/IP, however, its commercialisation has led to the centralisation of all popular web applications today. We refer not to any centralisation of physical infrastructure, but rather to the logical centralisation of power and control over the infrastructure. Two prominent examples are giant corporations like Google and Facebook: while they maintain servers all around the world in a physically decentralised fashion, these are ultimately controlled by a single entity.

A central entity controlling a system poses many risks for everyone. For example, they can stop the service at any moment, can sell users' data to third parties, and manipulate how the service is working without the users' agreement. This is in particular critical to users who heavily rely upon these services for business or private purposes.

With all the current interest related to personal data ownership, there is a growing need for a better security, freedom and control for net users, and with this a countermovement for more decentralised applications where no single entity controls the system. This tendency towards decentralisation is not new; it has been used in a number of areas of the development of web and other systems, such as the free software movement. %and there are hints of these tedencies throughout history, notably antitrust and antimonopoly legislation in the 1920s. %Today, we can see decentralisation practices in different areas of the online development such as as games \cite{}.

Blockchains are a recent technology proposed to address these problems, in the quest to build a decentralised web. However, it can only compete with the centralised web if it is usable for masses of end-users. One important aspect of this is that separate applications must be able to interact, otherwise each will become isolated and not adopted by as many users. Having to build such an interoperability mechanism introduces new challenges, many of which are missing in the centralised model because of the fundamental differences in the trust model between the two paradigms.
%The simple authentication mechanisms such as OAuth \cite{hardt2012oauth} can not easily be adapted to the decentralised web where there is no central entity guarding the secret key.
For example, Bitcoin\cite{nakamoto2008bitcoin} and Ethereum\cite{buterin2014ethereum} are proof-of-work (PoW) blockchains where security relies on assumptions about processing power; alternative proof-of-stake (PoS) systems' security instead rely on incentives and the ability to destroy security deposits. These differences present difficulties for one blockchain to trust another. Another challenge that blockchain technologies need to tackle is scalability. Existing blockchain systems generally have high latency and can only have tens of transactions per second \cite{scaling}, whereas credit card companies such as Mastercard or Visa carry out thousands of transactions per second \cite{Visa2020}.

One prominent solution to scalability for blockchains is to run many chains in parallel, often called sharding. %and enabling off-chain transactions \cite{gudgeon2019sok} where only the final balance is recorded on the blockchain.
Polkadot is a multi-chain system that aims to gather the security power of all these chains together in a shared security system. It was first introduced in 2016 by Gavin Wood \cite{2016:Wood:Polkadot}, and in this paper we expand on the details.

Briefly: Polkadot utilises a central chain called the \emph{relay chain} which communicates with multiple heterogeneous and independent sharded chains called \emph{parachains} (portmanteau of “parallel chains”). The relay chain is responsible for providing shared security for all parachains, as well as trust-free interchain transactability between parachains. In other words, the issues that Polkadot aims to address are those discussed above: interoperability, scalablility, and weaker security due to splitting the security power.


\paragraph{Paper Organisation} In the next section, we give a synopsis of the Polkadot network including its external interface with client parachains that we expand on in subsequent sections. We review preliminary information such as description of roles of Polkadot's participants and our adversary model in Section \ref{sec:preliminaries}. We explain what subprotocols and components of Polkadot try to achieve in Section \ref{sec:components} and then continue to review them in detail, including low-level cryptographic and networking primitives. Finally we shortly discuss some future work in Section \ref{sec:futurework}. In the appendices, we review relevant work such as a comparison to other multi-chain system  \ref{sec:comparison}, a short description of an interoperability scheme to bridge to external chains \ref{sec:bridge}, a secure execution schemes \ref{sec:SPREE} that we will use for  messaging, and a glossary with Polkadot-specific terms in Table \ref{t:time}. 

%\syed{Polkadot enables the parachains to communicate together and have shared security.}{So we do not need to repeat this}

%\handan{Furthermore, scalability is another challenge for blockchains to make them comparable with centralised services in functionality.}{it may be better to talk about it when we mention challenges of interoperability}\syed{}{I agree with Handan, specially that scalability is partially related to interoperability, for example, the reason that Ethereum tried to be generic computer is because it wanted to have all Dapps lives inside so they can communicate efficiently which resulted in scalability problem. But we also can talk much more about scalability }
%Polkadot's hierarchical structure addresses this challenge.


%The paper is organised as follows. In Section~\ref{sec:properties} we review the properties that Polkadot is aiming for and Section~\ref{sec:preliminiary} examines the structural components and roles that we defined for Polkadot.
%Section~\ref{sec:components} presents all the underlying components including protocols and primitives that have been designed for Polkadot as well as a note about incentives and economics.

