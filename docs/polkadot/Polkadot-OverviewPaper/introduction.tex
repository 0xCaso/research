\section{Introduction}\label{sec:intro}
\syed{Until recently, most of the applications on the web were controlled in a centralised fashion.}{I'd start less abruptly plus Web is still centralized, ``until recently'' suggests otherwise. I suggest something like this: Internet was originally designed as a collection of interconnected decentralized entities which is noticeable by the design of its early protocols such as TCP/IP and SMTP. However, its commercialisation has led to the centralisation of most of the applications present on the web}. \syed {In this paper, w}{W}e refer not to the centralisation of physical infrastructure, which is often economically efficient, but rather to the logical centralisation of the ability to deploy, take down, alter, and change the rules of the application. Two \syed{}{prominent} examples are \syed{giant corporations like} Google and Facebook: while they \syed {do have}{run/maintain} servers all around the world, these are ultimately controlled by a single legal entity.

Giving a central entity control over a system \syed{comes}{poses} several} risks \syed {for users and}{as much for the users as for the} application developers. \syed{T}{For example, t}he central entity can stop the service at any moment, can sell users' data \syed{to third parties} depending on the jurisdictions they provide service to, and manipulate how the service is working without \syed {the} users consent.

  \syed {With the need for security and the urge for more freedom and fairness \syed {on the web}, the \syed{era}{process} of decentralised web applications, where no single entity can control the system, is \syed{emerging}{moving to a next stage/evolving}.}
{With all this current development around corporative interests related to personal data and applications ownership, there is a current countermovement and a growing need for a better security, freedom and fairness for the users on the web, and with this - for more decentralised applications, where no single entity can control the whole system.} 
    
\syed {This decentralised web is comprised of many types of applications such as \handan{games, storage, decentralised exchanges, auctions, financial systems, etc.}{ we may add some references for each application.} } {The process of decentralization is not new. It has been used in a number of areas of the development of web and other systems, starting with the free software movement in the 1970s. Today, we can see decentralisation practices in different areas of the online development such as... } 

One of the fundamental challenges of \syed{the}{developing a} decentralised \syed{web}{application} is \syed{keeping state}{keeping state is too technical to jump at the intro, maybe a better alternative is: storing and communicating the current state of the application}. \syed{T}{That is because t}here is no central entity who follows the current state or can decide what the current valid state is if there is any doubt. Blockchains are \eray{one of the}{-remove-} technologies \syed{who}{which are proposed to} address this problem.

In order to make the decentralised web usable for end-users, \syed {these separate}{which ones?} blockchains need to interact, otherwise, each will become isolated and not adopted by as many users, and the \handan{sum}{replace: overall?} functionality of the system will be insufficient to compete with the centralised web. \handan{However, to interact with different chains, \syed{we need}{one needs/there is a need} to build an interoperability mechanism, which introduces challenges. Many of these challenges arise from the ways that different blockchains have different technical \syed {incompatible between them} characteristics.}{Therefore, interoperability mechanisms are urgent for blockchain systems but such mechanisms bring along new challenges. \syed{Many of these challenges arise because of different technical infrastructures and functionalities that blockchains have.}{I would argue that neither technical nor functional differences are the main source of the problem but the lack of trust is the culprit as it actually argued in the following sentences, which is also the reason why Cosmos is not a very useful system to solve the interoperability problem. I'd suggest this: Many of these challenges which are missing in the centralized model arise because of the fundamental differences in the trust model between the two paradigms. The simple authentication mechanisms such as OAuth \cite{hardt2012oauth} can not easily be adapted to the decentralized web where there is no central entity guarding the secret key. Furthermore, it is not straight forward for decentralized entities to establish trust between each other when they use different mechanisms to ensure the security of their respective systems}}. For example, \handan{Bitcoin and Ethereum}{refs} are proof-of-work (PoW) blockchains where \handan{preventing the control of the whole system by one entity}{replace: decentralization?} is carried out by introducing \handan{challenges}{maybe puzzle or something else is better than `challenges' not to confuse with the challenges that we talked one sentence before} that need a large amount of \handan{processing}{maybe computing?} power to solve. Their security relies on whether this amount of processing power exceeds the amount of processing power any single entity would possibly have. An alternative to PoW \eray{are}{is a} proof-of-stake (PoS) \eray{systems}{system}, where the entities \eray{who}{that} control the system have to lock a large amount of funds and would be punished if they misbehave in any way.
The security relies on the fact that the total amount of locked funds exceeds the budget any \syed {}{the} adversary is able to invest in the attack. These technical differences present difficulties for one blockchain to be able to trust another.

\handan{}{this paragraph looks like repetition of the previous one. Maybe it is better to combine them.}\syed{}{it can be integrated into the previous paragraph}. Currently, there have been hundreds of chains implemented in the wild for various functions and non-compatible underlying technology.
Most \eray{one}{-remove-} of these systems have different properties they aim to achieve and \eray{}{a} certain security threshold that they set up.
Hence, there is a need for a system that \syed{can connect these heterogeneous chains}{enables these heterogeneous chains to communicate and interact based on a common security framework}. 
\syed{Moreover, having multiple security setups causes a split in the security these systems can provide. Gathering all this security power and have a shared security system increases security substantially.}{This part should clearly goes behind the hence part}

\handan{}{we need to improve this paragraph}\syed{I}{As a solution to the above mentioned interoperability problem}, in this paper, we introduce Polkadot, a multi-chain system with shared security guarantees. \syed{}{There is an abrupt jump here into technicality}\syed{Polkadot consists of the main chain}{To provide such a shared security framework, Polkadot utilizes a central chain} called the \emph{Relay Chain} \syed{that guarantees the security and}{which communicates with} multiple heterogeneous \syed{parallel}{independent} chains called \emph{Parachains}\syed{(portmanteau of parallel chains)}.
\syed{The security goal of Polkadot is to be Byzantine fault-tolerant}{This BFT comment is out of its place}. \syed{Polkadot enables the parachains to communicate together and have shared security.}{So we do not need to repeat this}

\handan{Furthermore, scalability is another challenge for blockchains to make them comparable with centralised services in functionality.}{it may be better to talk about it when we mention challenges of interoperability}\syed{}{I agree with Handan, specially that scalability is partially related to interoperability, for example, the reason that Ethereum tried to be generic computer is because it wanted to have all Dapps lives inside so they can communicate efficiently which resulted in scalability problem. But we also can talk much more about scalability }
Polkadot's hierarchical structure addresses this challenge.

\section{Comparison with other multi-chain systems}\label{sec:comparison}
\paragraph{ETH2.0}

\subsubsection{Sidechains}
\paragraph{Cosmos} 

Much like Polkadot, Cosmos is a system designed to solve the blockchain interoperability problem that is fundamental to improve the scalability for the decentralized web. In this sense, there are surface similarities between the two systems. Hence, Cosmos consists of components which play similar roles and resemble the sub-components of Polkadot. For example, the Cosmos Hub is used to transfer messages between Comos' zones similarly to how the Polkadot Relay Chain oversees the passing of messages among Polkadot parachains.

There are however significant differences between the two systems. Most importantly, while the Polkadot system as a whole is a sharded state machine (See Section \ref{sec:relaychain}), Cosmos does not attempt to unify the state among the zones and so the state of individual zones is not reflected in the Hub's state. As the result, unlike Polkadot, Cosmos does not offer shared security among the zones. Consequently, the Cosmos cross-chain messages, are no longer trust-less. That is to say, that a receiver zone needs to fully trust the sender zone in order to act upon messages it receives from the sender. If one considers Cosmos system as a whole, including all zones in a similar way one analyses the Polkadot system, the security of such a system is equal to the security of the least secure zone. Similarly the security promise of Polkadot guarantees that validated parachain data are available at a later time for retrieval and audit (See Section \ref{sec:validity-and-availability}). In the case of Cosmos, the users are ought to trust the zone operators to keep the history of the chain state.

It is noteworthy that using the SPREE modules, Polkadot offers even stronger security than the shared security (See Section \ref{sec:spree}. When a parachain signs up for a SPREE module, Polkadot guarantees that certain XCMP messages received by that parachain are being processed by the pre-defined SPREE module set of code. No similar cross-zone trust framework is offered by the Cosmos system.

Another significant difference between Cosmos and Polkadot consists in the way the blocks are produced and finalized. In Polkadot, because all parachain states are strongly connected to relay chain states, the parachain can temporarily fork alongside the relay chain. This allows the block production to decouple from the finality logic. In this sense, the Polkadot blocks can be produced over unfinalized blocks and multiple blocks can be finalized at once. On the other hand, the Cosmos zone depends on the instant finality of the Hub's state to perform a cross-chain operation and therefore a delayed finalization halts the cross-zone operations.


%The paper is organised as follows. In Section~\ref{sec:properties} we review the properties that Polkadot is aiming for and Section~\ref{sec:preliminiary} examines the structural components and roles that we defined for Polkadot.
%Section~\ref{sec:components} presents all the underlying components including protocols and primitives that have been designed for Polkadot as well as a note about incentives and economics.

