\subsection{Parachains}\label{sec:parachains}

\subsubsection{Block Production}\label{sec:parachainblockproduction}

We will discuss block production for a general parachain. At the end of the section, we will discuss akternatives.

In outline, a collator produces a parachain block, sends it to the parachain validators,
who sign its header as valid, and the header with enough signatures is placed on the relay chain.
At this point, the parachain block is as canonical as the relay chain block its header appeared in.
If that is in the best chain according to BABE (See Section \ref{sec:babe}), so is the parachain block and when that is finalised, so is the parachain block.

Because the parachain validators switch parachains frequently, they are stateless clients of the parachain.
Thus we distinguish between the parachain block $B$, which is normally enough for full nodes of the parachain such as collators to update the parachain state,
and the {\em Proof of Validity(PoV)} block $B_{PoV}$, which a validator who does not have the parachain state can verify.

Any validator should be able to verify $B_{PoV}$ given the relay chain state using the parachain's {\em state transition validation function} (STVF),
the Web asssembly code for which is stored on the relay chain in a simailr way to the relay chain's runtime.
The STVF takes as an input the PoV block,
\begin{itemize}
	\item  the header of the last parachain block from this parachain and,
	\item the message roots of all parachain blocks that have sent messages to this parachain since the last parachain block
	(see Section \ref{sec:XCMP} for details such as the range needed)
\end{itemize}
The STVF outputs the validity of the block, the header of this block and its outgoing messages.
The PoV block contains any outgoing messages and the parachain block $B$. The parachain validators should gossip the parachain block to the parachain network, as a back up to the collator itself doing so.

The PoV block will be the parachain block, its outgoing messages, its header and light client proof witnesses. These witnesses are Merkle proofs that give all elements of the input and output state that are used or modified by the state transition from the input and output state roots.


To aid in censorship resistance, a parachain may want to use proof of work or proof of stake to selected collators, where the selection strategy is up to the given parachain.
This can be implemented in the STVF and need not be a part of the Polkadot protocol. So for proof of work,
the STVF would check that the hash of the block is sufficiently small.
However, for speed, it would be useful to ensure that most relay chain blocks can include a parachain block.
For PoW, this would necessitate it being probable that multiple collators are allowed to produce a block.
As such we will still need a tie-breaker for the parachain validators to coordinate on validating the same parachain block first.
This may be the golden ticked scheme of \cite{2016:Wood:Polkadot}. For proof of stake this may not be necessaery.

Optionally, for some parachains, the parachain block $B$ may not be enough for collators to update their state. This may happen for chains that use succinct zero-knowledge proofs to update their state, or even for permissioned chains that just give signatures from authorities for validity. Such chains may have another notion of parachain block which is actually needed to update their state and must have their own scheme to guarantee the availability of this data.





\subsubsection{Validity and Availability} \label{sec:validity-and-availability}
Once a block is created it is important that the parachain blob consisting of the parachain block, the PoV block and set of outgoing messages from the parachain is available for a while.
The naive solution for this would be broadcasting/gossip the parachain blobs to all relay chain nodes, which is not a feasible option because the are are many parachains and the PoV blocks may be big.
We want to find an efficient solution to ensure PoV blocks from any recently created parachain blocks are available.

For a single chain, such as bitcoin, as long as 51\% of hash power is honest, not making block data available ensures that an honest miner builds on it so it will not be in the final chain. However, parachain consensus for us is determined by relay chain consensus.
A parachain block is canonical when it's header is in the relay chain.
We have no guarantees that anyone other than the collator and parachain validators have seen the PoV block.
If these collude then the rest of the parachain network need not have it and then most collators cannot build a new block and its invalidity may not be discovered.
We would like the consensus participants, here the validators, to collectively guarantee the availability rather than relying on a few nodes.

To this end we designed an availability scheme that uses erasure coding \cite{} to distribute the parachain PoV to all validators.
When any misbehaviour, unavailability particularly in relation to invalidity, is detected the PoV can be reconstructed from the distributed erasure coded pieces.

If a block is available, then full nodes of the parachain, and any light client that has the PoV block can check its validity. We have three-level of validity check in Polkadot. The first validity check of a blob is executed by the corresponding parachain validators. If they verify the blob then they sign and distribute the erasure codes of the blob to each validator.
We rely on nodes acting as {\em fishermen} to report invalidity as second level of validity check. They would need to back any claim with their own stake in DOTs. We would assume that most collators would be fishermen, as they have a stake in continued validity of the chain and are already running full nodes, so all they need is stake in DOTs. The third level of validity check is executed by a few randomly and privately assigned validators just after a block containing the blob header is created in the relay chain. We determine the number of validator in the third check considering amount of fishermen reports and unavailability reports by collators. If  any invalid blob is detected in the relay chain, the validators who signed for its validity are slashed.

The security of our availability and validity scheme based on the security of GRANDPA finality gadget (see Section \ref{sec:grandpa}) and the good randomness generated in each BABE epoch (see Section \ref{sec:babe}). Please see \cite{availandvalid} for more details about the availability and validity scheme.


\subsubsection{Cross Chain Messaging Protocol (XCMP)} \label{sec:XCMP}
XCMP is the protocol that parachains use to send messages to each other. It aims to guarantee that messages arrive quickly, that messages from one parachain arrive to another in order,
and that messages that arrive were indeed sent in the finalized history of the chain.

As a result of these properties, we will need to require that a parachain accepts all incoming messages.
The way relay chain blocks include headers of parachain blocks gives a synchronous notion of time for parachain blocks,
just by relay chain block numbers. Additionally it allows us to authenticate messages as being sent in the history given by the relay chain
i.e. it is impossible that one parachain sends a message, then reorgs \footnote{reorganisation of the chain} so that that a message was not sent, but has been received. This holds even though the system may not have reached finality over whether the message was sent, because any relay chain provides a consistent history.
To this end, parachain headers contain a message root of outgoing messages, as well as a bitfield indicating which other parachains were sent messages in this block.
The message root is the root of a Merkle tree of message roots for each parachain that messages are sent to,
and these in turn are the root of a Merkle tree of the head of a hash chain that has hashes of the actual messages. This allows many messages from this source to one particular destination  to be verified at once from the message root.
However the messages themselves are passed, they should also be sent with the Merkle proof that allows nodes of the receiving parachain
to authenticate that they were send by a particular parachain block whose header was in a particular relay chain block. Parachains act on incoming messages in order. It always acts on messages sent by parachain blocks whose header was in earlier relay chain blocks first. When several such parachains have a header in the relay chain block, they are acted on in some predetermined order, either sequentially in order of increasing parachain id, or some shuffled version of this. It acts on all messages sent by one parachain in one parachain block or none of them. A parachain header contains a reference to the relay chain block, by block hash or block number, that the parachain block receives all messages up to and to the parachain id.
This will likely be constrained to advance in some manner. To produce a parachain block on parachain $\Par$ which builds on a particular relay chain block $B$,
a collator would need to look at which parachain headers were built between the relay chain block that the last parachain block of this chain built on,
and for each of those that indicated that they sent messages to $\Par$, it needs the corresponding message data.
Thus it can construct a PoV block so that the (STVF) can validate that all such messages were acted on. Since a parachain must accept all messages that are sent to it,
we implement a method for parachains to make it illegal for another parachain to send it any messages that can be used in the case of spam occurring. If any node is connected to both parachain networks, then it should forward messages and their proofs from one parachain to the other,
when the parachaim header is included in a relay chain block. The relay chain should at least act as a back up.
The receiving parachain validators are connected to the parachain network and if they do not receive messages on it, then they can ask for them from the parachain validators of the sending chain at the time the message was sent.

