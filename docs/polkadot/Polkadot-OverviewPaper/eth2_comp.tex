Ethereum 2.0 promises a partial transition to proof-of-stake and to deploy sharding to improve speed and throughput.  There are extensive similarities between the Polkadot and Ethereum 2.0 designs, including similar block production and finality gadgets.  

All shards in Ethereum 2.0 operate as homogeneous smart contract based chains, while parachains in Polkadot are independent heterogeneous blockchains, only some of which support different smart contract languages.  
At first blush, this simplifies deployment on Ethereum 2.0, but ``yanking'' contracts between shards dramatically complicates the Ethereum 2.0 design.  We have a smart contract language Ink! that exists so that smart contract code can more easily be migrated into being parachain code.  We assert that parachains inherent focus upon their own infrastructure should support higher performance far more easily than smart contracts.

Ethereum 2.0 asks that validators stake exactly 32 ETH, while Polkadot fixes one target number of validators, and attempts to maximise the backing stake with NPOS (See \S~\ref{sec:validators}).  At a theoretical level, we believe the 32 ETH approach results in validators being less ``independent'' than NPoS, which weakens security assumptions throughout the protocol.  We acknowledge however that Gini coefficient matters here, which gives Ethereum 2.0 an initial advantage in ``independence''.  We hope NPoS also enables more participation by dot holders will balances below 32 ETH too.

Ethereum 2.0 has no exact analog of Polkadot's availability and validity protocol (See \S~\ref{sec:validity-and-availability}).  We did however get the idea to use erasure codes from the Ethereum proposal \cite{availabilityETH2}, which aims at convincing lite clients.  
% TODO: Jeff: I dislike how this second part is written but I'm not going to rephrase it right now.
Validators in Ethereum 2.0 are assigned to each shard for attesting block of shards as parachain validators in Polkadot thus constitute the committee of the shard. The committee members receive a Merkle proof of randomly chosen code piece from a full node of the shard and verify them. If all pieces are verified and no fraud-proof is announced, then the block is considered as valid. The security of this scheme is based on having an honest majority in the committee while the security of Polkadot's scheme based on having at least one honest validator either among parachain validators or secondary checkers (See \S~\ref{sec:validity-and-availability}). Therefore, the committee size in Ethereum 2.0 is considerably large comparing to the size of parachain validators in Polkadot. 
% TODO: \cite{ByzCoin} as analogous security propertties here maybe?  Or do we talk about them elsewhere?

% TODO: Jeff: Could this last paragraph be folded into the first?
The beacon chain in Ethereum 2.0 is a proof-of-stake protocol as Polkadot's relay chain. Similarly, it has a finality gadget called Casper \cite{CasperFFG,CasperCBC} as GRANDPA in Polkadot. Casper also combines  eventual finality and  Byzantine agreement as GRANDPA but GRANDPA gives better liveness property than Casper \cite{2018:Stewart:Grandpa}.
%TODO Ask about the details of better liveness to Al