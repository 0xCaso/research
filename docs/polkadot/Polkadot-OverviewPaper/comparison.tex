\section{Comparison with other multi-chain systems}\label{sec:comparison}
\paragraph{ETH2.0}
Ethereum 2.0 promises a partial transition to proof-of-stake and to deploy sharding to improve speed and throughput.  There are extensive similarities between the Polkadot and Ethereum 2.0 designs, including similar block production and finality gadgets.  

All shards in Ethereum 2.0 operate as homogeneous smart contract based chains, while parachains in Polkadot are independent heterogeneous blockchains, only some of which support different smart contract languages.  
At first blush, this simplifies deployment on Ethereum 2.0, but ``yanking'' contracts between shards dramatically complicates the Ethereum 2.0 design.  We have a smart contract language Ink! that exists so that smart contract code can more easily be migrated into being parachain code.  We assert that parachains inherent focus upon their own infrastructure should support higher performance far more easily than smart contracts.

Ethereum 2.0 asks that validators stake exactly 32 ETH, while Polkadot fixes one target number of validators, and attempts to maximise the backing stake with NPOS (See \S~\ref{sec:validators}).  At a theoretical level, we believe the 32 ETH approach results in validators being less ``independent'' than NPoS, which weakens security assumptions throughout the protocol.  We acknowledge however that Gini coefficient matters here, which gives Ethereum 2.0 an initial advantage in ``independence''.  We hope NPoS also enables more participation by DOT holders will balances below 32 ETH too.

Ethereum 2.0 has no exact analog of Polkadot's availability and validity protocol (See \S~\ref{sec:validity-and-availability}).  We did however get the idea to use erasure codes from the Ethereum proposal \cite{availabilityETH2}, which aims at convincing lite clients.  
% TODO: Jeff: I dislike how this second part is written but I'm not going to rephrase it right now.
Validators in Ethereum 2.0 are assigned to each shard for attesting block of shards as parachain validators in Polkadot thus constitute the committee of the shard. The committee members receive a Merkle proof of randomly chosen code piece from a full node of the shard and verify them. If all pieces are verified and no fraud-proof is announced, then the block is considered as valid. The security of this scheme is based on having an honest majority in the committee while the security of Polkadot's scheme based on having at least one honest validator either among parachain validators or secondary checkers (See \S~\ref{sec:validity-and-availability}). Therefore, the committee size in Ethereum 2.0 is considerably large comparing to the size of parachain validators in Polkadot. 
% TODO: \cite{ByzCoin} as analogous security propertties here maybe?  Or do we talk about them elsewhere?

% TODO: Jeff: Could this last paragraph be folded into the first?
The beacon chain in Ethereum 2.0 is a proof-of-stake protocol as Polkadot's relay chain. Similarly, it has a finality gadget called Casper \cite{CasperFFG,CasperCBC} as GRANDPA in Polkadot. Casper also combines  eventual finality and  Byzantine agreement as GRANDPA but GRANDPA gives better liveness property than Casper \cite{2018:Stewart:Grandpa}.
%TODO Ask about the details of better liveness to Al

\subsubsection*{Sidechains}
An alternative way to scale blockchain technologies are using side-chains \footnote{that allow tokens from one blockchain to be considered valid on an independent blockchain and be used there}. These solutions are also addressing interoperability, in that they enabling bridging side chains to the main chain. For example, for Eth1.0 many side-chains were introduced that contributed to scalability such as Loom \cite{} and Plasma.
A prominent solution that is solely based on bridging independent chains to each other is Cosmos \cite{} that is reviewed and compared to Polkadot next.

\paragraph{Cosmos}

Cosmos is a system designed to solve the blockchain interoperability problem that is fundamental to improve the scalability for the decentralised web. In this sense, there are surface similarities between the two systems. Hence, Cosmos consists of components which play similar roles and resemble the sub-components of Polkadot. For example, the Cosmos Hub is used to transfer messages between Comos' zones similarly to how the Polkadot Relay Chain oversees the passing of messages among Polkadot parachains.

There are however significant differences between the two systems. Most importantly, while the Polkadot system as a whole is a sharded state machine (See Section \ref{sec:relaychain}), Cosmos does not attempt to unify the state among the zones and so the state of individual zones is not reflected in the Hub's state. As the result, unlike Polkadot, Cosmos does not offer shared security among the zones. Consequently, the Cosmos cross-chain messages, are no longer trust-less. That is to say, that a receiver zone needs to fully trust the sender zone in order to act upon messages it receives from the sender. If one considers Cosmos system as a whole, including all zones in a similar way one analyses the Polkadot system, the security of such a system is equal to the security of the least secure zone. Similarly the security promise of Polkadot guarantees that validated parachain data are available at a later time for retrieval and audit (See Section \ref{sec:validity-and-availability}). In the case of Cosmos, the users are ought to trust the zone operators to keep the history of the chain state.

It is noteworthy that using the SPREE modules, Polkadot offers even stronger security than the shared security.
%(See Section \ref{sec:spree})
When a parachain signs up for a SPREE module, Polkadot guarantees that certain XCMP messages received by that parachain are being processed by the pre-defined SPREE module set of code. No similar cross-zone trust framework is offered by the Cosmos system.

Another significant difference between Cosmos and Polkadot consists in the way the blocks are produced and finalised. In Polkadot, because all parachain states are strongly connected to relay chain states, the parachain can temporarily fork alongside the relay chain. This allows the block production to decouple from the finality logic. In this sense, the Polkadot blocks can be produced over unfinalised blocks and multiple blocks can be finalised at once. On the other hand, the Cosmos zone depends on the instant finality of the Hub's state to perform a cross-chain operation and therefore a delayed finalisation halts the cross-zone operations.

\paragraph{BITCOIN or ETH1.0 SIDECHAIN}
TODO