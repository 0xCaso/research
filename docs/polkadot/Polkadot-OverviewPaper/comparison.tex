\subsection{Comparison with other multi-chain systems}\label{sec:comparison}

\paragraph{Cosmos} 

Similar to Polkadot, Cosmos is a system which aims as solving blockchain interoperability problem to improve scalability. In this sense, there are many similarities between the two systems such as components which play similar roles as sub-components of Polkadot, For example Cosmos's Hub roles resembles' of Polkadot Relay chain. Or similar to Polkadot parachain, Cosmos's zone are the blockchains which use the Hub to communicate. There are however significant differences between the two systems. 

Most importantly while Polkadot system as whole is a sharded state machine (See Section \ref{sec:relaychain}), Cosmos does not attempt to unify the state among zones and therefore individual zone's state is not reflected in the Hub's state. As the result, unlike Polkadot, Cosmos does not offer shared security among the zones. In such a system, the cross-chain messages are no longer trust-less, That is to say that a receiver zone need to fully trust the sender zone in order to act upon messages it receives from the sender. If one consider Cosmos system as whole including all zones in a similar way one analyzes the Polkadot system, the security of such system is equal to the security of least secure zone. Similarly the security promise of Polkadot guarantees that validated parachain data are available in later time for retrieval and audit (See Section \ref{sec:validity-and-availability}). In the case Cosmos, users are ought to trust the Zone operators to keep the history of the chain state.

It is noteworthy that using SPREE modules, Polkadot offers even stronger security than shared security (See Section \ref{sec:spree}. When a parachain signs up for a SPREE module, Polkadot guarantees that certain XCMP messages received by that parachain are being processed by the  pre-defined SPREE module set of code. No similar cross-zone trust framework is offered by Cosmos system.

Another significant difference between Cosmos and Polkadot is that on how blocks are produced and finalized. In Polkadot because all parachain states are strongly connected to relay chain states, parachain can temporarily fork alongside the relay chain. This allows for block production to decouple from finality logic. In this sense, Polkadot blocks can be produced over unfinalized blocks and multi blocks can be finalized at once. One the other hand, Cosmos zone depends on instant finality of Hub states to perform cross-chain openation and therefore delayed finalization halts the cross zone operations.
