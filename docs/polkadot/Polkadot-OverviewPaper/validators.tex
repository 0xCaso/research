\subsection{Validators (Network Maintainers)}\label{sec:validators}
 \paragraph{Keys}

 \subsubsection{NPoS mechanism for validator selection}

 The Polkadot network selects a new set of validators at the beginning of each new era. We denote by \nval the target number of validators to be selected, which is in the order of hundreds or thousands and is proportional to the number of parachains.


A validator node should satisfy certain requirements of speed, responsiveness and security. Any DOT holder who is up to the task can submit their candidacy to become a validator. If selected, they will have their stake locked and they will be economically compensated for their work, or slashed in the eventual case of a misconduct (see Section~\ref{sec:economics}). Additionally, any DOT holder can participate in the security of the network as a \emph{nominator}. Each nominator publishes a list (of any size) of the validator candidates that she trusts, together with an amount of DOTs that she is willing to lock. If one or more of her trusted candidates is elected during an era, she will share with them the validation rewards or slashings on a pro-rata basis. 

Unlike the case for validators, there can be an unlimited number of nominators (passively) participating in any given era. This approach allows for a massive amount of DOTs to be staked, much larger than the DOTs owned solely by validators, which increases the security of the network. In fact, it is expected that a significant percentage of all available DOTs will be routinely locked in the NPoS mechanism.

\textbf{The validator election problem.} Formally, the validator election problem is a multi-winner election problem based on approval ballots, where nominators have a voting power proportional to their stake, and each nominator submits a list of validator candidates that she supports. A solution consists of a committee of $k$ validators, together with a distribution of each nominator's stake among the elected validators that she backs. We consider two objectives, both of which have been recently introduced in the literature of computational social choice. The first one is achieving the property of \emph{proportional justified representation} (PJR). The second objective, called \emph{maximin support}, is maximizing the minimum amount of stake that backs any elected validator. The former objective aligns with the notions of decentralization among validators as well as user satisfaction in the platform, while the latter aligns with the security level of the consensus protocol. 

A validator node should satisfy certain requirements of speed, responsiveness and security. And DOT holder who is up to the task can submit their candidacy to become a validator, and they will be compensated for it, or slashed in the eventual case of a misconduct; see Section~\ref{sec:economics}.

This setup leads to a multi-winner election problem based on approval ballots, where nominators have voting power proportional to their stake, and each nominator submits a list of validator candidates that she supports. A solution consists of a committee of $k$ validators, together with a distribution of each nominator's stake among the elected validators that she backs. We consider two objectives, both of which have been recently introduced in the literature of computational social choice. The first one is achieving the property of \emph{proportional justified representation} (PJR). The second objective, called \emph{maximin support}, is maximizing the minimum amount of stake that backs any elected validator. The former objective aligns with the notions of decentralization among validators as well as user satisfaction in the platform, while the latter aligns with the security level of the consensus protocol.


We motivate the pursuit of these two objectives for the selection of validators in Polkadot as wells as any other decentralized protocol based on proof of stake with delegation. We also provide several constant-factor approximation algorithms for the maximin support objective, as well as a hardness result showing that these are, in a sense, best possible. Finally, we present a efficient algorithm which, when executed as a post-computation for any approximation algorithm for maximin support, ensures the PJR property while also preserving the approximation guarantee. Consequently, we provide the first efficient algorithms that simultaneously achieve PJR and constant-factor approximation guarantees for maximin support.
