\subsection{Nominated proof-of-stake and validator election}\label{sec:validators}
% The role of the validator is central to Polkadot, and as such validators need to run costly operations,
% ensure high communication responsiveness, and build long-term reputation of reliability.
% They are heavily staked and prone to slashing in case of offenses,
% but are otherwise highly remunerated by the network.
Polkadot will use Nominated Proof-of-Stake (NPoS), our very own version of proof-of-stake (PoS).
Consensus protocols with deterministic finality, such as the one in Polkadot, 
require a set of registered validators of bounded size.
Polkadot will maintain a number $\nval$ of validators, in the order of hundreds or thousands.
This number will be ultimately decided by governance, and is intended to grow linearly with the number of parachains;
yet it will be independent of the number of users in the network, thus ensuring scalability.
However, NPoS allows for an unlimited number of DOT holders to participate as \emph{nominators},
who help maintain high levels of security by putting more value at stake.
As such, NPoS is not only much more \emph{efficient} than proof-of-work (PoW),
but also considerably more \emph{secure} than standard PoS.
Furthermore, we introduce new guarantees on \emph{decentralization} hitherto unmatched by any other PoS-based blockchain.

A new set of validators is elected at the beginning of every \emph{era} -- a period during roughly one day (see Table~\ref{t:time} in the Appendix) --
to serve for that era, according to the nominators' preferences.
More precisely, any DOT holder may choose to become a validator candidate or a nominator.
Each candidate indicates the amount of stake he is willing to stake and his desired commission fee for operational costs.
In turn, each nominator locks some stake and publishes a list with any number of candidates that she trusts.
Then a public protocol , discussed below, takes these lists as input and elects the candidates
with the most backing to serve as validators for the next era.

Nominators share the rewards, or eventual slashings, with the validators they nominated, on a per-staked-DOT basis; 
see \autoref{sec:economics} for more details. 
Nominators are thus economically incentivized to act as watchdogs for the system, and they will base their preferences 
on parameters such as validators' staking levels, commission fees, past performance, and security practices.
Our scheme allows for the system to elect validators with massive amounts of aggregate stake
- much higher than any single party's dot holdings -
and thus helps turn the validator election process into a meritocracy rather than a plutocracy.
In fact, at any given moment we expect there to be a considerable fraction of all the dot supply be staked in NPoS.
This makes it very difficult for an adversarial entity to get validators elected 
-- as it either needs a large amount of dots or high enough reputation to get the required nominators' backing -- 
as well as very costly to attack the system -- as it is liable to lose all of its stake and its earned reputation.

Polkadot elects validators via a decentralized protocol with carefully selected, simple and publicly known rules,
taking the nominators' lists of trusted candidates as input. Formally, the protocol solves a multi-winner election
problem based on approval ballots, where nominators have voting power proportional to their stake,
and where the goals are \emph{decentralization} and \emph{security}.

\paragraph{Decentralization.} In the late 19th century, Swedish mathematician Edvard Phragm\'{e}n
proposed a method for electing members to his country’s parliament~\cite{brill2017phragmen}.
He noticed that the election methods at the time tended to give a disproportionately large number of seats 
to the most popular political party; in contrast, his new method ensured that the number of seats
assigned to each party was proportional to the votes given to them, so it gave more representation to minorities.

In the literature of computational social choice, Phragm\'{e}n's method has been recently revisited, 
algorithmically optimized, and shown to achieve a property called \emph{proportional justified representation} (PJR), 
see \cite{sanchez2017proportional, brill2017phragmen}.
In the context of NPoS, this property turns out to be ideal to guarantee decentralization.
It ensures that the set of selected validators represents as many nominator minorities as possible,
proportional to their stake, and that no minority is unjustly under-represented.
Such minorities may naturally form due to political views, economic interests, geographical locations, etc.

Our validator selection protocol is an adaptation of Phragm\'{e}n's method and is guaranteed 
to observe the property of proportional justified representation. 
Formally, this means that if each nominator $\nom \in \Nom$ has stake $stake_\nom$ 
and backs a subset $\Can_\nom\subseteq \Can$ of candidates,%
%
\footnote{For ease of presentation, we consider here a model where candidates have no stake of their own. 
The general case can be reduced to this model by representing each candidate's stake as an additional nominator 
that exclusively nominates that candidate.} %
%
the protocol will elect a set $\Val\subseteq \Can$ of $\nval$ validators such that, 
if there is a minority $\Nom'\subseteq \Nom$ of nominators such that %
%
$$|\cap_{\nom\in \Nom'} \Can_n| \geq t \quad \text{ and } \quad
\frac{1}{t} \sum_{\nom\in \Nom'} stake_\nom \geq \frac{1}{\nval} \sum_{\nom\in \Nom} stake_\nom,$$
%
for some $1\leq t\leq \nval$, then $|\Val\cap (\cup_{\nom\in\Nom'} \Can_n)| \geq t$.
In words, if a minority $\Nom'$ has at least $t$ commonly trusted candidates, 
to whom it could "afford" to provide with an average support of at least 
$\frac{1}{\nval} \sum_{\nom\in \Nom} stake_\nom$ 
(which in turn is an upper bound on the average validator support in the elected set $\Val$), 
then this minority has a justified claim to be represented in $\Val$ by at least $t$ candidates 
(though not necessarily commonly trusted).

\paragraph{Security.} If a nominator gets two or more of its trusted candidates elected as validators,
the protocol must also establish how to split her stake and assign these fractions to them.
In turn, these assignations define the total stake support that each validator receives.
Our objective is to make these validators' supports as high and as balanced as possible.
In particular, we focus on maximizing the \emph{minimum validator support}.
Intuitively, the minimum support corresponds to a lower bound on the cost for an adversary to gain control
over one validator, as well as a lower bound on the slashable amount for a misconduct.

Formally, if each nominator $\nom \in \Nom$ has $stake_\nom$ and backs a candidate subset $\Can_\nom\subseteq \Can$,
the protocol must not only elect a set $\Val\subseteq \Can$ of $\nval$ validators
with the PJR property, but also define a distribution of each nominator's stake among the elected validators that she backs,
i.e. a function $f:\Nom\times\Val \rightarrow \mathbb{R}_{\geq 0}$ so that
$$\sum_{\val\in \Val\cap \Can_\nom} f(\nom, \val) = stake_\nom \quad \text{ for each nominator } \nom\in\Nom,$$
and the objective is then
$$\max_{(\Val, f)} \: \min_{\val\in \Val} \: support_f(\val),
\quad \text{ where } support_f(\val) := \sum_{\nom\in \Nom: \ \val\in \Can_\nom} f(\nom, \val). $$

The problem defined by this objective is called \emph{maximin support} in the literature~\cite{sanchez2016maximin}, and is known to be NP-hard.
We have developed for it several efficient algorithms which offer theoretical guarantees 
(constant-factor approximations), and also scale well and have been successfully tested on our testnet. 
\alfonso{These results will be published in a separate paper soon.}{Once the NPoS paper is ready, we should put it on arXiv and add a reference here.}
