\subsection{Networking}\label{sec:networking}

In the preceding sections we talk about nodes sending data to another node or
other set of nodes, without being too specific on how this is achieved. We do
this to simplify the model and to clearly delineate a separation of concerns
between different layers.

Of course, in a real-world decentralised system the networking part also must
be decentralised - it's no good if all communication passes through a few
central servers, even if the high-level protocol running on top of it is
decentralised with respect to its entities. As a concrete example: in certain
security models, including the traditional Byzantine fault-tolerant setting,
nodes are modelled as possibly malicious but no consideration is given to
malicious edges. A security requirement like “$> 1/3$ of nodes are honest” in
the model, in fact translates to “$> 1/3$ of nodes are honest and can all
communicate perfectly reliably with each other all the time” in reality.
Conversely, if an edge is controlled by a malicious ISP in reality, it is the
corresponding node(s) that must be treated as malicious in any analysis under
the model. More significantly, if the underlying communications network is
centralised, this can give the central parties the ability to “corrupt” $> 1/3$
of nodes within the model thereby breaking its security assumptions, even if
they don't actually have arbitrary execution rights on that many nodes.

In this section we outline and enumerate the communication primitives that we
require in Polkadot, and sketch a high-level design on how we achieve these in
a decentralised way, with the specifics to be refined as we move forward with a
production system.

\subsubsection{Networking overview}

Polkadot needs networking subprotocols for a number of its components as follows.
\begin{enumerate}
\item synchronising newer state for (a) parachains and (b) the relay chain
\item accepting transactions from users for (a) parachains and (b) the relay chain
\item collation protocol for parachain blocks, including distributing transactions
\item accepting collated parachain blocks from parachain collators
\item distributing parachain blocks and validity attestations, and making these available for a time \ref{sec:validity-and-availability} for auditing purposes
\item collation protocol for relay chain blocks \ref{sec:relaychainblockproduction}, including distributing transactions
\item finalisation protocol for relay chain blocks \ref{sec:grandpa}, including distributing relay chain blocks
\item passing messages between different parachains \ref{sec:XCMP}
\end{enumerate}

1(a), 2(a) and 3 are conceptually external to Polkadot, being chosen by each parachain for themselves as their own responsibility. However, schemes similar to the ones described below for the relay chain, may also be used by parachains if they decide they are suitable. Additionally, (3) is expected to require information about the relay chain to work. This is public information, and the collators should get it simply by being part of the relay chain gossip network, as an untrusted non-validator full-node.

For the remaining components, the following subprotocols are used:

\begin{enumerate}
\item (a) N/A - external to Polkadot, (b) Performed with one's \hyperref[sec:gossiping]{gossip neighbours}
\item (a) N/A - external to Polkadot, (b) Sender \hyperref[sec:net_service]{sends non-specifically}
\item N/A - external to Polkadot
\item Sender \hyperref[sec:net_service]{sends non-specifically}
\item Special case, see \hyperref[sec:net_storage]{below}
\item Broadcast via \hyperref[sec:gossiping]{gossip}
\item Broadcast via \hyperref[sec:gossiping]{gossip}
\item Special case, see \hyperref[sec:net_crosschain]{below}
\end{enumerate}

We talk about these in more detail in the next few sections. Finally, we talk about the lower layers underpinning all of these subprotocols, namely \nameref{sec:net_lowlevel}.

\subsubsection{Gossiping} \label{sec:gossiping}

This subprotocol is used for most relay-chain artifacts, where everyone needs to see more-or-less the same public information. Part of its structure is also used for when a node goes offline for a long time and needs to synchronise any newer data it hasn't seen before.

The Polkadot relay chain network forms a gossip overlay network on top of the physical communications network, as an efficient way to provide a decentralised broadcast medium. The network consists of a known number of trusted nodes (validators) who have been permissioned via staking, and an unknown number of untrusted nodes (full nodes that don't perform validation) from the permissionless open internet. As an aside, note that some of the untrusted nodes are expected to have other roles in the system, e.g. be a parachain collator, fishermen, etc.

Currently a naive flooding approach is implemented, with plans to make this more secure and efficient later on, as follows:

The trusted nodes will form a structured overlay with a known, deterministic, but unpredictible topology that rotates every era. The untrusted nodes will form an unstructured overlay around the structured core, optimised via latency measurements but in a more sophisticated and secure way than mere "closest first". Both topologies are chosen to mitigate eclipse attacks, as well as sybil attacks in the permissionless case.

The gossip protocol will also take into account various constraint rules from the higher-level protocols above, to avoid broadcasting obsolete or otherwise unneeded artifacts. For example, for GRANDPA we only allow two votes being received for each type of vote, round number, and voter; any further votes will be ignored. For block production only valid block producers are allowed to produce one block per round; any further blocks will be ignored.

The gossip protocol will also include some sort of set reconciliation protocol to reduce redundancy when many senders attempt to send the same object to the same recipient at once.

We also plan to use \emph{Sentry nodes} that are proxy servers who receive all the traffic that would go to a certain validator and forward the traffic as soon as the validator is able to receive it.

\subsubsection{Non-specific direct sending} \label{sec:net_service}

This subprotocol is used when some part of Polkadot is providing a service to some external entity, namely (2b) accepting relay chain transactions and (3) accepting collated blocks in the list above.

In our naive initial version this simply involves looking up a particular target set in the address book, selecting a few nodes from this set, and connecting to them. We have a few issues to consider:

\begin{itemize}
\item How to store and lookup non-specific targets in the address book in a secure and balanced way, and what the representation of the query term(s) should be.
\item Load-balancing actual transport-layer connection attempts over the whole target set, and ensuring availability.
\end{itemize}

\subsubsection{Storage and availability} \label{sec:net_storage}

This subprotocol is for \nameref{sec:validity-and-availability}.

Recall that for scalability, Polkadot does not require everyone to store the state of the whole system, namely all of the state pointed to by all of the blocks. Instead, every parachain block is split into pieces by erasure-coding, such that there is 1 piece for every validator for a total of $N$ piecese, the erasure threshold being $ceil(N/3)$ blocks for security reasons explained elsewhere. All the pieces are available initially at the relevant parachain validators, having been submitted by some of the collators. (In this role, the parachain validators are also called the \emph{primary checkers}.) The pieces are then selectively distributed in the following phases:

\begin{enumerate}
\item Initial distribution - every validator initially wants 1 of these pieces, and the parachain validators must distribute them
\item Secondary checks - the \emph{secondary checkers} all want any $ceil(N/3)$ of the pieces, and can get this from anyone.
\item Yet later, and optionally - other non-validator parties might also want to perform further checks, e.g. in response to fishermen alerts, and again will want any $ceil(N/3)$ of the pieces.
\end{enumerate}

This subprotocol therefore aims to ensure that the threshold is available and can be retrieved from the relevant validators for some reasonable amount of time, until at least the latter phases are complete. We will follow a bittorrent-like protocol with the following differences:

\begin{itemize}
\item The initial distribution can be pre-emptively pushed from the primary checkers to the validators. This will take part in two stages:
\begin{enumerate}
\item Primary distribution - each of the C primary checkers, each send $N/C$ pieces directly to the corresponding validators.
\item Secondary distribution - sometimes direct sending will fail due to various network anomalies such as NAT, in which case this method acts as a backup, involving sending pieces to some gossip neighbours of the unreachable nodes.
\end{enumerate}
\item Instead of a centralised tracker, tracker-like information such as who has what piece, is broadcast via the relay chain gossip network.
\end{itemize}

The primary checkers are expected to be fully- or mostly-connected; this is a pre-existing requirement for the collation protocol to work well. To help the secondary checks complete faster, the secondary checkers will also be fully- or mostly-connected.

Beyond this, nodes may communicate to any other nodes as the protocol sees fit, similar to bittorrent. To protect against DoS attacks they should implement resource constraints as in bittorrent, and furthermore nodes should authenticate each other and only communicate with other validators, including the primary and secondary checkers. Non-validator parties in the latter optional phase will be supplied with an authentication token for this purpose.

There are various reasons why we do not consider it very suitable, to use a
structured overlay topology for this component:

\begin{enumerate}
\item Each piece is sent to specific people, rather than everyone.
\item
\begin{enumerate}
\item People that want a specific piece of data, know where to get it - i.e.
      validators, for their own piece, i.e. the primary validators.
\item Other people want non-specific pieces - i.e. secondary validators,
      want any 1/3 of all pieces to be able to reconstruct.
\end{enumerate}
\end{enumerate}

Overlay topologies are generally more useful for the exact opposite of the
above usage requirements:

\begin{enumerate}
\item Each data piece is sent to nearly everyone, or
\item People want a specific data piece, but don't know where to get it from.
\end{enumerate}

For example, bittorrent has similar requirements and does not use a structured
overlay either. The peers there connect to other peers on a by-need basis.

\subsubsection{Cross-chain messaging} \label{sec:net_crosschain}

This subprotocol is for \nameref{sec:XCMP}.

To recap that section, parachains can send messages to each other. The contents
of outgoing messages are initially stored by the sending parachain block, and
are contained in the parachain PoV blocks and therefore also initially stored
by the parachain validators and distributed to other validators as part of the
\hyperref[sec:net_storage]{availability protocol}. Relay chain blocks contain
metadata that describes the relevant outgoing messages corresponding to the
incoming messages for every parachain.

The job of XCMP networking therefore, is for each parachain to retrieve its
incoming messages from the outboxes. In our initial implementation of Polkadot
these will be distributed via the gossip network, starting from the parachain
validators of the outgoing messages. This is obviously not optimal, because
everyone will also receive messages destined for other people, that they don't
need. Further work will add more capabilities to our gossip network and
possibly other subprotocols, to support a more restricted and optimised
distribution.

\subsubsection{Sentry nodes} \label{sec:net_sentry}

Sometimes, network operators want to arrange aspects of their physical network
for various operational security related reasons. Some of these arrangements
are independent and compatible with the design of any decentralised protocol,
which typically works in the layer above. However some other arrangements need
special treatment by the decentralised protocol, in particular arrangements
affecting the reachability of nodes.

We don't believe that a public address is itself a security liability when the
serving code is written well. However another benefit of a restricted physical
topology, is to support load-balancing and DoS protection across a number of
gateway proxies. The indirection can help to protect the private node if there
are exploits in the software - although note this does not help for the most
severe exploits that give arbitrary execution rights on the sentry node, which
can then be used as a launching pad for attacks on the private node.

For such use-cases, Polkadot supports running full-nodes as the sentry nodes of
another full-node that is only reachable by these sentry nodes. This works best
when one runs several sentry nodes for a single private full-node. Protocol
wise, this requires some additional metadata to be stored in the address book,
and for additional proxying protocols to be used when we need to send data
directly between validators instead of over the gossip broadcast network. These
additions are fairly straightforward and are documented elsewhere.

It is not required to run sentry nodes, for example if you believe the
aforementioned security benefits are not worth the added latency cost.

(An alternative possibility is for the network operator to run lower-level
proxies, such as IP or TCP proxies, for their private node. This certainly can
be done without any protocol support from Polkadot. One advantage of sentry
nodes compared to this scenario is that the traffic coming from a sentry node
has been through some level of verification and sanitization as part of the
Polkadot protocol, which would be missing for a lower-level proxy. Of course
there can be exploits against this, but these are dealt with with high priority
since they are amplification vectors usable against the whole network.)

\subsubsection{Authentication, transport, and discovery} \label{sec:net_lowlevel}

In secure protocols in general, and likewise with Polkadot, entities refer to each other by their cryptographic public keys. There is no strong security association with weak references such as IP addresses since those are typically controlled not by the entities themselves but by their communications provider.

Nevertheless in order to communicate we need some sort of association between entities and their addresses. In Polkadot we use a similar scheme as many other blockchains do, that is using the widely used distributed hash table (DHT), Kademlia \cite{Maymounkov:2002:Kademila}. Kademlia is a DHT that uses XOR distance metric for finding a node and is often used for networks with high churn. We use Protocols Labs' libp2p libraries \cite{} Kademlia implementation with some changes for this purpose. To prevent Eclipse attacks \cite{eclipseattack} we allow for routing tables that are large enough to contain at least some honest nodes.

Currently, this \emph{address book} service is also used as the primary discovery mechanism - nodes perform random lookups on the space of keys when they join the network, and connect to whichever set of addresses are returned. Likewise, nodes accept any incoming connections. This makes it easy to support light clients and other unprivileged users, but also makes it easy to perform DoS attacks.

It is planned to develop a stronger authentication mechanism, where nodes authenticate themselves - their cryptographic identities and their roles as part of Polkadot - and verify other nodes both when (a) connections are mode, and when (b) looking up new nodes in the address book. Furthermore it should be possible to discover nodes that match a particular role, e.g. \emph{collator for parachain C} and so on. Nevertheless it is also important to retain the ability to accept incoming connections from unauthenticated entities, e.g. users submitting transactions, or previously-privileged nodes that have been offline for a while that now want to synchronise, and this needs to be restricted on a resource basis, without starving the authenticated entities.
