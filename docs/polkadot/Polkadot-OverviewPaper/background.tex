\section{Glossary and Background}

\eray{- state machine
- data structures}{-comment: Is this meant to be a list? It looks strange like this.-}


%\eray{
\begin{longtable}{p{.15\textwidth}p{.55\textwidth}p{.1\textwidth}p{.1\textwidth}} \label{t:time}
    \textbf{Name}  & \textbf{Description} & \textbf{Symbol} & \textbf{Def} \\
    \hline
    BABE & Assigns validators randomly to block production && \ref{sec:babe} \\
    BABE Slot & a period for which a relay chain block is produced & \slot & \ref{sec:babe} \\
    Block & Data-structure containing extrinsics & \block & \\
    Blockchain & A chain of blocks & \bchain & \\
    Candidate\newline- validators & A set of candidate validators & \Can & \\
    CD & \emph{Chain Density} - Ensures that a long portion of blocks was created by honest actors && \ref{item:chain_density} \\
    CG & \emph{Chain Growth} - Guarantees a minimum gowth between slots && \ref{item:chain_growth} \\
    Collator & Assist validators in block production & \col & \ref{par:collators} \\
    Collators & A set of collators & \Col & \\
    CP & \emph{Common Prefix} - Finalized blocks cannot be changed && \ref{item:common_prefix} \\
    CQ & \emph{Chain Quality} - Ensures honest block contribution to any best chain owned by a honest party&& \ref{item:chain_quality} \\
    DOT & A Polkadot native token && \ref{sec:economics} \\
    Elected\newline- validators & A set of elected validators & \Val & \\
    Epoch & a period for which randomness is generated by BABE & \ep & \\
    Era & a period for which a new validator set is decided && \\
    Extrinsics & Input data supplied to the Relay Chain to transition states && \ref{par:extrinsics} \\
    Fishermen & Monitors the network for misbehavior && \ref{par:fishermen} \\
    Gossiping & Broadcast every newly received message to peers && \ref{sec:gossiping} \\
    GRANDPA & Mechanism to finalize blocks && \ref{sec:grandpa} \\
    GRANDPA\newline- Round & a state of the GRANDPA algorithm which lead to && \ref{sec:grandpa} \\
    Kademlia & Distributed Hash Table with distance metrics && \ref{sec:auth_discovery} \\
    Local clock & A local clock used by validators & \lclock & \ref{par:network_coms} \\
    Nominator & Stake-holding party who nominates validators & \nom & \ref{par:nominators} \\
    Nominators & A set of nominators & \Nom & \\
    NPoS & \emph{Nominated Proof-of-Stake} - Polkadot's version of PoS, where nominated validators get elected for block production && \ref{sec:validators} \\
    Parachain & Heterogeneous independent chain & \Par & \\
    Phragmén & Mechanism using PJR for selecting validators && \ref{par:decentralization} \\
    PJR & \emph{proportional justified representation} - Ensures that validators represent as many nominator minorities as possible && \ref{par:decentralization} \\
    PoS & \emph{Proof-of-Stake} - Alternative to PoW, where parties vote with locked funds && \ref{sec:validators} \\
    PoV & \emph{Proof-of-Validity} - Mechanism where a validator can verify a block without having its full state && \ref{sec:parachainblockproduction} \\
    PoW & \emph{Proof-of-Work} - Mechanism where parties vote with processing power && \\
    Relay Chain & Ensures global consensus among parachains && \ref{sec:relaychain} \\
    RPC &  && \ref{par:decentralization} \\
    Runtime & The Wasm blob which contains the state transition functions, including other core operations required for Polkadot && \ref{par:state_transition} \\
    Sentry nodes & Specialized proxy server which forward traffic to/from the validator && \\
    STVF & \emph{state transition validation function} - A function of the Runtime to verify the PoV && \ref{sec:parachainblockproduction} \\
    Validator & The highest in charge party who seals new blocks & \val & \ref{par:validators} \\
    Validators to elect & Number of validators to elect & \nval & \\
    VRF public key & Key used for the VRF function & \pkvrf & \ref{sec:session_keys} \\
    VRF secret key & Key used for the VRF function & \skvrf & \ref{sec:session_keys} \\
    XCMP & A protocol that parachains use to send messages to each other && \ref{sec:XCMP} \\
\caption{Time periods used in Polkadot}
\end{longtable}
%}

\alfonso{}{I think the table should contain more information. I would add a) possibly longer descriptions, b) a reference to the section that introduces them (and where we give an even longer description + its reason of being), and c) their lengths in seconds/minutes/hours, where we put a big note saying all lengths are tentative and subject to change considerably.}
\alfonso{}{Also, we should either add "session" to the table, or remove all mentions of sessions. Simplifying could be a good idea, so maybe the latter?}