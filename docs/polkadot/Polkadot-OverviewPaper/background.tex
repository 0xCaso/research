\section{Glossary}



%\eray{
\begin{longtable}{p{.15\textwidth}p{.55\textwidth}p{.1\textwidth}p{.1\textwidth}} \label{t:time}
    \textbf{Name}  & \textbf{Description} & \textbf{Symbol} (plural)& \textbf{Def} \\
    \hline
    BABE & A mechanism to assign elected validators randomly to block production for a certain slot. && \ref{sec:babe} \\
    BABE Slot & A period for which a relay chain block can be produced. It's about 5 seconds. & \slot & \ref{sec:babe} \\
    Collator & Assist validators in block production. A set of collators is defined as \Col . & \col (\Col) & \ref{par:collators} \\
    DOT & The Polkadot native token. && \ref{sec:economics} \\
    Elected\newline- validators & A set of elected validators. & \Val & \\
    Epoch & A period for which randomness is generated by BABE. It's about 4 hours. & \ep & \\
    Era & A period for which a new validator set is decided. It's about 1 day. && \\
    Extrinsics & Input data supplied to the Relay Chain to transition states. && \ref{par:extrinsics} \\
    Fishermen & Monitors the network for misbehavior. && \ref{par:fishermen} \\
    Gossiping & Broadcast every newly received message to peers. && \ref{sec:gossiping} \\
    GRANDPA & Mechanism to finalize blocks. && \ref{sec:grandpa} \\
    GRANDPA\newline- Round & A state of the GRANDPA algorithm which leads to block finalisation. && \ref{sec:grandpa} \\
    Nominator & Stake-holding party who nominates validators to be elected. A set of nominators is defined as \Nom . & \nom (\Nom) & \ref{par:nominators} \\
    NPoS & \emph{Nominated Proof-of-Stake} - Polkadot's version of PoS, where nominated validators get elected to be able to produce blocks. && \ref{sec:validators} \\
    Parachain & Heterogeneous independent chain. & \Par & \\
    PJR & \emph{Proportional-Justified-Representation} - Ensures that validators represent as many nominator minorities as possible. && \ref{par:decentralization} \\
    %PoS & \emph{Proof-of-Stake} - Alternative to PoW, where parties vote with locked funds. && \ref{sec:validators} \\
    PoV & \emph{Proof-of-Validity} - Mechanism where a validator can verify a block without having its full state. && \ref{sec:parachainblockproduction} \\
    %PoW & \emph{Proof-of-Work} - Mechanism where parties vote with processing power. && \\
    Relay\newline- Chain & Ensures global consensus among parachains. && \ref{sec:relaychain} \\
    Runtime & The Wasm blob which contains the state transition functions, including other core operations required by Polkadot. && \ref{par:state_transition} \\
    Sentry\newline- nodes & Specialized proxy server which forward traffic to/from the validator. && \\
    Session & ?? && \\
    STVF & \emph{State-Transition-Validation-Function} - A function of the Runtime to verify the PoV. && \ref{sec:parachainblockproduction} \\
    Validator & The elected and highest in charge party who has a chance of being selected by BABE to produce a block. A set of candidate validators is defined as \Can . The number of validators to elect is defined as \nval . & \val (\Val)& \ref{par:validators} \\
    VRF & \emph{Verifiable-Random-Function} - Cryptographic function for determining elected validators for block production. && \ref{sec:babe} \\
    XCMP & A protocol that parachains use to send messages to each other. && \ref{sec:XCMP} \\
\caption{Glossary for Polkadot}
\end{longtable}
%}

%\alfonso{}{I think the table should contain more information. I would add a) possibly longer descriptions, b) a reference to the section that introduces them (and where we give an even longer description + its reason of being), and c) their lengths in seconds/minutes/hours, where we put a big note saying all lengths are tentative and subject to change considerably.}
%\alfonso{}{Also, we should either add "session" to the table, or remove all mentions of sessions. Simplifying could be a good idea, so maybe the latter?}
\section{Comparison with other multi-chain systems}\label{sec:comparison}
\paragraph{ETH2.0}

\subsubsection{Sidechains}
\paragraph{Cosmos} 

Much like Polkadot, Cosmos is a system designed to solve the blockchain interoperability problem that is fundamental to improve the scalability for the decentralized web. In this sense, there are surface similarities between the two systems. Hence, Cosmos consists of components which play similar roles and resemble the sub-components of Polkadot. For example, the Cosmos Hub is used to transfer messages between Comos' zones similarly to how the Polkadot Relay Chain oversees the passing of messages among Polkadot parachains.

There are however significant differences between the two systems. Most importantly, while the Polkadot system as a whole is a sharded state machine (See Section \ref{sec:relaychain}), Cosmos does not attempt to unify the state among the zones and so the state of individual zones is not reflected in the Hub's state. As the result, unlike Polkadot, Cosmos does not offer shared security among the zones. Consequently, the Cosmos cross-chain messages, are no longer trust-less. That is to say, that a receiver zone needs to fully trust the sender zone in order to act upon messages it receives from the sender. If one considers Cosmos system as a whole, including all zones in a similar way one analyses the Polkadot system, the security of such a system is equal to the security of the least secure zone. Similarly the security promise of Polkadot guarantees that validated parachain data are available at a later time for retrieval and audit (See Section \ref{sec:validity-and-availability}). In the case of Cosmos, the users are ought to trust the zone operators to keep the history of the chain state.

It is noteworthy that using the SPREE modules, Polkadot offers even stronger security than the shared security (See Section \ref{sec:spree}. When a parachain signs up for a SPREE module, Polkadot guarantees that certain XCMP messages received by that parachain are being processed by the pre-defined SPREE module set of code. No similar cross-zone trust framework is offered by the Cosmos system.

Another significant difference between Cosmos and Polkadot consists in the way the blocks are produced and finalized. In Polkadot, because all parachain states are strongly connected to relay chain states, the parachain can temporarily fork alongside the relay chain. This allows the block production to decouple from the finality logic. In this sense, the Polkadot blocks can be produced over unfinalized blocks and multiple blocks can be finalized at once. On the other hand, the Cosmos zone depends on the instant finality of the Hub's state to perform a cross-chain operation and therefore a delayed finalization halts the cross-zone operations.


\section{SPREE}

SPREE (Shared Protected Runtime Execution Enclaves) is a way for parachains to have shared code, and furthermore for the execution and state of that code to be sandboxed. From the paoint of view of parachain A, how much can it trust parachain B? Polkadot's shared security guarantees the correct execution of B's code with as much security as it does A's code. However, if we do not know B's code itself and even if we know the code now, maybe the governance mechanism of B can change the code and we do not trust that. This changes if we knew some of B's code, that it's governance did not have control of, and which could be sent messages by A. Then we would know how B's code would act on those messages if it was executed correctly and so shared security gives us the guarantees we need.

A SPREE module is a piece of code placed in the relay chain, that parachains can opt into. This code is part of that chains state transition validation function (STVF). The execution and state of this SPREE module are sandboxed away from the rest of the STVF's execution. SPREE modules on a remote chain can be addressed by XCMP. The distribution of messages received by a parachain would itself be controlled by a SPREE module (which would be compulsory for chains that want to use any SPREE modules). 

We expect that most messages sent by XCMP will be from a SPREE module on one chain to the same SPREE module on another chain. When SPREE modules are upgraded, which involves putting updated code on the relay chain and scheduling an update block number, it is upgraded on all parachains in their next blocks. This is done in such a way as to guarantee that messages sent by a version of the SPREE module one one chain to the same module on another are never recieved by past versions. Thus message formats for such messages do not need to be forward compatible and we do not need standards for these formats.

For an example of the security guarantees we get from SPREE, if A has a native token, the A token, what we would like is to be sure that parachain B could not mint this token. We could enforce this by A keeping an account for B in A's state. However if an account on B want's to send some A token to a third parachain C, then it would need to inform A. A SPREE module for tokens would allow this kind of token transfer without this accounting. The module on A would just send a message to the same module on B, sending the tokens to some account. B could then send them on to C and C to A in a similar way. The module itself would account for tokens in accounts on chain B, and Polkadot's shared security and the module's code would enforce that B could never mint A tokens. XCMP's guarantee that messages will be delivered and SPREE'S guarantee that they will be interpreted correctly mean that this can be done by just sending one message per transfer and is trust free. This has applications far beyond token transfer and means that trust minimising protocols are far easier to design.

Parts of SPREEs design and implementation have yet to be fully designed. Credit goes to u/Tawaren for the initial idea behind SPREE.
\section{Interoperability with External Chains}\label{sec:bridge}
