\section{Introduction}

Polkadot~\cite{burdges2020overview} is a blockchain network launched in 2020. It proposes a platform where several heterogeneous distributed applications may run as parallel shards, coordinated by a single consensus protocol that is maintained by a common set of validators. The platform focuses on scalability, cross-chain interoperability, and pooled security. 
It uses \emph{nominated proof-of-stake} (NPoS) which, like other proof-of-stake (PoS) based mechanisms, 
substitutes the inefficient proof-of-work (PoW) component in Nakamoto’s consensus protocol~\cite{nakamoto2019bitcoin}, 
and offers security in the presence of a dishonest minority of users, where this minority is to be measured relative to stake. 
In NPoS, validating rights can be delegated. More specifically, a user can not only aspire for the role of a validator, who is responsible for running the consensus protocol, 
but also take the role of a \emph{nominator}, who backs with her stake validators that she trusts, 
and shares with them economic rewards coming from the minting of new tokens, as well as eventual sanctions in case these validators show negligent or adversarial behavior.  
Nominators are thus economically incentivized to monitor the validators' performance and security practices, and keep their lists of trusted validators up to date.

Networks whose consensus protocol offers deterministic finality, by means of some version of Byzantine agreement, require a committee of registered validators of bounded size. This is the case for Polkadot, where the number of validators is likely to remain in the order of hundreds to low thousands. 
Bearing this in mind, we establish two important benefits of having the nominator role in the network. 
First, the NPoS mechanism allows for a massive number of users to participate as nominators, in contrast to the limit on the number of validators, thus allowing for an unlimited amount of stake to back validators, much higher than any single user's holding. Intuitively, the higher the stake backing the validators, the more \textbf{secure} the network is; we make this intuition formal further below. As such, NPoS is not only much more efficient than PoW, but also considerably more secure than standard PoS without delegation.

Second, nominators play a key role in the validator selection process, as we explain now. Once per era -- where an era lasts roughly one day -- a new committee of validators is elected from a set of candidates, taking into consideration the nominators' current lists of trusted candidates and their stake levels. Naturally, one of the objectives of this election process is to maximize the stake backing the elected validators, thus maximizing security. The other objective is to pick a \textbf{decentralized} committee, by which we mean one where nominators are represented proportional to their stake, with no minority being under-represented. 
We highlight here that nominators -- and their lists of trusted candidates -- constitute a valuable gauge for the preferences of the general community of network users, and that diverse preferences and factions will naturally arise for reasons that range from economic and security related to political, geographical, etc. Such diversity of points of view is expected and welcome in a decentralized platform, and it is important to engage all minorities in its processes of governance and decision making. 

In this paper we formalize this validator election and its two stated objectives, namely decentralization and security.
We consider that the trust that nominators have on validator candidates is binary, i.e.~each nominator provides an unranked list of approved candidates of any size. 
This leads a to vote-weighted, approval-based, multiwinner election scheme. 
There are public sets $N$ and $C$ of nominators and validator candidates respectively, and every nominator $n$ in $N$ provides a list $C_n\subseteq C$ of approved candidates, and has a vote strength equal to its stake $s_n$. There is also a target number $1\leq k<|C|$ of validators to elect.


\textbf{Decentralization.}
Our decentralization objective corresponds to the classical notion of \emph{proportional representation}, that is, that a committee should represent each minority in the electorate proportional to their aggregate vote strength, with no minority being under-represented. 
The goal of designing an electoral system for multiwinner elections that achieves proportional representation has been present in the literature of social choice for a very long time. Of special note is the work of Scandinavian mathematicians Edvard Phragm\'{e}n and Thorvald Thiele in the late nineteenth century \cite{phragmen1894methode, phragmen1895proportionella, phragmen1896theorie, phragmen1899till, thiele1895om, janson2016phragmen}. 
Several axioms have been recently proposed to define it mathematically; we mention the most relevant ones. 
\emph{Justified representation} (JR)~\cite{aziz2017justified} states that if a group $N'\subseteq N$ of voters is cohesive enough in terms of candidate preferences and has a large enough aggregate vote strength, then it has a justified claim to be represented by a member of the committee.
\emph{Proportional justified representation} (PJR)~\cite{sanchez2017proportional} says that such a group $N'$ deserves not just one but certain number of representatives according its vote strength, where a committee member is said to represent the group as long as it represents any voter in it.
Finally, \emph{extended justified representation} (EJR)~\cite{aziz2017justified} strengthens this last condition and requires not only that $N'$ have enough representatives as a group, but some voter in it must have enough representatives individually.
It is known that EJR implies PJR, and PJR implies JR, but converse implications are not true~\cite{sanchez2017proportional}. 
For each of these properties, we say that a committee voting rule satisfies said property if its output committee is always guaranteed to satisfy the property for any input instance. 

Among these properties, for the election of validators we set to achieve proportional justified representation (PJR), defined formally in Section~\ref{s:prel}, for two reasons. 
First, because it is more \emph{Sybil resistant}~\cite{douceur2002sybil} than JR, meaning that a strategic nominator may be incentivized to assume several identities under JR, but not under PJR. 
Second, because PJR seems to be most compatible with our security objective, as we discuss further below; in particular, our security objective turns out to be incompatible with EJR. 
While the most common voting rules usually satisfy JR, they fail to satisfy the stronger properties of PJR and EJR \cite{aziz2017justified, sanchez2017proportional}, 
and up to very recently there were no known efficient voting rules that satisfy the latter two. 
For instance, the \emph{proportional approval voting} (PAV) method \cite{thiele1895om, janson2016phragmen} proposed by Thiele satisfies EJR but is NP-hard to compute, while efficient heuristics based on it, such as reweighted approval voting, fail PJR \cite{aziz2014computational, skowron2016finding, aziz2017justified}. 
Recently, efficient algorithms that achieve PJR have finally been proposed \cite{brill2017phragmen, sanchez2016maximin}. 

\textbf{Security.}
In a PoS-based consensus protocol, we want a guarantee that as long as most of the stake is in the hands of actors that behave honestly or rationally, an adversary needs a large amount of stake to carry on an attack. 
If we assume that the attack requires control of some minimum number of validators to succeed, the adversary would need to use its stake to get these validators elected. The security level is hence proportional to \emph{how difficult it is for an entity to gain over-representation}, in an electoral system where vote strength is equated with stake. 
Interestingly, this is in stark contrast to the classical approach taken in proportional representation, that only seeks to avoid under-representation. 

Recall that each nominator $n\in N$ has a stake (or vote strength) $s_n$ and a list of approved candidates $C_n\subseteq C$. 
Suppose we want to make it as expensive as possible for an adversary to gain a certain number $1\leq r\leq k$ of representatives into the $k$-validator committee, and that few to none of the honest nominators trust these candidates. Then, our goal would be to elect a committee $A\subseteq C$ that maximizes $\min_{A'\subseteq A, |A'|=r} \sum_{n\in N: \ C_n\cap A'\neq \emptyset} s_n$.
This gives a different objective for each value of threshold $r$. 
For example, for $r=1$, maximizing this objective is equivalent to the classical multi-winner approval voting: selecting the $k$ candidates $c$ with highest total approval $\sum_{n\in N: \ c\in C_n} s_n$. 
If we are only concerned about a particular threshold, e.g. $r=\lceil(k+1)/3\rceil$ for a $34\%$ attack, then we can fix the corresponding objective. However, different types of attacks require different thresholds, and some attacks succeed with higher probability with more attacking validators. Hence, a more pragmatic approach is to incorporate the threshold into the objective, and maximize \emph{the least possible per-validator cost over all thresholds}, i.e.  
\begin{align}\label{eq:security}
    \text{Maximize } \min_{A'\subseteq A, A'\neq \emptyset} \quad \frac{1}{|A'|} \sum_{n\in N: \ C_n\cap A' \neq \emptyset} s_n, \quad \text{over all committees $A\subseteq C$ with $|A|=k$}.
\end{align}

We prove in Lemma \ref{lem:equivalence} that this objective is equivalent to the \emph{maximin support objective}, recently introduced by Sánchez-Fernández et al.~\cite{sanchez2016maximin}. 
To define this last objective, which we do formally in Section~\ref{s:prel}, one needs the election rule to establish not only a winning committee $A\subseteq C$, but also a \emph{vote distribution}; that is, a fractional distribution of each voter $n$'s vote strength $s_n$ among her trusted committee members in $C_n\cap A$ (the authors in~\cite{sanchez2016maximin} call this a \emph{support distribution function}). For instance, for voter $n$ the election rule may assign a third of $s_n$ to $c_1$ and two thirds of $s_n$ to $c_2$, where $c_1, c_2\in C_n\cap A$. 
The objective is then to maximize, over all possible committees and distributions, the least amount of vote assigned to any committee member. 
We observe here that, unlike most other applications of multiwinner elections, in Polkadot there is practical utility in computing a vote distribution from the nominators to the elected validators: by reversing its sense, it establishes the exact way in which the validators' payouts or penalties must be distributed back to the nominators at the end of the era.

Sánchez-Fernández et al.~\cite{sanchez2016maximin} remark that the maximin support objective defines an NP-hard problem, but do not study its approximability. They also prove that in its exact version it is equivalent to another objective, $\maxphragmen$, devised by Phragm\'{e}n and recently analyzed by Brill et al.~\cite{brill2017phragmen}, and in this last paper it is shown that $\maxphragmen$ is incompatible with the EJR property. Thus, the same incompatibility holds for our security objective. 
Interestingly enough, the authors in \cite{brill2017phragmen} and \cite{sanchez2016maximin} use $\maxphragmen$ and the maximin support objective, respectively, to design efficient heuristics that achieve the PRJ property. 

\textbf{Our contribution.}
The goals of decentralization and security for validator election under NPoS lead us into the analysis of the proportional justified representation (PJR) property and the maximin support problem, both recently discussed in the social choice literature. We present novel results that greatly increase our understanding of each objective and of their relation to each other. 
%
We start with a thorough complexity analysis of the maximin support problem in Section~\ref{s:complexity}, 
where we show it to be constant-factor approximable, as we exhibit several approximation algorithms for it, and prove as well that it does not admit a PTAS% 
    \footnote{A \emph{polynomial time approximation scheme} (PTAS) for an optimization problem is an algorithm that, for any constant $\eps>0$ and any given instance, returns a $(1+\eps)$-factor approximation in polynomial time.}%
unless P=NP so constant-factor approximations are best possible.

In Section~\ref{s:heuristic} we propose a new heuristic, $\phragmms$, inspired in $\phragmen$~\cite{brill2017phragmen} but allowing for more flexibility and better guarantees than the latter on both objectives. We use it to obtain the fastest known algorithm that simultaneously guarantees a) the PJR property and b) a constant-factor approximation for maximin support, where we remark that our application requires highly efficient algorithms, as instances can be massively large. 
%
Furthermore, in Section~\ref{s:local} we show that the output solution of our proposed algorithm can be tested by a \emph{verifier} to achieve the two aforementioned guarantees in \emph{linear time} (in the size of the input), even if the algorithm is privately executed by an untrusted \emph{prover} who only communicates the solution. Such efficient verification helps ensure transparency in a decentralized governance process, and might be of independent interest. 
%
In Section~\ref{s:local} we also adapt our heuristic into a post-computation that takes an arbitrary solution $A$ as input and returns a new solution $A'$ that a) is at least as good as $A$ in terms of the maximin support objective and b) provably satisfies PJR; the post-computation thus makes any approximation algorithm for maximin support also satisfy PJR in a black-box manner.

Finally, in Section~\ref{s:objectives} we propose a protocol for validation election under NPoS which exploits our results. 
In particular, several parties each run the election algorithm privately 
and post their solutions on the ledger, at which point the consensus protocol publicly ranks them and verifies them on-chain with our proposed test.  
This results in an efficient, transparent protocol with hard and verifiable guarantees. 
We conclude with numerical details on the performance of the validator election protocol currently running in Polkadot, which is a slight variant of the proposal presented here.