\section{Introduction}

Polkadot~\cite{wood2016polkadot} is a blockchain network to be launched in 2020. It proposes a platform where several heterogeneous distributed applications may run as parallel shards, maintained by a common set of validators and a single consensus protocol. The platform focuses on scalability, cross-chain interoperability, and pooled security. 

It uses \emph{nominated proof-of-stake} (NPoS) which, like other proof-of-stake (PoS) based mechanisms, substitutes the inefficient proof-of-work (PoW) component in Nakamoto’s consensus protocol~\cite{nakamoto2019bitcoin}, and offers security in the presence of a dishonest minority of users, where this minority is to be measured relative to stake. In NPoS, staking rights can be delegated. More specifically, users can not only assume the role of validators, who are responsible for running the consensus protocol, but also take the role of \emph{nominators}, who back with their stake validators that they trust, and share with them the economic rewards coming from the minting of new tokens, as well as any slashing in case of a detected adversarial behavior. Nominators are thus economically incentivized to act as watchdogs in the system, keeping track of the validators' performance and security practices, and keeping their lists of trusted validators up to date.

Networks whose consensus protocol offers deterministic finality, by means of some version of Byzantine agreement, require a committee of registered validators of bounded size. This is the case for Polkadot, where the number of validators is likely to be in the order of hundreds or low thousands. 
Bearing this in mind, we establish two important benefits of having the nominator role in the network. 
First, the NPoS mechanism allows for a massive number of users to participate as nominators, in contrast to the limit on the number of validators, thus allowing for an unlimited amount of stake to back validators, much higher than any single user's holding. Intuitively, the higher the stake backing the validators, the more \textbf{secure} the network is; we make this intuition formal further below. As such, NPoS is not only much more efficient than PoW, but also considerably more secure than standard PoS without stake delegation.

Second, nominators are key for the selection of validators, as we explain now. Once per era, each lasting tentatively one day, a new committee of validators is elected from a set of candidates, taking into consideration the nominators' current lists of trusted candidates and their stake levels. Naturally, one of the objectives of this election mechanism is to maximize the stake backing the elected validators, thus maximizing security. The other objective is to pick a \textbf{decentralized} committee, by which we mean one where nominators are represented proportional to their stake, with no minority being under-represented. 
We highlight here that nominators -- and their lists of trusted candidates -- constitute a valuable gauge for the preferences of the general community, and that diverse preferences and factions will naturally arise not only due to economical and security-related reasons, but also political, geographical, etc. Such diversity of points of view is expected and welcome in a decentralized community, and it is important to engage all minorities in processes of governance and decision making, to ensure users are satisfied and not leaving the network. 

We now formalize this validator election mechanism and its two stated objectives, neamely decentralization and security.
We consider that the trust that nominators have on validator candidates is binary, i.e.~each nominator provides an unranked list of approved candidates of any size. 
This leads a to vote-weighted, approval-based, multiwinner election scheme: 
There are public sets $N$ and $C$ of nominators and validator candidates respectively, and every nominator $n$ in $N$ provides a list $C_n\subseteq C$ of approved candidates, and has a vote strength equal to its stake $s_n$. There is also a target number $1\leq k<|C|$ of validators to elect.


\paragraph{Decentralization.}
Our decentralization objective translates into the classical notion of proportional representation. That is, a committee should represent each minority in the electorate proportional to their aggregate voting strength, with no minority being under-represented. 
The goal of designing an electoral system for multiwinner elections that achieves proportional representation has been present in the literature of social choice for a very long time. Of special note is the work of Scandinavian mathematicians Edvard Phragm\'{e}n and Thorvald Thiele in the late nineteenth century \cite{phragmen1894methode, phragmen1895proportionella, phragmen1896theorie, phragmen1899till, thiele1895om, janson2016phragmen}. 
Several axioms have been recently proposed to define proportional representation mathematically; we mention the most relevant ones. 
\emph{Justified representation} (JR)~\cite{aziz2017justified} states that if a group $N'\subseteq N$ of voters is cohesive enough in terms of candidate preferences and has a large enough aggregate vote strength, then it has a justified claim to be represented by a member of the committee.
\emph{Proportional justified representation} (PJR)~\cite{sanchez2017proportional} says that such a group $N'$ deserves not just one but some minimum number of representatives according its vote strength, where a committee member is said to represent the group as long as it represents any voter in it.
Finally, \emph{extended justified representation} (EJR)~\cite{aziz2017justified} strengthens this last condition and requires not only that $N'$ has enough representatives as a group, but also that some voter in it has individually enough representatives in the committee.
It is known that EJR implies PJR, and PJR implies JR, but converse implications are not true~\cite{sanchez2017proportional}. 
For each of these properties, we say that a committee voting rule satisfies said property if its output committee is always guaranteed to satisfy the property for any input instance. 

Among these properties, for our application to NPoS validator election we set to achieve proportional justified representation (PJR), defined formally in Section~\ref{s:prel}, for two reasons. 
First, because it is more Sybil resistant than JR, meaning that a strategic nominator may be incentivized to assume several identities under JR, but not under PJR. 
Second, because PJR seems to be most compatible with our security objective, as we discuss below. In particular, our security objective turns out to be incompatible with EJR. 

While the most common voting rules usually satisfy JR, they fail to satisfy the stronger properties of PJR and EJR \cite{aziz2017justified, sanchez2017proportional}, 
and up to very recently there were no known efficient voting rules that satisfy the latter two. 
For instance, the \emph{proportional approval voting} (PAV) method \cite{thiele1895om, janson2016phragmen} proposed by Thiele satisfies EJR but is NP-hard to compute, while efficient heuristics based on it, such as reweighted approval voting, fail PJR \cite{aziz2014computational, skowron2016finding, aziz2017justified}. 
Recently, Brill et al.~\cite{brill2017phragmen} and S{\'a}nchez-Fern{\'a}ndez et al.~\cite{sanchez2016maximin} have finally proposed efficient algorithms that achieve PJR. 

\paragraph{Security.}
In a PoS-based system, we want a guarantee that as long as most of the stake is in the hands of actors that behave honestly or rationally, an adversary needs a large amount of stake to carry on an attack. 
If we assume that the attack requires control of a certain number of validators to succeed, the adversary would need to use its stake to get these validators elected. The security level is hence proportional to \emph{how difficult it is for an entity to gain over-representation}, in an electoral system where the vote strength is equated with stake. 
Interestingly, this is in stark contrast to the standard approach for proportional representation, that only seeks to avoid under-representation. 

Recall that each nominator $n\in N$ has a vote strength $s_n$ and a list of approved candidates $C_n\subseteq C$. 
Suppose we want to make it as expensive as possible for an adversary to get a certain number $1\leq r\leq k$ of representatives into the $k$-validator committee, and that few to none of the honest nominators trust these candidates. Then, our goal would be to find a committee $A\subseteq C$ that maximizes $\min_{A'\subseteq A, |A'|=r} \sum_{n\in N: \ C_n\cap A'\neq \emptyset} s_n$.
This gives a different objective for each value of threshold $r$. 
For example, for $r=1$, maximizing this objective is equivalent to the classical multi-winner approval voting: selecting the $k$ candidates with highest total approval $\sum_{n\in N: \ c\in C_n} s_n$. 
If we are only concerned about a particular threshold, e.g. $r=\lceil(k+1)/3\rceil$ for a $34\%$ attack, then we can fix the corresponding objective. However, different types of attacks require different thresholds, and some attacks succeed with higher probability with more attacking validators. Hence, a more pragmatic approach is to incorporate the threshold into the objective, and maximize the least possible per-validator cost over all thresholds, i.e.  
\begin{align}\label{eq:security}
    \text{Maximize } \min_{A'\subseteq A, A'\neq \emptyset} \quad \frac{1}{|A'|} \sum_{n\in N: \ C_n\cap A' \neq \emptyset} s_n, \quad \text{over all committees $A\subseteq C$ with $|A|=k$}.
\end{align}

We prove in Lemma \ref{lem:maximin-support-eqiv} that this objective is equivalent to the \emph{maximin support objective}, recently introduced by Sánchez-Fernández et al.~\cite{sanchez2016maximin}. 
To define this last objective, which we do formally in Section~\ref{s:prel}, one needs the election mechanism to establish not only a winning committee $A\subseteq C$, but also a \emph{vote distribution}; that is, a fractional distribution of each voter $n$'s vote strength $s_n$ among her trusted committee members in $C_n\cap A$ (the authors in~\cite{sanchez2016maximin} call this a \emph{support distribution function}). For instance, for voter $n$ the election mechanism may assign a third of $s_n$ to $c_1$ and two thirds of $s_n$ to $c_2$, where $c_1, c_2\in C_n\cap A$. 
The objective is then to maximize, over all possible committees and distributions, the minimum amount of vote assigned to any committee member. 

We make two observations about this notion of vote distribution. First, it is often convenient to think of it as a flow over the bipartite approval graph with node set $N\cup C$, where nodes in $N$ have a net excess at most equal to their vote strength, and we are interested in the resulting net demands over $C$ which are called supports. In this paper we often take this approach and apply results from network flow theory. 
Second, and unlike most other applications of multiwinner elections, in Polkadot there is practical utility in establishing a vote distribution from the nominators to the winning validator committee: by reversing the sense of the flow, it establishes the exact way in which the validators' payouts or slashings must be distributed back into the nominators.

Sánchez-Fernández et al.~\cite{sanchez2016maximin} remark that the maximin support objective defines an NP-hard problem, but do not study its approximability. They also prove that in its exact version it is equivalent to another objective, $\maxphragmen$, devised by Phragm\'{e}n and recently analyzed by Brill et al.~\cite{brill2017phragmen}, and in this last paper it is shown that $\maxphragmen$ is incompatible with the EJR property. Thus, the same incompatibility holds for our security objective. Interestingly enough, the authors in~\cite{sanchez2016maximin} use the maximin support objective to design $\MMS$, an efficient heuristic that achieves the PJR property, while the authors in~\cite{brill2017phragmen} prove that $\phragmen$, an efficient heuristic based on the $\maxphragmen$ objective, also achieves PJR.

\paragraph{Our contribution.}
Our formalization of the decentralization and security goals for the NPoS validator election protocol leads us into the analysis of the proportional justified representation (PJR) property and the maximin support problem, both recently discussed in the social choice literature. In this paper we present novel results on each of these fronts, as well as election protocols that achieve guarantees on both objectives simultaneously.
%
We start with a thorough analysis of the complexity of the maximin support problem in Section~\ref{s:complexity}. We show that the problem is constant-factor approximable, and prove as well that it does not admit a PTAS% 
    \footnote{A \emph{polynomial time approximation scheme} (PTAS) for an optimization problem is an algorithm that, for any constant $\eps>0$ and any given instance, returns a $(1+\eps)$-factor approximation in polynomial time.}%
unless P=NP, so constant-factor approximations are best possible. In particular, we show that the $\MMS$ heuristic~\cite{sanchez2016maximin} provides a $2$-factor approximation. 

In Section~\ref{s:heuristic} we propose a new heuristic which, besides being fast, lends itself for a robust analysis both in terms of PJR and maximin support, and allows us to make the connection between these to objectives more evident. As an application, we use it to obtain a $3.15$-approximation algorithm for maximin support that is considerably faster than $\MMS$, where we remark that having the choice of faster protocols is of key utility for our application in Polkadot, as instances can be massively large. 

Furthermore, in Section~\ref{s:local} we use the new heuristic to develop a ``PJR enabler'': an efficient procedure that takes an arbitrary solution to the maximin support problem as input, and returns a new solution which a) preserves or improves the objective value of the input solution,  b) satisfies PJR, and in particular c) can be efficiently verified to satisfy PJR by an untrusting third party. Thus, by running this PJR enabler as a post-computation along any algorithm for maximin support, we obtain new algorithms that simultaneously achieve a constant-factor approximation for maximin support and provably ensure the PJR property. We also define a quantitative version of the PJR property, which more accurately describes the level of proportional representation achieved by a committee, and develop an efficient ``PJR test'' that computes (a bound on) the PJR level for any committee, as well as certifies (standard) PJR. 

Finally, in Section~\ref{s:objectives} we propose a possible implementation of the validation election protocol in Polkadot which exploits our results. In particular, the election algorithm is run off-chain, with only the PJR test executed on-chain to verify the correctness of the computation. 
 As a result, we obtain a highly efficient and transparent protocol with hard and verifiable guarantees. 
In Appendix~\ref{s:balanced} we consider the problem of finding a \emph{balanced} vote distribution for a given committee (which we define in the Preliminaries section) and present new and faster algorithms for it.
