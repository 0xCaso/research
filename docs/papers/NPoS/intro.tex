\section{Introduction}

The property of proportional representation in approval-based committee elections has been discussed in the literature of computational social choice for a long time, with new mathematical formulations recently proposed to formalize it. 
The property is typically understood as ensuring that no minority in the electorate is underrepresented by the winning committee. 
In this paper we complement this notion with an analysis of the opposite goal: fighting overrepresentation of any minority. 
We establish such objective formally as an optimization problem, provide a thorough complexity analysis for it, and propose efficient approximation algorithms. 

Our work is motivated by the problem of electing participants to the consensus protocol -- whom we refer to as \emph{validators} -- in a decentralized blockchain network, where overrepresentation may affect security. 
More precisely, under a proof-of-stake mechanism it is assumed that a majority of the native token is in the hands of honest actors, and this proportion of honest actors is to be preserved among the elected validators; 
hence, if the validator committee is selected via an election process where users have a vote strength proportional to their token holdings, an adversarial minority should not be able to gain overrepresentation. 
We propose an efficient election rule for the selection of validators which offers strong guarantees on security and proportional representation, and successfully adapts to the distributed and resource-limited nature of blockchain protocols.

\emph{Background on Nominated Proof-of-Stake.}
Many blockchain projects launched in recently years substitute the inefficient proof-of-work (PoW) component of Nakamoto's consensus protocol~\cite{nakamoto2019bitcoin} with proof-of-stake (PoS), which is much more efficient and offers security in the presence of a dishonest minority of users, where this minority is to be measured relative to token holdings. 
We focus on \emph{nominated proof-of-stake} (NPoS), a variant of PoS where at regular intervals a committee of validators is elected by the community to participate in the consensus protocol. 
Users are free to become validator candidates, or become \emph{nominators}, who approve of candidates that they trust and back them with their tokens. 
Both validators and nominators thus put their tokens at stake and receive economic rewards (coming from transaction fees and minting of new tokens) on a pro-rata basis, and may also be penalized in case a validator shows negligent or adversarial behavior.
Polkadot~\cite{burdges2020overview} is a blockchain network launched in 2020 which implements NPoS.

Having the nominator role in the system allows for a massive number of users to indirectly participate in the consensus protocol, in contrast to the number of elected validators which due to operational limitations remains bounded, typically in the order of hundreds. 
This allows for an unlimited amount of stake to back validators, much higher than any single user's holding. 
Intuitively, the higher the stake backing the validators, the higher the \textbf{security} of the network; we make this intuition formal further below. 
As such, one of the goals of the validator election rule is to maximize the stake backing all elected validators. 
The second, equally important goal is \textbf{proportionality}, i.e.~picking a committee where nominators are represented in proportion to their stake. 
We highlight that diverse preferences and factions will naturally arise among nominators for reasons that range from economic and technical to political, geographical, etc. 
Such diversity of points of view is expected and welcome in a distributed system, and a proportional committee will keep the system decentralized and its users satisfied.


\emph{Proportional representation in multiwinner elections.}
Let us now formalize this validator election and its two stated goals. 
For simplicity, we consider that only nominators have stake, not candidates, and that a nominator's approval of candidates is dichotomous, i.e.~she provides an unranked list of approved candidates of any size. 
This leads a to vote-weighted, approval-based committee election scheme. 
There are finite sets $N$ and $C$ of voters (nominators) and candidates respectively, and every voter $n\in N$ provides a list $C_n\subseteq C$ of approved candidates and has a vote strength $s_n$ (their stake). 
There is also a target number $1\leq k< |C|$ of candidates to elect.

One of our goals corresponds to the classical notion of proportional representation, i.e.~the committee should represent each group in the electorate proportional to their aggregate vote strength, with no minority being underrepresented. 
Electoral system designs that achieve some version of proportional representation have been present in the literature of social choice for a very long time. Of special note is the work of Scandinavian mathematicians Edvard Phragm\'{e}n and Thorvald Thiele in the late nineteenth century \cite{phragmen1894methode, phragmen1895proportionella, phragmen1896theorie, phragmen1899till, thiele1895om, janson2016phragmen}. 
Several axioms have been recently proposed to define the property mathematically; we mention the most relevant ones. 
\emph{Justified representation} (JR)~\cite{aziz2017justified} states that if a group $N'\subseteq N$ of voters is cohesive enough in terms of candidate preferences and has a large enough aggregate vote strength, then it has a justified claim to be represented by a member of the committee.
\emph{Proportional justified representation} (PJR)~\cite{sanchez2017proportional} says that such a group $N'$ deserves not just one but certain number of representatives according its vote strength, where a committee member is said to represent the group as long as it represents any voter in it.
Finally, \emph{extended justified representation} (EJR)~\cite{aziz2017justified} strengthens this last condition and requires not only that $N'$ have enough representatives as a group, but some voter in it must have enough representatives individually.
It is known that EJR implies PJR, and PJR implies JR, but converse implications are not true~\cite{sanchez2017proportional}. %
For each of these properties, we say that a committee voting rule satisfies said property if its output committee is always guaranteed to satisfy the property for any input instance. 
While the most common voting rules usually satisfy JR, they fail to satisfy the stronger properties of PJR and EJR, and up to recently there were no known efficient voting rules that satisfy the latter two. 
For instance, the \emph{proportional approval voting} (PAV) method \cite{thiele1895om, janson2016phragmen} proposed by Thiele satisfies EJR but is NP-hard to compute, while efficient heuristics based on it, such as reweighted approval voting, fail PJR \cite{aziz2014computational, skowron2016finding, aziz2017justified}. 
Only very recently have efficient algorithms that achieve PJR or EJR finally been proposed \cite{brill2017phragmen, sanchez2016maximin, aziz2018complexity, peters2019proportionality}. 

Among these axioms, we set to achieve \textbf{PJR}, defined formally in Section~\ref{s:prel}, for two reasons. 
First, because it is more \emph{Sybil resistant}~\cite{douceur2002sybil} than JR, meaning that a strategic voter may be incentivized to assume several nominator identities in the network under JR, but not under PJR. 
Second, because PJR seems to be most compatible with our security objective, as we discuss further below; in particular, our security objective turns out to be incompatible with EJR.%
\footnote{As argued in~\cite{peters2019proportionality} and \cite{lackner2020approval}, the PJR and EJR axioms correspond to different notions of proportionality. While EJR is primarily concerned with the general welfare or satisfaction of the voters, PJR considers proportionality of the voters' decision power.} 

\emph{The security goal: fighting overrepresentation.} 
In a PoS-based consensus protocol, we want a guarantee that as long as most of the stake is in the hands of actors that behave honestly or rationally, carrying any attack will be very costly. If we assume that the attack requires control of some minimum number of validators to succeed, the adversary would need to use its stake to get these validators elected. The security level is hence proportional to how difficult it is for an entity to gain overrepresentation in the validator election protocol. 
Interestingly, this is in stark contrast to the classical approach taken in proportional representation axioms, that only seek to avoid underrepresentation. 

Recall that each voter $n\in N$ has a vote strength $s_n$ and a list of approved candidates $C_n\subseteq C$. 
Suppose we want to make it as expensive as possible for an adversary to gain a certain number $1\leq r\leq k$ of representatives into the $k$-validator committee, and that few to none of the honest nominators trust these candidates. Then, our goal would be to elect a committee $A\subseteq C$ that maximizes 
$\min_{A'\subseteq A, |A'|=r} \sum_{n\in N: \ C_n\cap A'\neq \emptyset} s_n$.
This gives a different objective for each value of threshold $r$. 
For example, for $r=1$, maximizing this objective is equivalent to the classical multiwinner approval voting: selecting the $k$ candidates $c\in C$ with highest total approval $\sum_{n\in N: \ c\in C_n} s_n$. 
If we are only concerned about a particular threshold, e.g. $r=\lceil(k+1)/3\rceil$ for a $34\%$ attack%
\footnote{This is the minimum threshold required to carry on a successful attack in classical Byzantine fault tolerant protocols~\cite{pease1980reaching}.}, 
then we can fix the corresponding objective. However, different types of attacks require different thresholds, and some attacks succeed with higher probability with more attacking validators. Hence, a more pragmatic approach is to incorporate the threshold into the objective, and maximize \emph{the least possible per-validator cost over all thresholds}, i.e.  
\begin{align}\label{eq:security}
    \text{Maximize } \min_{A'\subseteq A, A'\neq \emptyset} \quad \frac{1}{|A'|} \sum_{n\in N: \ C_n\cap A' \neq \emptyset} s_n, \quad \text{over all committees $A\subseteq C$ with $|A|=k$}.
\end{align}

We establish in Lemma \ref{lem:equivalence} that this objective is equivalent to the \textbf{maximin support objective}, recently introduced by Sánchez-Fernández et al.~\cite{sanchez2016maximin}, and which we thus set to optimize. 
%To define this last objective, which we do formally in Section~\ref{s:prel}, one needs the election rule to establish not only a winning committee $A\subseteq C$, but also a \emph{vote distribution}; that is, a fractional distribution of each voter $n$'s vote strength $s_n$ among her approved committee members in $C_n\cap A$.%
%\footnote{This is called a \emph{support distribution function} in~\cite{sanchez2016maximin}, and is related to the notion of a \emph{price system} in~\cite{peters2019proportionality}.} 
%For instance, for voter $n$ the election rule may assign a third of $s_n$ to $c_1$ and two thirds of $s_n$ to $c_2$, where $c_1, c_2\in C_n\cap A$. 
%The objective is then to maximize, over all possible committees and distributions, the least amount of vote assigned to any committee member. 
%We observe here that unlike most other applications of multiwinner elections, in NPoS there is practical utility in computing a vote distribution from nominators to the elected validators: by reversing its sense, it establishes the exact way in which the validators' payouts or penalties must be distributed back to the nominators.
The authors in~\cite{sanchez2016maximin} remark that the corresponding problem is NP-hard, but do not study its approximability. They also prove that in its exact version it is equivalent to another objective, $\maxphragmen$, devised by Phragm\'{e}n and recently analyzed in~\cite{brill2017phragmen}, and in this last paper it is shown that $\maxphragmen$ is incompatible with EJR. Thus, the same incompatibility holds for our security objective. 

\textbf{Our contribution.}
Our security analysis in the election of validators leads us to pursue the goal of fighting overrepresentation, and more specifically to the maximin support objective.  
Conversely, we formalize our goal of proportionality to the PJR property, which fights underrepresentation. 
We show in this paper that these two objectives complement each other well, and prove the existence of efficient election rules that achieve guarantees for both objectives. 

\begin{theorem}\label{thm:intro1}
There is an efficient election rule for approval-based committee elections that simultaneously achieves the PJR property and a 3.15-factor approximation guarantee for the maximin support objective.
\end{theorem}

Our optimization approach to proportional representation also provides new tools to discern between different election rules. For instance, $\MMS$~\cite{sanchez2016maximin} and $\phragmen$~\cite{brill2017phragmen} are two efficient rules that achieve the PRJ property, but while the former provides a constant-factor approximation guarantee for maximin support, the latter does not (Section~\ref{s:complexity}). 
%
Next comes the question of applicability: the blockchain architecture adds very stringent time constraints to computations, and may render all but the simplest algorithms unimplementable.%
%
\footnote{For instance, the PoS-based project EOS uses multiwinner approval voting to elect a validator committee, which is a highly efficient election rule but is known to perform poorly in terms of proportionality. This has lead to user dissatisfaction and claims of excessive centralization. 
For an analysis, we refer to the blogpost ``EOS voting structure encourages centralization'' by Priyeshu Garg at \url{https://cryptoslate.com/eos-voting-structure-encourages-centralization/}} %
% 
This is because, in order to reach consensus on a protocol, validators typically need to execute it individually, and no further blocks can be produced until most validators finish the computation and agree on the output. 
However, if an output can be \emph{verified} much faster than it can be computed, then it is possible to dump the computation to \emph{off-chain workers}, who execute it privately and separately from block production, leaving only the output verification task to validators. This is the case for our election rule.

\begin{theorem}\label{thm:intro2}
There is a test that takes as input an election instance and an arbitrary solution to it, such that a) the test runs in linear time in the size of the input, b) if the solution passes the test, then it simultaneously satisfies the PJR property and provides a 3.15-factor approximation guarantee for the maximin support objective, and c) the output of the election rule mentioned in Theorem~\ref{thm:intro1} always passes this test.
\end{theorem}

 We believe this to be the most important feature of our proposed election rule relative to others in the literature. 
Such efficient verification helps ensure transparency in a decentralized governance process, and might be of independent interest.  
In the context of our motivating application, it makes our election rule implementable into a fast blockchain validator election protocol which provides strong and verifiable guarantees on security and proportionality. 
We provide details on such a protocol in Section~\ref{s:objectives}, which will be the basis for an implementation in the Polkadot network.
%
Finally, we derive from the new election rule a post-computation which, when paired with any approximation algorithm for the maximin support problem, makes it also satisfy PJR in a black-box manner.

\begin{theorem}\label{thm:intro3}
There is an efficient computation that takes as input an election instance and an arbitrary solution to it, and outputs a new solution which a) is no worse than the input solution in terms of the maximin support objective, b) satisfies the PJR property, and in particular c) can be efficiently tested to satisfy the PJR property.
\end{theorem}

\textbf{Organization of the paper.}
We start with a thorough complexity analysis of the maximin support problem in Section~\ref{s:complexity}, 
where we exhibit several approximation algorithms for it, and prove as well that it does not admit a PTAS% 
    \footnote{A \emph{polynomial time approximation scheme} (PTAS) for an optimization problem is an algorithm that, for any constant $\eps>0$ and any given instance, returns a $(1+\eps)$-factor approximation in polynomial time.} %
unless P=NP so constant-factor approximations are best possible.
%
In Section~\ref{s:heuristic} we propose a new heuristic, $\phragmms$, inspired in $\phragmen$~\cite{brill2017phragmen} but allowing for a more robust analysis and better guarantees than the latter both in terms of the PJR property and the maximin support objective. We use it to prove Theorem~\ref{thm:intro1} and obtain the fastest known algorithm that simultaneously guarantees a) the PJR property and b) a constant-factor approximation for maximin support.

In Section~\ref{s:local} we prove Theorems \ref{thm:intro2} and \ref{thm:intro3}, and explore how the two aforementioned guarantees can be efficiently tested by a \emph{verifier}, even if the algorithm is privately executed by an untrusted \emph{prover} who only communicates the solution. 
To do so, we define a parametric version of the PJR property, and a notion of local optimality for solutions. 
%
Finally, in Section~\ref{s:objectives} we propose a validator election protocol for an NPoS-based blockchain network, where we exploit our results. This proposal will be the basis for an implementation in Polkadot. 
