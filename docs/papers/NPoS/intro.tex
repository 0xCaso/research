\section{Introduction}


%Mention that EOS only has 21 block producers at any time, so very centralized. Check how similar Tron is, which also uses DPoS. What about Steem?

%Mention EOS was the biggest ICO at the time
%Mention bound on number of active validators is part of general effort to make blockchain more efficient

% Mention terms stake (ie collateral), stake pools and reward sharing schemes, representative democracy, and cite https://arxiv.org/ftp/arxiv/papers/1807/1807.11218.pdf
% Cite same source for reasons to bound # of validators, and say another downside is that operational costs and extreme variance in income can discourage participation from small stakeholders, resulting in so-called "whales" dominating the system.
% The remaining amount is distributed among the pool members, including the pool leader, proportionally to the stake that they contributed to the pool. In our analysis we will take advantage of automatic enforcement of our reward scheme, as e.g., this can be guaranteed by a smart contract built-in the underlying ledger
%  elected representatives, despite getting equal power, are rewarded according to votes received; this inconsistency between representation and power may result in a relatively small fraction of stake controlling the system (e.g., at some point, controlling EOS delegates representing just 2.2% of stakeholders was sufficient to halt the system, which ideally could withstand a ratio less than 1/3). It may leave a large fraction of stakeholders without representation (e.g., in EOS, at some point, only 8% of total stake is represented by the 21 leading delegates).
% This provides a game theoretical analysis for a proposed reward sharing scheme for stake pools, though unfortunately it has not been implemented as far as we can tell.

% Mention we consider an adversary with a Sybil behavior.


% Mention that, while the maximin support objective tries to maxim make it as expensive as possible to reach a certain number of representatives in terms of voting strength for a fixed committee size, the property of PJR (and in particular its parameterized version which we define in this paper) can be thought of as minimizing the 

% Mention DPoS and Casper

% Say we provide a robust analysis from the point of view of computational social choice. Don't say we're the "first" robust analysis.

% Mention analysis of objective function is useful to motivate choice of objective, but from now on we'll only focus on the maximin support objective and not on (1)



In an approval-based committee election, each voter either approves or disapproves of each candidate, without declaring any preferences among the approved ones~\cite{lackner2020approval}. From the voters' ballots taken as input, an election rule outputs a winning committee of candidates of a certain size $k$, in the pursuit of some goals or criteria. Proportional representation is one of the most prominent such criteria, having been discussed in the literature of computational social choice for a long time. It is typically understood as a guarantee that small minorities within the electorate are not \emph{underrepresented} by the winning committee, which is certainly an imperative in any fair election process as it ensures that all voices are heard and all communities are satisfied and engaged. 
In this paper we complement this notion by formalizing the opposite goal of preventing the \emph{overrepresentation} of any minority. We consider this to be a matter of security, and analyze a scenario where an adversarial minority may use strategic voting to attempt to gain overrepresentation in order to capture the governance body or interfere with its correct functioning. 

Consequently, we consider the election to be at risk of attack if the winning committee contains a subset of seats whose aggregate vote support is too low relative to the subset size, and establish an optimization problem that eliminates this scenario. 
In this work we justify this problem from first principles and provide a thorough analysis of its computational complexity. 
%We also establish its connection to relevant axioms of proportional representation in the literature, 
We also study the performance of the most prominent polynomial-time election rules in the literature of proportional representation, the conclusion being that each of them either fails to provide a security guarantee or is considerably slow. Finally, we present a new, more efficient election rule that provides strong guarantees in terms of security as well as traditional proportional representation. Full details are provided further below.


\textbf{Blockchain architecture and verifiable computing.}
Our work is motivated by an application on public, permissionless blockchain networks. These networks are equipped with computational and economic capabilities and have no central authority nor single point of failure, which provides them with unprecedented levels of decentralization and resistance to attacks, and for the first time removes the need for trusted intermediaries in peer-to-peer value transfers across the world. Recent years have seen an explosion of blockchain-based applications in finance, commerce, logistics, art and gaming; see~\cite{maesa2020blockchain} for a survey. 
Rather than controlling the identity and correct execution of each node, a blockchain network freely allows nodes to join or leave the network pseudonymously, and adds enough redundancy to resist the erroneous execution of any one of them. 
Nodes that participate in the consensus mechanism are known as \emph{validators}, and the correct functioning of the network is guaranteed as long as a supermajority of validators execute correctly.

Yet, the advantages of a blockchain architecture come at the cost of hard computational limitations. For every new block of transactions, all validators around the globe need to perform the same computations locally, and the network must wait for all of them to finish and agree on the results before processing the next block. Furthermore, a robust design should account for computationally weak validators such as consumer-grade computers, as imposing high performance requirements would lead to centralization, so the per-block computational load must remain conservatively low. Because of this, earlier networks such as Bitcoin~\cite{nakamoto2019bitcoin} and Ethereum 1.0~\cite{wood2014ethereum} can only process tens of transactions per second~\cite{chauhan2018blockchain}. 

As a consequence, only the simplest and most efficient algorithms admit a straightforward implementation over a blockchain network. This represents a considerable hindrance for the use of committee election rules, in particular those sophisticated enough to provide guarantees on proportional representation or security. As a case in point, the $\MMS$ rule~\cite{sanchez2016maximin} provides these guarantees, as we establish in this paper, but its slow (polynomial) runtime makes it unsuitable for implementation. On the other hand, the EOS network~\cite{griggeos} applies the multiwinner approval voting rule on its validator selection protocol, a rule that is highly efficient yet known to perform very poorly in terms of proportional representation~\cite[Table 2]{lackner2020approval}. 
The choice of this rule, in all likelihood based on operational considerations, has led to user dissatisfaction and claims of excessive centralization of the EOS network.%
%
\footnote{See the 2019 opinion piece ``EOS voting structure encourages centralization''~\cite{garg} and the news article ``Crypto ratings agency downgrades EOS for serious centralization problems''~\cite{chong}. 
Br{\"u}njes et al.~\cite{brunjes2020reward} state ``At some point, controlling EOS delegates representing just 2.2\% of stakeholders was sufficient to halt the system... [and] only 8\% of total stake is represented by the 21 leading delegates.'' 
Protocol designer Aarin Hagerty~\cite{hagerty} mentions this about proposed election rules for EOS: ``Personally, I am not satisfied with any of these solutions because none of them addresses the proportional representation criterion. There are voting mechanisms that do satisfy it... The problem tends to be that they are actually very computationally intensive... For small numbers of winners and candidates it may be feasible (though this is still a difficult engineering challenge to build within smart contracts running on a blockchain), but if you go to even moderate numbers of winners and candidates it quickly explodes combinatorially and can even become infeasible to do off-chain... In summary, social choice theory is hard.''}
%We expand on \emph{validator selection protocols} further below.

Several solutions have been proposed and sucessfully implemented for scaling up the computational capabilities of new-generation blockchain networks, such as sharding and layer-2 solutions; see~\cite{zhou2020solutions} for a survey. Of relevance to our work is the use of \emph{verifiable computing schemes}~\cite{gennaro2010non}. Such a scheme offloads a heavy task to one or more \emph{off-chain workers}, that is, entities that are logically separated from the network and may process the task on high-performance machines and/or with relaxed time frames as their operations do not interfere with block production. Once the task is completed and the output is loaded back onto the network, its correctness is verified by the validators. 
This is a sensible scheme for any task that admits a \emph{verification process} on its output that a) guarantees correctness, even when the task is performed by an untrusted party that only communicates such output, and b) is much more efficient than performing the task itself from scratch. One of our main contributions is showing that, from the winning committee that is output by our proposed election rule, along with the original input voters' ballots, one can verify the guarantees on proportionality and security in only linear time. Hence, our election rule can be implemented as a verifiable computing scheme, and in fact has been successfully implemented as such in the Polkadot network~\cite{burdges2020overview}, launched in 2020, where it is applied in its validator selection protocol; we include details of this implementation in one of our sections. Therefore, our work constitutes a first effort towards applying verifiable computing to election rules. Of course, one may argue that developing election processes that facilitate the public verification of results in accordance to clearly defined criteria is a worthwhile pursuit in itself beyond any concerns of implementation over blockchain networks.

Still, the issue of implementability of election rules has become ever more relevant in recent years, and not only for blockchain-based solutions expressly built for online voting. New blockchain networks with any functionality are likely to run elections in two of its core protocols. 
The first one is \emph{on-chain governance}~\cite{beck2018governance}: many projects are abandoning the notion of immutable code in favor of a more flexible design that facilitate future code upgrades via an embedded voting process of all holders of the native token. This process not only helps in terms of coordination but also legitimizes the result and avoids hard forks. On-chain governance may also allow token holders to vote on committees, councils and referenda, launch their own candidacies, raise proposals, form commissions, etc. 
The second core protocol is validator selection, which we mentioned before and we describe in detail next as it is our motivating application and the starting point for our problem definition. 

\textbf{Validator selection in Proof-of-Stake.}
Many blockchain networks launched in recent years substitute the highly inefficient Proof-of-Work (PoW) component of the consensus mechanism~\cite{nakamoto2019bitcoin} with Proof-of-Stake (PoS), in which the level of participation of validators depends on their token holdings --their stake-- as opposed to their computational power. 
While a pure PoS system allows any willing token holder to participate directly, most projects place a hard bound $k$ on the number of validators that can be active at any given moment. 
Arguments for setting such a bound are that the increase in operational costs and communication complexity eventually outmatches the marginal increase in benefits stemming from decentralization as $k$ grows, and that most users with little stake would find it inconvenient to keep a node constantly online for only sporadic participation, and would rather form validation pools in order to decrease the variance on their revenue and profit from economies of scale. %
%
Instead, a system with a ``representative democracy'' approach formalizes and facilitates the formation of these pools by allowing users to vote for candidates that they trust, and then automatically selects a committee with $k$ of the most trusted candidates as validators. Networks that broadly follow this approach include Polkadot~\cite{burdges2020overview}, Cardano~\cite{brunjes2020reward}, and more generally those that apply Delegated Proof-of-Stake (DPoS) or Ouroboros~\cite{kiayias2017ouroboros} protocols. 

While similar in spirit, the approaches taken by these projects vary in several regards, most significantly in terms of incentives and the electoral system used. These design choices are of the utmost importance as they affect the decentralization and security levels achieved by the network; we refer again to the case of the EOS network, which implements DPoS and whose centralization issues are mentioned above. 
Yet, rigorous analyses behind these design choices are generally lacking in the space. 
A notable exception is the recent work by Br{\"u}njes et al.~\cite{brunjes2020reward}, that proposes an incentive scheme for stake pools backed by a game theoretical analysis. In turn, in the present work we propose for the first time an electoral system for the selection of validators and analyze it from the perspective of social choice. 

We focus on Nominated Proof-of-Stake (NPoS), the design implemented by the Polkadot and Kusama networks. In NPoS, any stakeholder is free to become a validator candidate, or a \emph{nominator} who provides an unranked list of candidates that she trusts. At regular intervals of a few hours, a committee of $k$ validators --in the order of hundreds or thousands-- is elected according to the current nominators' preferences. 
As a security measure, both validators and nominators have their stake locked as collateral, so that if a validator ever shows negligent or adversarial behavior, backing nominators are susceptible to losing their stake. Conversely, during normal execution the network provides economic rewards to all validators and their backing nominators in proportion to their stake and in a non-custodial way. Nominators are thus indirect participants in the consensus mechanism with a vested economical interest to guard the performance of validators and support only the most capable and trustworthy candidates.

%\vspace{5mm}
\textbf{Problem definition.}
For the sake of simplicity, in what follows we consider a model where only nominators have stake, not candidates, and we equate their stake amount to their voting strength. This leads to a vote-weighted, approval-based committee election problem. We remark here that most of the following concepts are described only in terms of unit votes in the social choice literature; in this paper we consider natural generalizations to real-valued vote strengths, following the principle that a voter with two units of strength is equivalent to two voters with unit strength and identical ballots. 
As mentioned before, our problem is defined by the goals of proportional representation and security. We formalize each of these goals next. 

\emph{Proportional representation.} We aim to guarantee that nominators are not \emph{underrepresented} relative to their stake by the elected validators. We highlight that diverse preferences and factions may naturally arise among nominators for reasons that range from economically and technically motivated to political, geographical, etc., and that preserving this diversity among the validators helps ensure that the network stays decentralized. 
Electoral system designs that achieve some version of proportional representation have been present in the literature of social choice for a very long time. Of special note is the work of Scandinavian mathematicians Edvard Phragm\'{e}n and Thorvald Thiele in the late nineteenth century \cite{phragmen1894methode, phragmen1895proportionella, phragmen1896theorie, phragmen1899till, thiele1895om, janson2016phragmen}. 
Several axioms have been recently proposed to define the property mathematically -- we mention the most relevant ones. 
\emph{Justified representation} (JR)~\cite{aziz2017justified} states that if a group of voters is cohesive enough in terms of candidate preferences and has a large enough aggregate vote strength, then it has a justified claim to be represented by a member of the committee.
\emph{Proportional justified representation} (PJR)~\cite{sanchez2017proportional} says that such a group deserves not just one but a minimum number of representatives according to its vote strength, where a committee member is said to represent the group as long as it represents any voter in it.
Finally, \emph{extended justified representation} (EJR)~\cite{aziz2017justified} strengthens this last condition and requires not only that the group have enough representatives collectively, but some voter in it must have enough representatives individually.
It is known that EJR implies PJR and PJR implies JR, but converse implications are not true~\cite{sanchez2017proportional}. %
For each of these properties, a committee voting rule is said to satisfy said property if its output committee always satisfies it for any input instance. 
While the most common voting rules usually achieve JR, they fail the stronger properties of PJR and EJR, and up to recently there were no known efficient voting rules that satisfy the latter two. 
For instance, the proportional approval voting (PAV) method \cite{thiele1895om, janson2016phragmen} proposed by Thiele satisfies EJR but is NP-hard to compute, while efficient heuristics based on it, such as reweighted approval voting, fail PJR \cite{aziz2014computational, skowron2016finding, aziz2017justified}. 
Only very recently have efficient algorithms that achieve PJR or EJR finally been proposed \cite{brill2017phragmen, sanchez2016maximin, aziz2018complexity, peters2019proportionality}. 

Among these axioms, \textbf{we set to achieve PJR}, defined formally in Section~\ref{s:prel}, for two reasons. 
First, because it is more \emph{Sybil resistant}~\cite{douceur2002sybil} than JR, meaning that in our blockchain application a strategic voter may be incentivized to assume several nominator identities in the network under JR, but not under PJR. 
Second, because PJR seems to be most compatible with our security objective, as we argue below. Indeed, as claimed in~\cite{peters2019proportionality} and \cite{lackner2020approval}, the PJR and EJR axioms seem to correspond to different notions of proportionality: while EJR is primarily concerned with the general welfare or satisfaction of the voters, PJR considers proportionality of the voters' decision power, and our security objective aligns best with the latter notion.

\emph{Security.} 
As is the case in any PoS-based blockchain network, under NPoS the basic security assumption is that most of the stake is held by actors who behave honestly or rationally. Under this assumption, we consider an adversary that attempts to carry out an attack on the network, and has the power to create any number of identities including both nominators and candidates (via Sybil behavior), yet has a fixed stake budget. Depending on the type of attack, in order to succeed it will require that a minimum number of candidates under its control get elected in the committee, and the adversary may recur to strategic voting to achieve this. Therefore, the security level corresponds to how difficult it is for a voter or group of voters with limited aggregate voting strength to gain \emph{overrepresentation} in the elected committee. 

Further formalizing our election problem, we consider finite sets $N$ and $C$ of voters and candidates respectively, where every voter $n\in N$ provides a list $C_n\subseteq C$ of approved candidates and has a vote strength $s_n$. 
%There is also a target number $1\leq k< |C|$ of candidates to elect.
Suppose we want to make it as difficult as possible for an adversary to gain a certain threshold $1\leq r\leq k$ of representatives within the $k$-validator committee. 
Then, our goal would be to elect a committee $A\subseteq C$ that maximizes 
$$\min_{A'\subseteq A, |A'|=r} \sum_{n\in N: \ C_n\cap A'\neq \emptyset} s_n.$$ 

For any subset $A'\subseteq A$ of $r$ seats in committee $A$, the quantity above is the total vote strength that is backing any seat in $A'$, with no vote being double counted. In our blockchain application, this quantity also corresponds to the total collateral susceptible to being lost if $A'$ carries out an attack; hence, maximizing this amount not only makes it difficult for the adversary to gain enough representatives, but also costly to carry the attack if it does. Of course, we remark that on top of the potential loss of collateral, rational validators should consider the potential loss of reputation, which translates in opportunity loss of future payouts. %
%
We thus obtain a different optimization objective for each value of threshold $r$. 
If we are only concerned about a particular threshold $r$, then we can fix the corresponding objective. 
For example, for $r=1$, maximizing this objective is equivalent to the classical multiwinner approval voting rule: selecting the $k$ candidates $c\in C$ with highest total approval $\sum_{n\in N: \ c\in C_n} s_n$. 
Or, we could set $r$ to $\lceil k/3\rceil$ or to $\lceil k/2\rceil$, which are respectively the thresholds required to carry on a successful attack in classical Byzantine fault tolerant consensus~\cite{pease1980reaching} and in Nakamoto consensus~\cite{stifter2018agreement}. 
However, different types of attacks require different thresholds, and some attacks succeed with higher probability with more attacking validators. Hence, a more pragmatic approach is to incorporate the threshold into the objective and maximize \emph{the least possible cost per seat over all thresholds}, i.e.,  
\begin{align}\label{eq:security}
    \text{Maximize } \min_{A'\subseteq A, A'\neq \emptyset} \quad \frac{1}{|A'|} \sum_{n\in N: \ C_n\cap A' \neq \emptyset} s_n, \quad \text{over all committees $A\subseteq C$ with $|A|=k$}.
\end{align}

We establish in Lemma~\ref{lem:equivalence} that this objective is equivalent to the \textbf{maximin support objective}, recently introduced by Sánchez-Fernández et al.~\cite{sanchez2016maximin}, and which we thus set to optimize. 
We define it formally in Section~\ref{s:prel}.
%To define this last objective, which we do formally in Section~\ref{s:prel}, one needs the election rule to establish not only a winning committee $A\subseteq C$, but also a \emph{vote distribution}; that is, a fractional distribution of each voter $n$'s vote strength $s_n$ among her approved committee members in $C_n\cap A$.%
%\footnote{This is called a \emph{support distribution function} in~\cite{sanchez2016maximin}, and is related to the notion of a \emph{price system} in~\cite{peters2019proportionality}.} 
%For instance, for voter $n$ the election rule may assign a third of $s_n$ to $c_1$ and two thirds of $s_n$ to $c_2$, where $c_1, c_2\in C_n\cap A$. 
%The objective is then to maximize, over all possible committees and distributions, the least amount of vote assigned to any committee member. 
%We observe here that unlike most other applications of multiwinner elections, in NPoS there is practical utility in computing a vote distribution from nominators to the elected validators: by reversing its sense, it establishes the exact way in which the validators' payouts or penalties must be distributed back to the nominators.
The authors in~\cite{sanchez2016maximin} remark that in its exact version, maximin support is equivalent to another objective, $\maxphragmen$, devised by Phragm\'{e}n and recently analyzed in~\cite{brill2017phragmen}, and in this last paper it is shown that $\maxphragmen$ is NP-hard and incompatible with EJR. 
Thus, the same hardness and incompatibility with EJR holds true for our security objective. 
%To the best of our knowledge, the approximability of maximin support has not previously been studied.

%\vspace{5mm}
\textbf{Our contribution.}
Our analysis of security of the validator selection protocol leads us to pursue the maximin support objective, which fights overrepresentation. Conversely, we equate our proportionality goal with the PJR property, which fights underrepresentation. We show that these two goals complement each other well, and prove the existence of efficient election rules that achieve guarantees for both of them. 

\begin{theorem}\label{thm:intro1}
There is an efficient election rule for approval-based committee elections that simultaneously achieves the PJR property and a 3.15-factor approximation guarantee for the maximin support objective.
\end{theorem}

Our proposed election rule is inspired in the $\phragmen$ method~\cite{brill2017phragmen}, and to the best of our knowledge we present the first analysis of approximability for a Phragm\'{e}n objective. 
In contrast, several approximation algorithms for Thiele objectives have been proposed; see~\cite{lackner2020approval} for a survey. To complement this result, we also prove that a constant-factor approximation is theoretically best possible for the maximin support objective. 
%
Next comes the question of applicability: as we previously mentioned, the blockchain architecture adds very stringent time constraints to computations. However, if the output can be \emph{verified} much faster than it can be obtained from scratch, then the task can be implemented as a verifiable computing scheme. This is the case for our election rule.

\begin{theorem}\label{thm:intro2}
There is a linear-time test that takes as input an election instance and an arbitrary solution to it, such that if the test passes then the input solution satisfies the PJR property and a 3.15-factor approximation guarantee for the maximin support objective. 
Moreover, the output of the election rule mentioned in Theorem~\ref{thm:intro1} always passes this test.
\end{theorem}

We remark that a solution passing said test is a sufficient but not a necessary condition for it to have the properties above, hence the fact that the output of our election rule always passes the test is not straightforward.
In the context of our blockchain application, this result enables the implementation of a fast validator selection protocol with strong theoretical guarantees on security and proportionality. 
We provide details of such an implementation over the Polkadot network in Section~\ref{s:objectives}.
%
Finally, we derive from the new election rule a post-computation which, when paired with any approximation algorithm for the maximin support problem, makes it also satisfy the PJR property in a black-box manner.

\begin{theorem}\label{thm:intro3}
There is an efficient computation that takes as input an election instance and an arbitrary solution to it, and outputs a new solution which a) is no worse than the input solution in terms of the maximin support objective, b) satisfies the PJR property, and in particular c) can be efficiently tested to satisfy the PJR property.
\end{theorem}

%This result shows that the PJR property is strongly compatible with the maximin support objective, and can be easily added to future approximation algorithms that may be developed for this objective.

%\vspace{5mm}
\textbf{Organization of the paper and technical overview.}
In Section~\ref{s:prel} we formalize our multiwinner election problem and its objectives and provide all the required technical definitions. Then, in Section~\ref{s:complexity} we present a thorough analysis of the complexity of the maximin support objective, where we exhibit constant-factor approximation algorithms for it, as well as prove a hardness result showing that a constant-factor guarantee is best possible. 
We also compare the performance of several common election rules with respect to this objective, and show that in fact most of them do not achieve constant-factor approximations for it.
In Section~\ref{s:heuristic} we propose a new heuristic, $\phragmms$, inspired in $\phragmen$~\cite{brill2017phragmen} but allowing for a more robust analysis and better guarantees than the latter both in terms of the PJR property and the maximin support objective, and we use it to prove Theorem~\ref{thm:intro1}. 

In these sections we formalize the \emph{load balancing} technique used in $\phragmen$ in terms of network flow, give a precise definition of what a \emph{balanced solution} is, and provide an algorithm to compute one efficiently using notions of parametric flow. 
We then synthesize the strategies of different relevant heuristics in terms of how they compute or approximate balanced solutions, and use the flow decomposition theorem to establish approximation guarantees. 
Our comparison provides new tools to discern between election rules. For instance, the survey paper by Lackner and Skowron~\cite{lackner2020approval} mentions $\phragmen$ and $\MMS$~\cite{sanchez2016maximin} as two efficient rules that achieve the PJR property and leaves as an open question which of the two is preferable, whereas we show that the latter provides a constant-factor approximation guarantee for maximin support but the former does not. 
Furthermore, contrary to the common notion that the EJR property is categorically superior than PJR, we establish PJR as the better choice in terms of security.

In Section~\ref{s:local} we prove Theorems \ref{thm:intro2} and \ref{thm:intro3}, and explore how the PJR property and a constant-factor approximation guarantee for maximin support can be efficiently tested, even if the algorithm is privately executed by an untrusted party who only communicates the solution. 
To do so, we define a parametric version of the PJR property, and link it to a notion of local optimality for our new heuristic. Finally, in Section~\ref{s:objectives} we provide details on how our proposed voting rule has been implemented in the validator selection protocol of Polkadot as a verifiable computing scheme.
%We present some conclusion and open question in Section~\ref{s:conclusions}.
