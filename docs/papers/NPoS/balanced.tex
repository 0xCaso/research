\section{Properties and algorithms for balanced solutions} \label{s:balanced}

Consider an instance $(G=(N\cup C, E), s, k)$ as defined in Section~\ref{s:prel}. 
For a fixed (possibly partial) committee $A\subseteq C$, recall from Section~\ref{s:prel} that a feasible edge weight vector $w\in\R^E$ is said to be \emph{balanced for $A$} if %
%
\begin{enumerate}
    \item it maximizes $\sum_{c\in A} supp_{w'}(c)$, over all feasible weight vectors $w'$, and \label{balanced1}
    \item it minimizes $\sum_{c\in A} supp_{w'}^2(c)$, over all vectors $w'$ that satisfy the condition above. \label{balanced2}
\end{enumerate}

\begin{proof}[Proof of Lemma~\ref{lem:balanced}]
The first statement is simply a special case of the second statement, for $r=1$. 
We temporarily skip the second statement and start by proving the third one, which we do by contradiction. 
We thus assume there is a voter $n\in N$ and two candidates $c\in C_n$, $c'\in C_n\cap A$, such that $w_{nc}>0$ and $supp_w(c) > supp_w(c')$. 
Let $\eps:=\min\{w_{nc}, (supp_w(c)-supp_w(c'))/2\}>0$, and define a new vector $w'\in\mathbb{R}^E$ where $w'_{nc}=w_{nc}-\eps$, $w'_{nc'}=w_{nc'}+\eps$, and $w'_e=w_e$ for every other edge $e\in E$. 
It can be checked that $w'$ is non-negative and feasible, and that it does not decrease the sum of member supports, i.e. $\sum_{d\in A} supp_{w'}(d)\geq \sum_{d\in A} supp_{w}(d)$, where the inequality is tight only if $c\in A$. We assume this to be the case, as otherwise the first condition on vector $w$ being balanced for $A$ is violated. On the other hand,
\begin{align*}
    \sum_{d\in A} supp_{w}^2(d) - \sum_{d\in A} supp_{w'}^2(d) &= supp_w^2(c) + supp_w^2(c) - supp_{w'}^2(c) - supp_{w'}^2(c') \\
    &= supp_w^2(c) + supp_w^2(c') - (supp_w(c)-\eps)^2 - (supp_w(c')+\eps)^2 \\
    &= 2\eps( supp_w(c) - supp_w(c')) -2\eps^2 \\
    &\geq 2\eps(2\eps) - 2\eps^2 = 2\eps^2 >0,
\end{align*}
which contradicts the second condition on vector $w$ being balanced for $A$.

We now prove the second statement, which says that function 
$F_{w'}^k:=\min_{A'\subseteq A, |A'|=r} \quad \sum_{c\in A'} supp_{w'}(c)$ 
is maximized by vector $w$ over all feasible vectors $w'\in\R^E$, for all $1\leq r\leq |A|$. 
Assume by contradiction that there is a threshold $r$ and feasible $w'$ such that $F_r^A(w')>F_r^A(w)$. We can also assume without loss of generality that 
\begin{enumerate}
    \item $\sum_{c\in A} supp_{w'}(c)=\sum_{c\in A} supp_{w}(c)$, i.e. $w'$ also satisfies the first condition of a balanced weight vector; and  
    \item we enumerate the candidates in $A=\{c_1, \cdots, c_{|A|}\}$ so that whenever $i<j$ we have that $supp_w(c_i)\leq supp_w(c_j)$, and in case of a tie, $supp_w(c_i)= supp_w(c_j)$, we have that $supp_{w'}(c_i)\leq supp_{w'}(c_j)$. 
\end{enumerate}

With a candidate enumeration as above, it follows that $F^A_r(w)=\sum_{i=1}^r supp_w(c_i)$, while for vector $w'$ we have the inequality $F^A_r(w')\leq \sum_{i=1}^r supp_{w'}(c_i)$. 
Thus, by our assumption by contradiction, 
$$\sum_{i=1}^r supp_{w'}(c_i) \geq F_r^A(w') > F_r^A(w) = \sum_{i=1}^r supp_{w}(c_i), \quad \text{and}$$
\begin{align*}
    \sum_{i=r+1}^{|A|} supp_{w'}(c_i) &= \sum_{i=1}^{|A|} supp_{w'}(c_i) - \sum_{i=1}^{r} supp_{w'}(c_i) \\
    & = \sum_{i=1}^{|A|} supp_{w}(c_i) - \sum_{i=1}^{r} supp_{w'}(c_i) \\
    & < \sum_{i=1}^{|A|} supp_{w}(c_i) - \sum_{i=1}^{r} supp_{w}(c_i) 
    = \sum_{i=r+1}^{|A|} supp_{w}(c_i). \\
\end{align*}
Now define the edge vector $f:=w'-w\in\mathbb{R}^E$ and consider it as a flow over the network $(N\cup A, E)$. 
We have that $f$ has a zero net demand over set $N$, by our first assumption, and the previous two inequalities show that $f$ has a positive net demand over set $\{c_1, \cdots, c_r\}$ and a positive net excess over set $\{c_{r+1}, \cdots, c_{|A|}\}$. Thus, by the flow decomposition theorem, $f$ can be decomposed into circulations and simple paths, where every path starts in a vertex with positive demand and ends in a vertex with positive excess, and there must be a simple path carrying some $\delta>0$ flow from $c_j$ to $c_i$ for some $1\leq i\leq r<j\leq |A|$. 
Moreover, by our second assumption, it must be the case that $supp_w(c_i)<supp_w(c_j)$, because in case of a tie we would have that $supp_{w'}(c_i)<supp_{w'}(c_j)$ and so $f$ would have a net excess on $c_i$ and net demand on $c_j$, and the $c_i$-to-$c_j$ path would not exist in the flow decomposition.

Let $f'$ be a sub-flow of $f$ that carries $\eps$ units of flow from $c_j$ to $c_i$, where $\eps:=\min\{\delta, \frac{1}{2}(supp_w(c_j) - supp_w(c_i))\}>0$. 
The fact that $f'$ is a sub-flow of $f$ implies that $w'':=w+f'$ is a feasible non-negative solution,%
\footnote{For a proof of feasibility and non-negativity, see the argument at the end of the proof in Lemma~\ref{lem:2sols}.} 
with 
\begin{align*}
    \sum_{c\in A} supp_{w}^2(c) - \sum_{c\in A} supp_{w''}^2(c) &= supp_{w}^2(c_i) + supp_w^2(c_j) - (supp_{w}(c_i) +\eps)^2 - (supp_{w}(c_j) - \eps)^2\\
    &= 2\eps(supp_w(c_j) - supp_w(c_i)) - 2\eps^2 \\
    &\geq 2\eps(2\eps) - 2\eps^2 = 2\eps^2>0,
\end{align*}
which contradicts the fact that $w$ minimizes the sum of supports squared. This completes the proof. 
\end{proof}

\subsection{Algorithms to compute balanced weight vectors}

Let $E_A\subseteq E$ be the restriction of the input edge set over edges incident to committee $A$, and let $D\in\{0,1\}^{A\times E_A}$ be the vertex-edge incidence matrix for $A$. 
For any weight vertex $w\in\R^{E_A}$, the support that $w$ assigns to candidates in $A$ is given by vector $Dw$, so that $suppw(c)=(Dw)_c$ for each $c\in A$. 
We can now write the problem of finding a balanced weight vector as a convex program:
\begin{align*}
    \text{Minimize} \quad & \|Dw\|^2 \\
    \text{Subject to } \quad & w\in\R^{E_A}, \\
    & \sum_{c\in C_n} w_{nc} \leq s_n \quad \text{for each } n\in N, \text{ and} \\
    & \mathbbm{1}^{\intercal} Dw = \sum_{n\in \cup_{c\in A} N_c} s_n,
\end{align*}
where the first two lines of constraints corresponds to non-negativity and feasibility conditions (see inequality~\ref{eq:feasible}), and the last line ensures that the sum of supports is maximized, where $\mathbbm{1}\in\mathbb{R}^A$ is the all-ones vector. 
A balanced weight vector can thus be computed with numerical methods for quadratic convex programs.

However, there is a more efficient method using techniques for parametric flow, which we sketch now. Hochbaum and Hong~\cite[Section 6]{hochbaum1995strongly} consider a network resource allocation problem which generalizes the problem of finding a balanced weight vector: given a network with a single source, single sink and edge capacities, minimize the sum of squared flows over the edges reaching the sink, over all maximum flows. 
They show that this is equivalent to a parametric flow problem called \emph{lexicographically optimal flow}, studied by Gallo, Gregoriadis and Tarjan~\cite{gallo1989fast}. 
In turn, in this last paper the authors show that, even though a parametric flow problem usually requires solving several consecutive max-flow instances, this particular problem can be solved running a single execution of the FIFO preflow-push algorithm proposed by Goldberg and Tarjan~\cite{goldberg1988new}.

Therefore, the complexity of finding a balanced weight vector is bounded by that of Goldberg and Tarjan’s algorithm, which is $O(n^3)$ for a general $n$-node network. 
However, Ahuja et al.~\cite{ahuja1994improved} showed how to optimize several popular network flow algorithms for the case of bipartite networks, where one of the partitions is considerably smaller than the other. Assuming the sizes of the bipartition are $n_1$ and $n_2$ with $n_1 \ll n_2$, they implement a two-edge push rule that allows one to "charge" most of the computation weight to the nodes on the small partition, and hence obtain algorithms whose running times depend on $n_1$ rather than $n$. 
In particular, they show how to adapt Goldberg and Tarjan’s algorithm to run in time $O(e\cdot n_1+n_1^3)$, where $e$ is the number of edges. 
For our particular problem, which can be defined on a bipartite graph $(N\cup A, E_A)$ where $|A|\leq k\ll |N|$, we obtain thus an algorithm that runs in time $O(|E_A|\cdot k + k^3)$.