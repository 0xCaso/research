\section{Exploiting local optimality}\label{s:local}

We start the section with an key property of algorithm $\balanced$ as motivation.

\begin{theorem}\label{thm:315guarantee}
If $(A,w)$ is a balanced solution such that $supp_w(A)\geq \max_{c'\in C\setminus A} score(c')$, then it satisfies PJR and guarantees a 3.15-factor approximation for the maximin support problem. 
Furthermore, testing the required conditions (feasibility, balancedness and the previous inequality) can be done in time $O(|E|)$. 
Finally, the output solution of the $\balanced$ algorithm is guaranteed to satisfy these conditions.
\end{theorem}
\begin{proof}
The first statement follows from Lemmas \ref{lem:localopt} and \ref{lem:candidate315}, and the third one from Lemma~\ref{lem:315localoptimality}. 
Feasibility (inequality~\ref{eq:feasible}) can clearly be checked in time $O(|E|)$, as can balancedness by Lemma~\ref{lem:balanced}, because properties 2 and 3 in that lemma can both be tested in this time. 
Finally, if $t:=supp_w(A)$, the inequality in the statement is equivalent to $t\geq \max_{c'\in C\setminus A} prescore(c',t)$, which is tested with algorithm $\maxprescore(A,w,t)$ in time $O(|E|)$ by Theorem~\ref{thm:runtimes}.
\end{proof}

This in turn proves Theorem~\ref{thm:intro2}. 
A result such as the one above is essential in a scenario where a computationally limited user (the \emph{verifier}) offloads a heavy task -- in this case an election algorithm -- to one or more external entities with more computational power (the \emph{prover}). Yet, as the task is executed privately and the user does not trust the entities, the user must be in condition to provably and efficiently check the quality of the output.
%
In the case of an election protocol over a decentralized blockchain, it would be very costly to run the full election algorithm as an \emph{on-chain} process, meaning that validators must execute it simultaneously and the chain cannot progress until they have finished the execution and agreed on the output. 
Instead, a much more scalable solution is to execute the protocol \emph{off-chain}, meaning that one or more parties run it privately and separate from block production, and then submit the output which is approved if it passes the verification test on-chain. 
The fact that algorithm $\balanced$ can have both of its guarantees efficiently verified on it outputs is in our opinion its most relevant feature. 

The inequality $supp_w(A)\geq \max_{c'\in C\setminus A} score(c')$ mentioned in Lemma~\ref{lem:localopt} and Theorem~\ref{thm:315guarantee} corresponds to a notion of local optimality, for a local search variant of the $\phragmms$ heuristic. 
In Section~\ref{s:tPJR} we explore the relation between this notion and the PJR property. 
Then, in Section~\ref{s:LS} we use the corresponding local search algorithm to devise a post-computation which takes an arbitrary solution $(A,w)$ as input, and returns an output $(A',w')$ which observes $supp_{w'}(A')\geq supp_w(A)$ and provably satisfies PJR.

\subsection{A parametric version of proportional justified representation}\label{s:tPJR}

We define next a parametric version of the PJR property that measures just how well represented the voters are by a given committee $A$. It is a generalization of the property which turns it from binary to quantitative.

\begin{definition}
For any $t\in\R$, a committee $A\subseteq C$ (of any size) satisfies Proportional Justified Representation with parameter $t$ ($t$-PJR for short) if there is no group $N'\subseteq N$ of voters and integer $0<r\leq |A|$ such that:
\begin{itemize}
\item[a)] $\sum_{n\in N'} s_n \geq r\cdot t$,
\item[b)] $|\cap_{n\in N'} C_n|\geq r$, and
\item[c)] $|A\cap (\cup_{n\in N'} C_n)|<r$.
\end{itemize}
\end{definition}

In words, if there is a group $N'$ of voters with at least $r$ commonly approved candidates, who can afford to provide these candidates with a support of value $t$ each, then this group is represented by at least $r$ members in committee $A$, though not necessarily commonly approved. Notice that the standard version of PJR is equivalent to $\hat{t}$-PJR for $\hat{t}:=\sum_{n\in N} s_n / |A|$ (see Section~\ref{s:prel}), and that if a committee satisfies $t$-PJR then it also satisfies $t'$-PJR for each $t'\geq t$, i.e.~the property gets stronger as $t$ decreases. 
Notice as well that this is in contrast to the maximin support objective, which implies a stronger property as it increases; we will exploit the opposite natures of these two parameters of quality in Section~\ref{s:LS} to define a procedure that improves upon a given solution.

%For instance, in the example described in Figure~\ref{fig:phragmen}, the optimal committee satisfies $(1/H_k)$-PJR, while the output of $\phragmen$ does not even satisfy $1$-PJR. However, since in this case we have that $\hat{t}=\frac{k+(1/H_k)}{k}=1+\frac{1}{k\cdot H_k}>1$ and the output does satisfy $\hat{t}$-PJR, it narrowly satisfies PJR. This example shows how much more expressive our parameterized definition of PJR is to assess the quality of a committee. 

We remark that in~\cite{sanchez2017proportional} the related notion of \emph{average satisfaction} is introduced, also to quantify the level of proportional representation achieved by a committee. Roughly speaking, that notion measures the average number of representatives in the committee that each voter in a group has, for any group of voters with sufficiently high vote strength and cohesiveness level. 
In contrast, with parametric PJR we focus on providing sufficient representatives to the group as a whole and not to each individual voter, and we measure the required vote strength for a group to achieve it.
Interestingly, the average satisfaction measure is closely linked to the property of extended justified representation (EJR)~\cite{aziz2017justified}, and in~\cite{aziz2018complexity} that measure is used to prove that a local search algorithm achieves EJR. 
Similarly, we use parametric PJR to prove that a local search algorithm achieves standard PJR.

Testing whether an arbitrary solution satisfies standard PJR is known to be coNP-complete~\cite{aziz2018complexity}, hence the same remains true for its parametric version.
We provide next a sufficient condition for a committee to satisfy $t$-PJR, which is efficiently testable, based on our definitions of pre-score and score.  

\begin{lemma} \label{lem:locality}
If for a feasible solution $(A,w)$ there is a parameter $t\in\R$ such that $\max_{c'\in C\setminus A} prescore(c',t)<t$, or equivalently, such that $\max_{c'\in C\setminus A} score(c') <t$, then committee $A$ satisfies $t$-PJR. 
\end{lemma}

\begin{proof} 
We prove the contrapositive of the claim. If $A$ does not satisfy $t$-PJR, there must be a subset $N'\subseteq N$ of voters and an integer $r>0$ that observe points a) through c) in the definition above. By points b) and c), set $(\cap_{n\in N'} C_n)\setminus A$ must be non-empty: let $c'$ be a candidate in it. 
We will prove that for any feasible weight vector $w\in \R^E$, it holds that $prescore(c',t)\geq t$, and consequently $score(c')\geq t$ by the definition of score. We have
%
\begin{align*} 
prescore(c',t) &= \sum_{n\in N_{c'}}  slack(n,t) \geq \sum_{n\in N'} slack(n,t) & (\text{as } N'\subseteq N_{c'}) \\
&\geq \sum_{n\in N'} \Big(s_n - t\cdot \sum_{c\in A\cap C_n} \frac{w_{nc}}{supp_w(c)}\Big)  
& (\text{by definition \ref{eq:slack}})\\
&= \sum_{n\in N'} s_n - t \cdot \sum_{c\in A\cap (\cup_{n\in N'} C_n)} 
\frac{\sum_{n\in N'\cap N_c} w_{nc}}{\sum_{n\in N_c} w_{nc}} 
& (\text{where fraction is } \leq 1)\\ 
& \geq t\cdot r - t\cdot |A\cap (\cup_{n\in N'} C_n)| & (\text{by a)})\\
& \geq t\cdot r - t\cdot (r-1) = t & (\text{by c)}). 
\end{align*}
%
This proves that $prescore(c',t) \geq t$, which is what we needed to show.
\end{proof}

For a given solution $(A,w)$ and parameter $t$, one can verify the condition above in time $O(|E|)$ by checking whether $\maxprescore(A,w,t)<t$; see Theorem~\ref{thm:runtimes}. Alternatively, from just solution $(A,w)$ one can execute $\maxscore(A,w)$ in time $O(|E|\cdot \log k)$ to obtain the highest score $t_{\max}$ and ascertain that $A$ satisfies $t$-PJR for each $t>t_{\max}$. 
The proof of Lemma~\ref{lem:localopt} now follows as a corollary.

\begin{proof}[Proof of Lemma~\ref{lem:localopt}]
Let $t_{\max}:=\max_{c'\in C\setminus A} score(t_{\max})$ and let $c_{\max}\in C\setminus A$ be a candidate with such highest score. 
If $supp_w(A)\geq t_{\max}$, it follows from Lemma~\ref{lem:insert} that if we execute $\ins(A,w,c_{\max}, t_{\max})$, we obtain a solution $(A+c_{\max}, w')$ with $supp_{w'}(A+c_{\max})=t_{\max}$. 
Now, by feasibility of vector $w'$, we have the inequality $\sum_{n\in N} s_n \geq \sum_{c\in A+c_{\max}} supp_{w'}(c) \geq (k+1)\cdot t_{\max}$, and thus $t_{\max}\leq \sum_{n\in N} s_n / (k+1) < \sum_{n\in N} s_n / k = \hat{t}$. 
By Lemma~\ref{lem:locality} above, having $\max_{c'\in C\setminus A} score(t_{\max}) < \hat{t}$ implies that $A$ satisfies $\hat{t}$-PJR, which is standard PJR.
\end{proof}

\subsection{A local search algorithm}\label{s:LS}

Suppose that we know of an efficient algorithm with good guarantees for maximin support but no guarantee for PJR, or we happen to know of a high quality solution in terms of maximin support but we ignore if it satisfies PJR. Can we use it to find a new solution of no lesser quality which is also guaranteed to satisfy PJR? And can we efficiently prove to a verifier that the new solution indeed satisfies PJR? We answer these questions in the positive for the first time.

We present a local search algorithm that takes an arbitrary feasible solution as input, and iteratively drops a member of least support and inserts a new candidate using the $\phragmms$ heuristic. The procedure always maintains or increases the value of the least member support, hence the quality of the solution is preserved. Furthermore, as the solution converges to a local optimum, it is guaranteed to satisfy the PJR after a certain number of iterations. 
Therefore, when used as a post-computation, this procedure can make any approximation algorithm for maximin support also satisfy PJR in a black-box manner. 
We remark here that such an application of the $\phragmms$ heuristic goes to show the robustness of the election rule; in particular, there is no evident way to build a similar post-computation from $\phragmen$~\cite{brill2017phragmen}, as the analysis of the PJR property in that rule is heavily dependent on the way the solution is constructed. 

As we did in Section~\ref{s:inserting}, in the following algorithm we assume that the background instance $(G=(N\cup C, E), s, k)$ is known and that does not need to be passed as input. 
Instead, the input is a feasible full solution $(A,w)$, and a parameter $\eps>0$. 
Our proposed algorithm $\local$ (Algorithm~\ref{alg:localpjr}) is presented below. 

\begin{algorithm}[htb]\label{alg:localpjr}
\SetAlgoLined
\KwData{Full feasible solution $(A,w)$, parameter $\eps>0$.}

Let $\hat{t}\leftarrow \sum_{n\in N} s_n / |A|$\;
\While{True}{
  Find tuple $(c_{\min}, t_{\min})$ so that $c_{\min}\in A$ and $t_{\min}=supp_w(c_{\min})=supp_w(A)$\;
  Let $(c_{\max}, t_{max})\leftarrow \maxscore(A,w)$ \quad \emph{// the candidate with highest score, and its score}\;
  \lIf {($t_{\max}< \min\{ (1+\eps)\cdot t_{\min}, \hat{t}\}$)} {\Return $(A,w)$}
  Update $(A,w)\leftarrow \ins(A-c_{\min},w,c_{\max},t_{\max})$\;
}
\caption{$\LSPJR(A,w,\eps)$}
\end{algorithm}

\begin{theorem}\label{thm:enabler}
For any parameter $\eps>0$ and a feasible full solution $(A,w)$, let $(A',w')$ be the output solution to $\LSPJR(A,w,\eps)$. Then: 
\begin{enumerate}
    \item solution $(A',w')$ is feasible and full, and $supp_{w'}(A')\geq supp_w(A)$; \label{item:support}
    \item solution $(A', w')$ satisfies the condition on Lemma~\ref{lem:locality} for parameter $t=\min\{(1+\eps)\cdot supp_{w'}(A'), \hat{t}\}$, where $\hat{t}=\sum_{n\in N} s_n / k$, so $A'$ satisfies $t$-PJR and standard PJR; 
		\label{item:tPJR}
		\item if $(A,w)$ has an $\alpha$-approximation guarantee for the maximin support objective, for some $\alpha\geq 1$, then the algorithm performs at most $k\cdot (1+\log_{1+\eps} \alpha)+1$ iterations, each taking time $O(|E|\cdot \log k)$; and \label{item:iterations}
		\item by setting $\eps\rightarrow\infty$, the algorithm finds a solution satisfying standard PJR in at most $k+1$ iterations. \label{item:infinity}

\end{enumerate}
\end{theorem}

This proves Theorem~\ref{thm:intro3}. 
Notice that point~\ref{item:iterations} establishes that $\LSPJR$ can be executed as a post-computation of any constant-factor approximation algorithm for maximin support in time $O(|E|\cdot k\log k)$. In particular, this complexity is dominated by that of all constant-factor approximations presented in this paper.
By point~\ref{item:infinity}, the algorithm can be sped up if we only care about standard PJR, or run further iterations to achieve a stronger parametric PJR guarantee. 
In the proof, we use the observation below whose proof is skipped as it follows from the definitions of slack and pre-score.

\begin{lemma}\label{lem:sameweight}
For a feasible vector $w\in \R^E$, two committees $A\subseteq A'$, and any threshold $t\geq 0$ it holds that $prescore_{(A,w)}(c',t)\geq prescore_{(A',w)}(c',t)$ for each candidate $c'\in C\setminus A'$.
\end{lemma}

\begin{proof}[Proof of Theorem~\ref{thm:enabler}]
We start with some notation. We use $i$ as a superscript to indicate the value of the different variables at the beginning of the the $i$-th iteration. We define $t^i_{stop}:=\min\{(1+\eps)\cdot t^i_{min}, \hat{t}\}$, so that the stopping condition reads $t^i_{\max} \stackrel{?}{<} t^i_{stop}$. Notice that $t^i_{stop}\geq t^i_{min}$ always holds by feasibility of $w^i$ and definition of $\hat{t}$. 

We prove point~\ref{item:support} by induction on $i$. 
If the stopping condition does not hold then $t^i_{\max} \geq t^i_{stop}\geq t^i_{\min}$. 
Next, by Lemma~\ref{lem:sameweight} we have $prescore_{(A^i-c_{\min}^i, w^i)}(c^i_{\max}, t^i_{\max}) \geq prescore_{(A^i, w^i)}(c^i_{\max}, t^i_{\max})=t^i_{\max}\geq t^i_{\min}$. 
On the other hand, it is evident that $supp_{w^i}(A^i-c^i_{\min})\geq supp_{w^i}(A^i)=t^i_{\min}$. 
Therefore, by Lemma~\ref{lem:insert} we have that  
$$supp_{w^{i+1}}(A^{i+1})\geq \min\{supp_{w^i}(A^i-c^i_{\min}), prescore_{(A^i-c_{\min}^i, w^i)}(c^i_{\max}, t^i_{\max})\} \geq t^i_{\min},$$
and that $(A^{i+1},w^{i+1})$ is feasible.  
This proves point~\ref{item:support}. Point~\ref{item:tPJR} is apparent from the stopping condition. 

We continue to point \ref{item:iterations}. Each iteration is dominated in complexity by the call to Algorithm $\maxscore$, which takes time $O(|E|\cdot \log k)$ by Theorem~\ref{thm:runtimes}. 
Hence, it remains to give a bound on the number $T$ of total iterations. 
To do that, we analyze the evolution of the least member support $t^i_{\min}=supp_{w^i}(A^i)$. First, we argue that if ever $t^i_{\min}=\hat{t}$, then the algorithm terminates immediately, i.e.~$i=T$; this is because in this case all members in $A^i$ must have a support of exactly $\hat{t}$, all voters a zero slack for threshold $\hat{t}$, and all candidates in $C\setminus A^i$ a zero prescore for threshold $\hat{t}$ and hence a score strictly below $\hat{t}$, and the stopping condition is fulfilled.
Next, we claim that for any iteration $i$ with $1\leq i<T-k$, we have $t^{i+k}_{\min}\geq (1+\eps)\cdot t^{i}_{\min}$. This is because, by Lemma~\ref{lem:insert}, in each iteration $j\geq i$ we are removing a member with least support while not increasing the number of members with support below 
$$prescore_{(A^j - c^j_{\min}, w^j)}(c^j_{\max}, t^j_{\max})\geq t^j_{\max}\geq t^j_{stop}\geq t^i_{stop}.$$
 
As there are only $k$ candidates, we must have that 
$t^{i+k}_{\min}\geq t^i_{stop}=\min\{(1+\eps)\cdot t^i_{\min}, \hat{t}\}$. 
Thus $t^{i+k}_{\min}$ is either at least $(1+\eps)\cdot t^i_{\min}$, or it is $\hat{t}$, but it cannot be the latter by the previous claim and the fact that $i+k<T$. This proves the new claim. Therefore, if the algorithm outputs a solution $(A^T,w^T)$ and has an input solution $(A^1, w^1)$ with an $\alpha$-approximation guarantee for the maximin support objective, then 
$$\alpha\cdot supp_{w^1}(A^1) \geq supp_{w^T}(A^T) \geq (1+\eps)^{\lfloor (T-1)/k \rfloor} \cdot supp_{w^1}(A^1).$$
Hence, $\alpha\geq (1+\eps)^{\lfloor (T-1)/k \rfloor}$, and $T\leq k\cdot(1+\log_{1+\eps} \alpha)+1$. This completes the proof of point \ref{item:iterations}.

For point \ref{item:infinity}, we consider again the previous inequality $t^{i+k}_{\min}\geq t^i_{stop}=\min\{(1+\eps)\cdot t^i_{\min}, \hat{t}\}$. By setting $i=1$ and $\eps\rightarrow \infty$, we have that either the stopping condition is triggered before the $(k+1)$-st iteration, or it must be the case that $t^{k+1}_{\min}\geq \hat{t}$, and the algorithm terminates immediately as argued previously. In either case, the claim on standard PJR follows from point~\ref{item:tPJR}.
\end{proof}