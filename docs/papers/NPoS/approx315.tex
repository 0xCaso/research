\subsection{A faster constant-factor approximation algorithm for maximin support}\label{s:315}

We proved in Section~\ref{s:complexity} that a 2-approximation algorithm for maximin support can be computed in time $O(\bal\cdot |C|\cdot k)$ or $O(\bal\cdot |C|)$ (Theorems \ref{thm:mms} and \ref{thm:2eps} respectively). 
We use the $\phragmms$ heuristic to develop a $3.15$-approximation algorithm that runs in time $O(\bal\cdot k)$, and satisfies PJR as well. 
We highlight that this is the fastest known algorithm to achieve a constant-factor guarantee for maximin support, and that gains in speed are of paramount importance for our blockchain application, where there are hundreds of candidates and a massive number of voters.

We propose $\balanced$ (Algorithm~\ref{alg:balanced}), an iterative greedy algorithm that starts with an empty committee and alternates between inserting a new candidate with the $\phragmms$ heuristic, and \emph{fully rebalancing} the weight vector, i.e., replacing it with a balanced one. This constitutes a middle ground between the approach taken in $\phragmen$ where the weight vector is never rebalanced, and the approach in $\MMS$ where $O(|C|)$ balanced vectors are computed in each iteration. 
We formalize the procedure below. Notice that running the Insert procedure (Algorithm~\ref{alg:ins}) before rebalancing is optional, but the step simplifies the analysis and may provide a better starting point to the balancing algorithm.

\begin{algorithm}[htb]\label{alg:balanced}
\SetAlgoLined
\KwData{Approval graph $G=(N\cup C, E)$, vector $s$ of vote strengths, committee size $k$.}
Initialize $A=\emptyset$\ and $w=0\in\R^E$\;
\For{$i$ from $1$ to $k$}{
Let $(c_{\max},t_{\max})\leftarrow \maxscore(A,w)$ \quad \emph{// the candidate with highest score, and its score} \;
Update $(A,w)\leftarrow \ins(A,w,c_{\max},t_{\max})$ \quad \emph{// or optionally just update $A\leftarrow A+c_{\max}$} \;
%Update $A\leftarrow A+c_{\max}$\;
Replace $w$ with a balanced weight vector for $A$\;
}
\Return $(A,w)$\;
\caption{$\balanced$}
\end{algorithm}

\begin{theorem}\label{thm:315}
$\balanced$ offers a $3.15$-approximation guarantee for the maximin support problem, satisfies the PJR property, and executes in time $O(\bal\cdot k)$, assuming that $\bal= \Omega(|E|\cdot \log k)$.
\end{theorem}

This in turn proves Theorem~\ref{thm:intro1}. 
The claim on runtime is straightforward: we established in Theorem~\ref{thm:runtimes} that the $\phragmms$ heuristic runs in time $O(|E|\cdot \log k)$, so each one of the $k$ iterations of $\balanced$ has a runtime of $O(|E|\cdot \log k + \bal)=O(\bal)$, where in the last equality we assume that $\bal= \Omega(|E|\cdot \log k)$. 
In fact, in Appendix~\ref{s:algorithms} we improve upon this analysis and prove that each iteration can be made to run in time $O(|E| + \bal)$.
Next, in order to prove the PJR property we need the following technical lemmas.

\begin{lemma}\label{lem:2balanced}
If $(A,w)$ and $(A',w')$ are two balanced partial solutions with $A\subseteq A'$, then $\supp_w(c)\geq \supp_{w'}(c)$ for each candidate $c\in A$. 
Furthermore, for each $c'\in C\setminus A'$ we have that $\score_{(A,w)}(c')\geq \score_{(A',w')}(c')$.
\end{lemma}

\begin{lemma}\label{lem:localopt}
If $\supp_w(A)\geq \max_{c'\in C\setminus A} \score(c')$ holds for a full solution $(A,w)$, then $A$ satisfies PJR.
\end{lemma}

Lemma \ref{lem:2balanced} formalizes the intuition that as more candidates are added to committee $A$, supports and scores can only decrease, never increase; its proof is delayed to Appendix~\ref{s:proofs}.
Lemma~\ref{lem:localopt} provides a sense of local optimality to solution $(A,w)$, because if the corresponding inequality did not hold we could attempt to improve the solution by swapping the member with least support for the unelected candidate with highest score; its proof is delayed to Section~\ref{s:tPJR} where we explore this notion of local optimality and its relation with the PJR property. 
We prove now that the output of $\balanced$ satisfies the condition in Lemma~\ref{lem:localopt}, and hence satisfies PJR.

\begin{lemma}\label{lem:315localoptimality}
At the end of each one of the $k$ iterations of Algorithm $\balanced$, if $(A,w)$ is the current partial balanced solution, we have that $\supp_{w}(A)\geq \max_{c'\in C\setminus A} \score_{(A,w)}(c')$.
\end{lemma}
\begin{proof}
Let $(A_i,w_i)$ be the partial solution at the end of the $i$-th iteration. We prove the claim by induction on $i$, where the base case for $i=0$ holds trivially as we use the convention that $\supp_{w_0}(\emptyset)=\infty$ for any $w_0$. For $i\geq 1$, suppose that on iteration $i$ we insert a candidate $c_i$ with highest score, and let $w'$ be the weight vector obtained from running $\ins(A_{i-1}, w_{i-1}, c_i, \score_{(A_{i-1}, w_{i-1})}(c_i))$ (Algorithm~\ref{alg:ins}). Then

\begin{align*}
\supp_{w_i}(A_i) &\geq \supp_{w'}(A_i) &\text{(as $w_i$ is balanced for $A_i$)}\\
&\geq \min\{ \supp_{w_{i-1}}(A_{i-1}), \score_{(A_{i-1}, w_{i-1})}(c_i) \} &\text{(by Lemma~\ref{lem:insert})}\\
&\geq \max_{c'\in C\setminus A_{i-1}} \score_{(A_{i-1}, w_{i-1})}(c') &\text{(by induction hyp.~and choice of $c_i$)}\\
&\geq \max_{c'\in C\setminus A_{i}} \score_{(A_{i}, w_{i})}(c'). &\text{(by Lemma~\ref{lem:2balanced})}
\end{align*}
This completes the proof.
\end{proof}


It remains to prove the claimed approximation guarantee of Algorithm $\balanced$. 
To do that, we need the following technical result which, informally speaking, establishes that if a partial solution is balanced, then there must be a subset of voters with large slacks, and neighboring unelected candidates with high pre-scores and scores.   

\begin{lemma}\label{lem:N_a}
Let $(A^*, w^*)$ be an optimal solution to the maximin support instance, $t^*=\supp_{w^*}(A^*)$, and let $(A,w)$ be a balanced solution with $|A|\leq k$ and $A\neq A^*$. For each $0\leq a\leq 1$, there is a subset $N(a)\subseteq N$ of voters such that 
\begin{enumerate}
	\item each voter $n\in N(a)$ has a neighbor in $A^*\setminus A$;
	\item for each voter $n\in N(a)$, we have that $\supp_w(A\cap C_n):=\min_{c\in A\cap C_n} \supp_w(c)\geq at^*$;
	\item $\sum_{n\in N(a)} s_n \geq |A^* \setminus A|\cdot (1-a) t^*$; and
	\item for any $b$ with $a\leq b\leq 1$ we have that $N(b)\subseteq N(a)$, and for each $n\in N(a)$ we have that $n$ is also in $N(b)$ if and only if property 2 above holds for $n$ with parameter $a$ replaced by $b$.
\end{enumerate}
\end{lemma}

\begin{proof}
Fix a parameter $0\leq a\leq 1$ and define the set $N':=\{n\in N: \ \supp_w(A\cap C_n)\geq at^*\}$, where $\supp_w(\emptyset)=\infty$ by convention. If we define $N(a)\subseteq N'$ as those voters in $N'$ that have a neighbor in $A^*\setminus A$, then properties 1, 2 and 4 become evident. Hence, it only remains to prove the third property.

We claim that there is no edge with non-zero weight between $N\setminus N'$ and $A':=\{c\in A: \ \supp_w(c)\geq at^*\}$. 
Indeed, if there was a pair $n\in N\setminus N'$, $c\in A'$ with $w_{nc}>0$, then by point 3 of Lemma~\ref{lem:balanced} we would have that $\supp_w(A\cap C_n)=\supp_w(c)\geq at^*$, contradicting the fact that $n\not\in N'$. 
Thus, we get the inequality
$$\sum_{n\in N\setminus N'} s_n \leq \sum_{c\in A\setminus A'} \supp_w(c) < |A\setminus A'|\cdot at^*< |A^*\setminus A'|\cdot at^*.$$

By reducing some components in vectors $w$ and $w^*$, we can assume without loss of generality that $\supp_{w^*}(c)=t^*$ if $c\in A^*$, zero otherwise, and $\supp_{w}(c)=a t^*$ if $c\in A^*\cap A'$, zero otherwise -- we call this the ``wlog assumption''.
Define $f:=w^* - w\in\mathbb{R}^E$, which we interpret as a vector of flows over the network induced by $N\cup A^*$, with positive signs corresponding to flow leaving $N$, and vice-versa. We partition the network nodes into four sets: $N'$, $N\setminus N'$, $A^*\cap A'$, and $A^*\setminus A'$. Relative to $f$, we have that a) $N$ has a net excess of $|A^*|\cdot t^* - |A^*\cap A'|\cdot a t^*$, b) $N\setminus N'$ has a net excess of at most $\sum_{n\in N\setminus N'} s_n < |A^*\setminus A'|\cdot at^*$ (by the previous inequality), and c) $A^*\cap A'$ has a net demand of $|A^*\cap A'|\cdot (1-a) t^*$. 
Using the flow decomposition theorem, we can decompose flow $f$ into circulations and simple paths. If we define $f'$ to be the sub-flow of $f$ that contains only the simple paths that start in $N'$ and end in $A^*\setminus A'$, then %
%
\begin{align*}
    \text{flow value in } f' &\geq (\text{ net excess in $N$ } - \text{ net excess in $N\setminus N'$ } - \text{ net demand in $A^*\cap A')$ w.r.t. } f\\
    &\geq |A^*|\cdot t^* - |A^*\cap A'|\cdot a t^* -|A^*\setminus A'|\cdot a t^* -|A^*\cap A'|\cdot (1-a) t^*\\
    & = |A^*\setminus A'|\cdot (1-a)t^* \geq |A^* \setminus A|\cdot (1-a)t^*.
\end{align*}

A key observation now is that none of the flow in $f'$ can pass by any node in $N\setminus N'$ or $A^*\cap A\setminus A'$; see Figure~\ref{fig:sets}. 
This is because any path in $f'$ starts in $N'$, nodes in $N'$ have no neighbors in $A\setminus A'$ (by definitions of $N'$ and $A'$), and furthermore there is no flow possible from $(A^*\setminus A)\cup (A^*\cap A')$ to $N\setminus N'$ in $f=w^*-w$, because $w$ has no flow from $N\setminus N'$ toward $A'$ (by our claim) nor toward $A^*\setminus A$ (by the wlog assumption).
Therefore, the formula above is actually a lower bound on the flow going from $N'$ to $A^*\setminus A$. Finally, we notice that for each path in $f'$, the last edge goes from $N'$ to $A^*\setminus A$, so it originates in $N(a)$. This proves that $\sum_{n\in N(a)} s_n\geq \text{ flow value in } f'\geq |A^* \setminus A|\cdot (1-a) t^*$.%, which is the third property.
\end{proof}

\begin{figure}[htb]
\begin{center}
\scalebox{.55}{
\definecolor{wrwrwr}{rgb}{0.3803921568627451,0.3803921568627451,0.3803921568627451}
\definecolor{aqaqaq}{rgb}{0.6274509803921569,0.6274509803921569,0.6274509803921569}
\definecolor{cqcqcq}{rgb}{0.7529411764705882,0.7529411764705882,0.7529411764705882}
\definecolor{rvwvcq}{rgb}{0.08235294117647059,0.396078431372549,0.7529411764705882}
\begin{tikzpicture}[line cap=round,line join=round,>=triangle 45,x=1cm,y=1cm]
\clip(-6.6,-3.1) rectangle (17,7.1);
\draw [line width=2pt,color=rvwvcq] (-2.3,4) circle (1.5cm);
\draw [line width=2pt,color=rvwvcq] (-2.3,0) circle (1.5cm);
\draw [line width=2pt,color=rvwvcq] (6.8,2) circle (1.5cm);
\draw [line width=2pt,color=rvwvcq] (6.8,5.5) circle (1.5cm);
\draw [line width=2pt,color=rvwvcq] (6.8,-1.5) circle (1.5cm);
\draw [line width=2pt,dashed,color=cqcqcq] (-0.5,3.5)-- (4.8,-0.7);
\draw [line width=2pt,color=cqcqcq] (-0.5,-0.5)-- (4.8,-1.3);
\draw [line width=2pt,color=cqcqcq] (-0.5,-0)-- (4.8,1.5);
\draw [line width=2pt,color=cqcqcq] (-0.5,0.5)-- (4.8,4.5);
\draw [line width=2pt] (-0.5,4.5)--(4.8,5);
\draw [line width=2pt] (-0.5,4)-- (4.8,2.3);
\draw (3.7,0.5) node[anchor=north west] {no edges};
\draw (-0.5,1) node[anchor=north west] {no backward flow};
\draw (-2.5,4.2) node[anchor=north west] {$N'$};
\draw (-2.5,0.2) node[anchor=north west] {$N \setminus N'$};
\draw (6,5.5) node[anchor=north west] {$A^* \cap A'$};
\draw (6,2) node[anchor=north west] {$A^* \setminus A$};
\draw (6,-1.5) node[anchor=north west] {$A^*\cap A\setminus A'$};
\draw (-6.5,4.2) node[anchor=north west] {start of flow $f'$};
\draw (12,-0.5) node[anchor=north west] {end of flow $f'$};
\draw (10,0.5) node[anchor=north west] {$(A^* \setminus A) \cup (A^* \cap A\setminus A') = A^* \setminus A'$};
\draw [line width=2pt,color=wrwrwr] (8.7,3)-- (9.5,2.5)-- (9.5,-2.5)-- (8.7,-3.0);
\begin{scriptsize}
\draw [fill=black,shift={(4.8,5)},rotate=270] (0,0) ++(0 pt,3.75pt) -- ++(3.2475952641916446pt,-5.625pt)--++(-6.495190528383289pt,0 pt) -- ++(3.2475952641916446pt,5.625pt);
\draw [fill=black,shift={(-0.5,4.5)},rotate=90] (0,0) ++(0 pt,3.75pt) -- ++(3.2475952641916446pt,-5.625pt)--++(-6.495190528383289pt,0 pt) -- ++(3.2475952641916446pt,5.625pt);
\draw [fill=black,shift={(-0.5,4)},rotate=90] (0,0) ++(0 pt,3.75pt) -- ++(3.2475952641916446pt,-5.625pt)--++(-6.495190528383289pt,0 pt) -- ++(3.2475952641916446pt,5.625pt);
\draw [fill=black,shift={(4.8,2.3)},rotate=270] (0,0) ++(0 pt,3.75pt) -- ++(3.2475952641916446pt,-5.625pt)--++(-6.495190528383289pt,0 pt) -- ++(3.2475952641916446pt,5.625pt);
\draw [fill=aqaqaq,shift={(-0.5,-0.5)},rotate=90] (0,0) ++(0 pt,3.75pt) -- ++(3.2475952641916446pt,-5.625pt)--++(-6.495190528383289pt,0 pt) -- ++(3.2475952641916446pt,5.625pt);
\draw [fill=aqaqaq,shift={(4.8,-1.3)},rotate=270] (0,0) ++(0 pt,3.75pt) -- ++(3.2475952641916446pt,-5.625pt)--++(-6.495190528383289pt,0 pt) -- ++(3.2475952641916446pt,5.625pt);
\draw [fill=aqaqaq,shift={(4.8,1.5)},rotate=270] (0,0) ++(0 pt,3.75pt) -- ++(3.2475952641916446pt,-5.625pt)--++(-6.495190528383289pt,0 pt) -- ++(3.2475952641916446pt,5.625pt);
\draw [fill=aqaqaq,shift={(4.8,4.5)},rotate=270] (0,0) ++(0 pt,3.75pt) -- ++(3.2475952641916446pt,-5.625pt)--++(-6.495190528383289pt,0 pt) -- ++(3.2475952641916446pt,5.625pt);
\end{scriptsize}
\end{tikzpicture}
}
\end{center}
\caption{Flow $f'$ can only visit nodes in $N'$, $A^*\cap A'$ and $A^*\setminus A$.}
\label{fig:sets}
\end{figure}

As a warm-up, we show how this last result easily implies a $4$-approximation guarantee for $\balanced$.

\begin{lemma}
If $(A,w)$, $(A^*,w^*)$ and $t^*$ are as in Lemma~\ref{lem:N_a}, there is a candidate $c'\in A^*\setminus A$ with $\score(c')\geq t^*/4$. Therefore, $\balanced$ provides a $4$-approximation for the maximin support problem.
\end{lemma}

\begin{proof}
Apply Lemma~\ref{lem:N_a} with $a=1/2$. In what follows we refer to the properties stated in that lemma. We have that

\begin{align*}
    \sum_{c'\in A^*\setminus A} \prescore(c',t^*/4) &=\sum_{c'\in A^*\setminus A} \sum_{n\in N_{c'}} \slack(n,t^*/4)
    \geq \sum_{n\in N(a)} \slack(n,t^*/4) & \text{(by property 1)}\\
    &\geq \sum_{n\in N(a)} \Big[ s_n - \frac{t^*}{4}\sum_{c\in A\cap C_n} \frac{w_{nc}}{\supp_w(c)} \Big] &\text{(by equation \ref{eq:slack})}\\
    &\geq \sum_{n\in N(a)} \Big[ s_n - \frac{1}{2}\sum_{c\in A\cap C_n} w_{nc} \Big] &\text{(by property 2)}\\
    &\geq \frac{1}{2}\sum_{n\in N(a)} s_n 
		\geq \frac{1}{2} (|A^*\setminus A|\cdot t^*/2) = |A^*\setminus A|\cdot t^*/4. 
    &\text{(by ineq.~\ref{eq:feasible} and property 3)}
\end{align*}

Therefore, by an averaging argument, there must be a candidate $c'\in A^*\setminus A$ with $\prescore(c',t^*/4)\geq t^*/4$, which implies that $\score(c')\geq t^*/4$, by definition of score. The $4$-approximation guarantee for Algorithm $\balanced$ easily follows by induction on the $k$ iterations, using Lemma~\ref{lem:insert} and the fact that rebalancing a partial solution never decreases its least member support.
\end{proof}

To get a better approximation guarantee for Algorithm $\balanced$ and finish the proof of Theorem~\ref{thm:315}, we apply Lemma~\ref{lem:N_a} with a different parameter $a$. For this, we will use the following result whose proof is delayed to Apppendix~\ref{s:proofs}.

\begin{lemma}\label{lem:Lebesgue}
Consider a strictly increasing and differentiable function $f:\mathbb{R}\rightarrow \mathbb{R}$, with a unique root $\chi$. For a finite sum $\sum_{i\in I} \alpha_i f(x_i)$ where $\alpha_i\in\mathbb{R}$ and $ x_i\geq \chi$ for each $i\in I$, we have that
$$\sum_{i\in I} \alpha_i f(x_i) = \int_{\chi}^{\infty} f'(x) \big(\sum_{i\in I: \ x_i\geq x} \alpha_i\big)dx.$$
\end{lemma}

\begin{lemma}\label{lem:candidate315}
If $(A,w)$, $(A^*,w^*)$ and $t^*$ are as in Lemma~\ref{lem:N_a}, there is a candidate $c'\in A^*\setminus A$ with $\score(c')\geq t^*/3.15$. Therefore, $\balanced$ provides a $3.15$-approximation for the maximin support problem.
\end{lemma}

\begin{proof}
Again we apply Lemma~\ref{lem:N_a} and its properties, for a parameter $0\leq a\leq 1$ to be defined later. We have
\begin{align*}
    \sum_{c'\in A^*\setminus A} \prescore(c',at^*) &= \sum_{c'\in A^*\setminus A} \sum_{n\in N_c} \slack(n, at^*) 
    \geq \sum_{n\in N(a)} \slack(n, at^*) & \text{(by property 1)}\\
    &\geq \sum_{n\in N(a)} \Big[ s_n - at^* \sum_{c\in A\cap C_n} \frac{w_{nc}}{\supp_w(c)} \Big] &\text{(by equation \ref{eq:slack})}\\
    &\geq \sum_{n\in N(a)} \Big[ s_n - \frac{at^*}{\supp_w(A\cap C_n)} \sum_{c\in A\cap C_n} w_{nc} \Big] &\text{(by property 2)}\\
    &\geq \sum_{n\in N(a)} s_n\Big[ 1- \frac{at^*}{\supp_w(A\cap C_n)} \Big], &\text{(by ineq.~\ref{eq:feasible})}
\end{align*}
%
where $\supp_w(\emptyset)=\infty$ by convention. 
At this point, we apply Lemma~\ref{lem:Lebesgue} over function $f(x):=1-a/x$, which has the unique root $\chi=a$, and index set $I=N(a)$ with $\alpha_n=s_n$ and $x_n=\supp_w(A\cap C_n)/t^*$. We obtain
\begin{align*}
    \sum_{c'\in A^*\setminus A} \prescore(c',at^*) &\geq \int_{a}^{\infty} f'(x) \Big( \sum_{n\in N(a): \ \supp_w(A\cap C_n)\geq xt^*} s_n \Big)dx\\
    &=\int_{a}^{\infty} \frac{a}{x^2}\Big( \sum_{n\in N(x)} s_n \Big)dx & \text{(by property 4)}\\
    &\geq \int_{a}^1 \frac{a}{x^2} \Big( |A^*\setminus A|\cdot (1-x)t^* \Big)dx & \text{(by property 3)}\\
    & = |A^*\setminus A|\cdot at^* \int_{a}^1 \Big( \frac{1}{x^2} - \frac{1}{x} \Big)dx \\
		&= |A^*\setminus A|\cdot at^*\Big(\frac{1}{a} - 1 + \ln  a\Big).
\end{align*}

If we now set $a=1/3.15$, we have that $1/a - 1 + \ln a\geq 1$, and thus by an averaging argument there is a candidate $c'\in A^*\setminus A$ for which $\prescore(c',at^*)\geq at^*$, and hence $\score(c')\geq at^*$. The approximation guarantee for Algorithm $\balanced$ follows by induction on the $k$ iterations, as before.
\end{proof}
