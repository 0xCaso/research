\section{A new heuristic}\label{s:heuristic}

The $\phragmen$ heuristic~\cite{brill2017phragmen} is highly efficient, with a runtime of $O(|E|\cdot k)$; see Algorithm~\ref{alg:phragmen} in Appendix~\ref{s:algorithms}. 
However, as proved in the previous section, it fails to provide a good guarantee for the maximin support objective, whereas $\MMS$~\cite{sanchez2016maximin} gives a constant-factor guarantee albeit with a considerably worse running time.
In this section we introduce $\phragmms$, a new heuristic inspired in $\phragmen$ that maintains a comparable runtime to it, yet lends itself to more robust analyses both for the maximin support objective and for the PJR property. 

\subsection{Inserting one candidate to a partial solution}\label{s:inserting}

We start with a brief analysis of the approaches taken in $\MMS$ and $\phragmen$. 
Both are iterative greedy algorithms that start with an empty committee and add to it a new candidate over $k$ iterations, following some specific rule for candidate selection.
For a given partial solution, $\MMS$ computes a balanced weight vector for each possible augmented committee resulting from adding a candidate, and keeps the one that offers the largest support. 
Such a heuristic offers robust guarantees for maximin support but is slow as computing balanced vectors is costly. 
On the other hand, $\phragmen$ forgoes balancing and replaces it with a lazy version that performs only local modifications to the current weight vector, in the neighborhood of the newly inserted candidate. 
The $\phragmms$ heuristic takes a similar approach but uses a somewhat smarter and less lazy version of balancing, with a minimal increase in runtime. 

In all algorithms described in this section we assume that there is a known background instance $(G=(N\cup C, E), s, k)$ that does not need to be passed as input. Rather, the input is a partial solution $(A,w)$ with $|A|\leq k$. We also assume that the list of committee member supports $(\supp_w(c))_{c\in A}$ is implicitly passed by reference and updated in every algorithm, so it does not need to be recomputed every time.
%
Let $c'\in C\setminus A$ be a candidate that we consider adding to $(A,w)$. To do so, we need to modify weight vector $w$ into a new feasible vector $w'$ that redirects towards $c'$ some of the vote strength of voters in $N_{c'}$, in turn decreasing the support of the current committee members approved by these voters. Now, for a given threshold $t\geq 0$, we want to make sure not to reduce the support of any member $c$ below $t$, assuming it starts above $t$, and not to reduce it at all otherwise. A simple rule to ensure this is as follows: for each voter $n$ in $N_{c'}$ and each member $c\in A\cap C_n$, reduce the weight on edge $nc$ from $w_{nc}$ to $w_{nc}\cdot \min\{1, t/\supp_w(c)\}$, and assign the difference to edge $nc'$. That way, even if all edges incident to a member $c$ are so reduced in weight, the support of $c$ is scaled by a factor at least $\min\{1, t/\supp_w(c)\}$ and hence its new support does not fall below $t$.

Thus, if for each voter $n\in N$ and threshold $t\geq 0$ we define that voter's \emph{slack} as

\begin{align}
    \slack_{(A,w)}(n,t):= s_n - \sum_{c\in A\cap C_n} w_{nc} \cdot\min \Big\{ 1, t/\supp_w(c)\Big\} \label{eq:slack}
\end{align}
%
and for each $c'\in C\setminus A$ and $t\geq 0$ we define that candidate's \emph{pre-score} as
%
\begin{equation}\label{eq:prescore}
    \prescore_{(A,w)}(c',t) := \sum_{n\in N_{c'}} \slack_{(A,w)}(n,t),
\end{equation}
%
then we can add $c'$ to the solution with a support of $\prescore_{(A,w)}(c',t)$, while not making any other member's support decrease below threshold $t$. The resulting weight modification rule is formalized in Algorithm~\ref{alg:ins}. The next lemma easily follows from the previous exposition and its proof is skipped.

\begin{algorithm}[htb]\label{alg:ins}
\SetAlgoLined
\KwData{Partial feasible solution $(A,w)$, candidate to insert $c'\in C\setminus A$, threshold $t\geq 0$.}
Initialize the new weight vector $w'\leftarrow w$\;
\For{each voter $n\in N_{c'}$}{
Set $w'_{nc'} \leftarrow s_n$\;
\For{each current member $c\in A\cap C_n$}{
\lIf{$\supp_w(c)>t$}{update $w'_{nc} \leftarrow w'_{nc}\cdot\frac{t}{\supp_w(c)}$}
Update $w'_{nc'}\leftarrow w'_{nc'} - w'_{nc}$\;
}
}
\Return $(A+c',w')$\;
 \caption{$\ins(A,w,c',t)$}
\end{algorithm}

\begin{lemma}\label{lem:insert}
For a feasible partial solution $(A,w)$, candidate $c'\in C\setminus A$ and threshold $t\geq 0$, 
Algorithm $\ins(A,w,c',t)$ executes in time $O(|E|)$ and returns a feasible solution $(A+c',w')$ 
where $\supp_{w'}(c')=\prescore_{(A,w)}(c',t)$ and $\supp_{w'}(c)\geq \min\{\supp_w(c),t\}$ for each member $c\in A$. 
In particular, if $\prescore_{(A,w)}(c',t)\geq t$ then $\supp_{w'}(A+c')\geq \min\{\supp_w(A),t\}$.
\end{lemma}

Whenever partial solution $(A,w)$ is clear from context, we drop the subscript from our notation of slack and pre-score. 
When we add the new candidate $c'$ to the solution, we want to ensure that inequality $\prescore(c',t)\geq t$ holds, as we want to avoid increasing the number of validators with support below threshold $t$. 
Thus, for each unelected candidate $c'\in C\setminus A$ we define its \emph{score} to be the highest value of $t$ such that $\prescore(c',t) \geq t$ holds, i.e., 
%
\begin{align}
    \score_{(A,w)}(c'):=\max\{t\geq 0: \ \prescore_{(A,w)}(c',t)\geq t\},
\end{align}
%
where again we drop the subscript if $(A,w)$ is clear from context. Our heuristic now becomes apparent.

\begin{heuristic}[$\phragmms$]
Given a partial solution $(A,w)$, find a candidate $c_{\max}\in C\setminus A$ with highest score $t_{\max}=\max_{c'\in C\setminus A} \score(c')$, and execute $\ins(A,w,c_{\max},t_{\max})$ so that its output solution $(A+c_{\max},w')$ observes 

$$\forall c\in A, \ \supp_{w'}(c)\geq \min\{\supp_w(c), t_{\max}\}, \quad \text{ and } \quad \supp_{w'}(A+c_{\max})\geq \min \Big\{ \supp_w(A), t_{\max}\Big\}.$$
\end{heuristic}

In Appendix~\ref{s:algorithms} we describe efficient algorithms to find the candidate with highest pre-score for a given threshold $t$, as well as the candidate with overall highest score.

\begin{theorem}\label{thm:runtimes}
For a partial solution $(A,w)$ and threshold $t\geq 0$, there is an algorithm $\maxprescore(A,w,t)$ that executes in time $O(|E|)$ and returns a tuple $(c_t,p_t)$ such that $c_t\in C\setminus A$ and $p_t=\prescore(c_t,t)=\max_{c'\in C\setminus A} \prescore(c',t)$.
Furthermore, there is an algorithm $\maxscore(A,w)$ that runs in time $O(|E|\cdot \log k)$ and returns a tuple $(c_{\max}, t_{\max})$ such that $t_{\max}=\score(c_{\max})=\max_{c'\in C\setminus A} \score(c')$.
\end{theorem}

Our heuristic, which finds a candidate with highest score and adds it to the current partial solution (Algorithm $\maxscore$ followed by $\ins$) executes in time $O(|E|\cdot \log k)$. 
It thus matches up to a logarithmic term the running time of $\phragmen$ which is $O(|E|)$ per iteration. 
In Appendix~\ref{s:algorithms} we also draw parallels between the $\phragmen$ and $\phragmms$ heuristics, and explain how the latter can be seen as a natural complication of the former which always grants higher score values to candidates and thus inserts them with higher supports.

