\begin{abstract}

\emph{Polkadot} is a decentralized blockchain platform launched in 2020. It implements \emph{nominated proof-of-stake} (NPoS), a proof-of-stake-based mechanism where $k$ nodes are periodically selected by the network as \emph{validators} to participate in the consensus protocol, according to the preferences expressed by token holders who take the role of \emph{nominators}. 
This setup leads to an approval-based multiwinner election problem, where each nominator submits a list of trusted candidates, and has a vote strength proportional to their stake. 
A solution consists of a committee of $k$ elected validators, along with a fractional distribution of each nominator's vote among them. 
We consider two objectives for the election rule, both recently studied in the literature of social choice. The first one is ensuring the property of \emph{proportional justified representation} (PJR). The second objective, called \emph{maximin support}, is to maximize the minimum amount of nominators' vote assigned to any elected validator. 
We argue that the former objective aligns with the notion of decentralization in this governance process while the latter aligns with the security level of the system.

We present new results on each of these objectives as well as explore their relation to each other.
We prove that the maximin support objective is constant-factor approximable, as well as present a negative result showing that a constant-factor approximation is best possible. 
We then introduce a new heuristic whose output solution simultaneously a) guarantees a constant-factor approximation for maximin support and b) satisfies the PJR property. 
In fact, we prove that its output has a special structure which implies the two previous guarantees and which can be verified in linear time in the size of the input, even when the algorithm is privately run by an untrusted party who only communicates the result. 
Furthermore, our heuristic can be adapted into a post-computation which, when paired with any approximation algorithm for maximin support, returns a new solution that a) preserves the approximation guarantee and b) can be efficiently verified to satisfy the PJR property.
Our contribution on verifiable guarantees for election results can be of independent interest; in our application, it lets the blockchain platform dump the election algorithm to \emph{off-chain} workers, leaving only the solution verification (and ranking in case of several solutions) to be run \emph{on-chain}. 
Our results thus enable an efficient validator election protocol with formal and verifiable guarantees on decentralization and security. %hitherto unmatched by other variants of proof-of-stake. 
\end{abstract}
