\begin{abstract}

The property of proportional representation in approval-based committee elections has appeared in the social choice literature for over a century, and is typically understood as avoiding the underrepresentation of minorities. 
However, we argue that the security of some distributed systems is directly linked to the opposite goal of \emph{avoiding the overrepresentation} of any minority, a goal not previously formalized which leads us to an optimization objective known as \emph{maximin support}, closely related to the axiom of \emph{proportional justified representation} (PJR). 
We provide a new inapproximability result for this objective, and propose a new election rule inspired in Phragm\'{e}n's methods that achieves a) a constant-factor approximation guarantee for the objective, and b) the PJR property. Furthermore, a structural property allows one to quickly \emph{verify} that the winning committee satisfies the two aforementioned conditions, even if the algorithm was executed by an untrusted party who only communicates the output. Finally, we present an efficient post-computation that, when paired with any approximation algorithm for maximin support, returns a new solution that a) preserves the approximation guarantee, b) satisfies PJR, and c) can be efficiently verified to satisfy PJR.

Our work is motivated by an application on blockchains that implement \emph{nominated proof-of-stake} (NPoS), where the community must elect a committee of validators to participate in its consensus protocol, and where fighting overrepresentation protects the system against attacks by an adversarial minority. Our results enable a validator election protocol with formal and verifiable guarantees on security and proportionality. We propose a specific protocol that can be successfully implemented in spite of the stringent time constraints of a blockchain architecture, and that will be the basis for an implementation in the \emph{Polkadot} network, launched in 2020.

\end{abstract}