\section{A negative example for the sequential Phragm\'{e}n heuristic}\label{s:phragmen}

In this section we analyze the performance of $\phragmen$~\cite{brill2017phragmen} with respect to the maximin support objective. 

\begin{algorithm}[htb]
\SetAlgoLined
\KwData{Bipartite approval graph $G=(N\cup C, E)$, vector $s$ of vote strengths, target committee size $k$.}
Initialize $A=\emptyset$, $w=0\in \R^E$, $load(n)=0$ for each $n\in N$, and $load(c')=0$ for each $c'\in C$\;
\For{$i=1,2,\cdots k$}{
\lFor{each candidate $c'\in C\setminus A$}{
update $load(c') \leftarrow \frac{1+\sum_{n\in N_{c'}} s_n\cdot load(n)}{\sum_{n\in N_{c'}} s_n}$}
Find $c_{\min}\in \arg\min_{c'\in C\setminus A} load(c')$\;
Update $A\leftarrow A+c_{\min}$\;
\For{each voter $n\in N_{c_{\min}}$}{
Update $w_{nc_{\min}} \leftarrow load(c_{\min}) - load(n)$\;
Update $load(n)\leftarrow load(c_{\min})$\;
}
}
\emph{Scaling edge weights:} for each $nc\in E$ with $w_{nc}>0$, update $w_{nc}\leftarrow w_{nc}\cdot \frac{s_n}{load(n)}$\; 
\Return $(A,w)$\;
\caption{$\phragmen$, proposed in~\cite{brill2017phragmen}}
\label{alg:phragmen}
\end{algorithm}

The method proposed in~\cite{brill2017phragmen} only considers unit votes, and outputs only a committee. 
In Algorithm~\ref{alg:phragmen} we present a generalization that admits weighted votes in the input, and outputs a feasible edge weight vector along with the committee, for ease of comparison with other algorithms in this paper. 
We remark however that the numerical example we present below can be easily converted into another with unit votes -- where each voter $n$ is replaced with a cluster of voters with unit vote strength and where the cluster size is proportional to $s_n$ -- that would lead to the same negative result for the method in~\cite{brill2017phragmen}, even when the output committee is coupled with a balanced weight vector.

We exhibit now a family of instances where the approximation ratio offered by the output of Algorithm~\ref{alg:phragmen} equals the $k$-th harmonic number $H_k:=\sum_{i=1}^k 1/i = \Theta(\log k)$. This proves Lemma~\ref{lem:phragmen}.

\begin{figure}[htb]
\begin{center}
\scalebox{.8}{

\definecolor{rvwvcq}{rgb}{0.08235294117647059,0.396078431372549,0.7529411764705882}
\begin{tikzpicture}[line cap=round,line join=round,>=triangle 45,x=1cm,y=1cm]
\clip(-1,1.5) rectangle (7,7);
\draw [-,line width=1pt] (2,6) -- (6,6);
\draw [-,line width=1pt] (2,5) -- (6,5);
\draw [-,line width=1pt] (2,4) -- (6,5);
\draw [-,line width=1pt] (2,3) -- (6,5);
\draw [-,line width=1pt] (2,2) -- (6,5);
\draw [-,line width=1pt] (2,4) -- (6,4);
\draw [-,line width=1pt] (2,3) -- (6,4);
\draw [-,line width=1pt] (2,2) -- (6,4);
\draw [-,line width=1pt] (2,3) -- (6,3);
\draw [-,line width=1pt] (2,2) -- (6,3);
\draw [-,line width=1pt] (2,2) -- (6,2);
\draw (-0.5,6.3) node[anchor=north west] {$s_0=\sfrac{1}{(H_4 - \varepsilon)}$};
\draw (-0.5,5.3) node[anchor=north west] {$s_1=1$};
\draw (-0.5,4.3) node[anchor=north west] {$s_2=1$};
\draw (-0.5,3.3) node[anchor=north west] {$s_3=1$};
\draw (-0.5,2.3) node[anchor=north west] {$s_4=1$};
\begin{scriptsize}
\draw [fill=rvwvcq] (2,6) circle (2.5pt);
\draw[color=rvwvcq] (2,6.3) node {$n_0$};
\draw [fill=rvwvcq] (2,5) circle (2.5pt);
\draw[color=rvwvcq] (2,5.3) node {$n_1$};
\draw [fill=rvwvcq] (2,4) circle (2.5pt);
\draw[color=rvwvcq] (2,4.3) node {$n_2$};
\draw [fill=rvwvcq] (2,3) circle (2.5pt);
\draw[color=rvwvcq] (2,3.3) node {$n_3$};
\draw [fill=rvwvcq] (2,2) circle (2.5pt);
\draw[color=rvwvcq] (2,2.3) node {$n_4$};
\draw [fill=rvwvcq] (6,6) circle (2.5pt);
\draw[color=rvwvcq] (6,6.3) node {$c_0$};
\draw [fill=rvwvcq] (6,5) circle (2.5pt);
\draw[color=rvwvcq] (6,5.3) node {$c_1$};
\draw [fill=rvwvcq] (6,4) circle (2.5pt);
\draw[color=rvwvcq] (6,4.3) node {$c_2$};
\draw [fill=rvwvcq] (6,3) circle (2.5pt);
\draw[color=rvwvcq] (6,3.3) node {$c_3$};
\draw [fill=rvwvcq] (6,2) circle (2.5pt);
\draw[color=rvwvcq] (6,2.3) node {$c_4$};
\end{scriptsize}
\end{tikzpicture}
}
\end{center}
\caption{Instance of maximin support where $\phragmen$ gives an approximation ratio of $H_k - \eps$, for $k=4$.}
\label{fig:phragmen}
\end{figure}

\begin{proof}[Proof of Lemma~\ref{lem:phragmen}]
For an arbitrarily small constant $\eps>0$ and a committee size $k$, consider an instance where $N=\{n_0, \cdots, n_k\}$, $C=\{c_0, \cdots, c_k\}$, $s_{n_j}=1$ and $C_{n_j}=\{c_i: \ 1\leq i\leq j\}$ for each $1\leq j\leq k$, and $s_0=1/(H_k - \eps)$ with $C_{n_0}=\{c_0\}$; see Figure~\ref{fig:phragmen}. 
It is easy to see that if committee $\{c_1, \cdots, c_k\}$ is selected, each member can be given a support of value $1$. On the other hand, any committee that selects candidate $c_{0}$ can only provide it a support of value $s_0$. We will prove that Algorithm~\ref{alg:phragmen} selects committee $\{c_0, c_1, \cdots, c_{k-1}\}$, and thus its approximation ratio is $1/s_0 = H_k - \eps$. 

In the first round, we have that $load(c_0)=1/s_0$ and $load(c_j)=\frac{1}{k+1-j}$ for each $j\geq 1$, so we add $c_1$ to the committee, which has $load(c_1)=1/k = H_k - H_{k-1}$, and update $load(n_j)\leftarrow H_k - H_{k-1}$ for each $j\geq 1$. 
More generally, in the $i$-th round for $1\leq i\leq k-1$, we have that $load(c_0)=s_0$ and $load(c_j)=\frac{1+(k+1-j)(H_k - H_{k+1-i})}{k+1-j}=\frac{1}{k+1-j}+H_k-H_{k+1-i}$ for each $j\geq i$, so we add $c_i$ to the committee, which has $load(c_i)=H_k - H_{k-i}$, and update $load(n_j)\leftarrow H_k - H_{k_i}$ for each $j\geq i$. 
Finally, in the $k$-th round we have that $load(c_0)=1/s_0 = H_k - \eps$ and $load(c_k)=H_k$, thus we elect $c_0$ as the final committee member. This completes the proof.
\end{proof}