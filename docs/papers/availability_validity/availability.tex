
\section{Protocols}. 
\label{sec:protocol}

% Availability and Validity



The protocol consists of two phases ``Parachain phase'' and ``Relay Chain Phase''. The parachain phase is executed between collators and parachain validators. In the end of this phase, the parachain validators validate the block and provide its erasure code pieces to the validators. Then, the relay chain phase begins. If the parachain phase is executed correctly, then the relay chain phase includes extra validation of a parachain block, adding the block header to the relay chain and finalizing that relay chain block. Otherwise, unavailability protocol is run between validators. The details are as follows: 

\paragraph{Parachain Phase:} We first describe the protocol by which collators of a parachain $\rho$ submit a candidate block to the parachain validators assigned to $\vals_{\rho,t}$, at some time $t$.

\begin{itemize}
    \item For each parachain block $B$, a collator runs the $\prove(B,PC,M)$ and obtains the proof $\pi$ for the validity of  block $B$. Then, he sends the blob $\bb = (B,\pi,M)$ to the corresponding parachain validators to show that the block is valid and can be added to the relay chain. 
    
    \item Each parachain validator $V \in \vals_\rho$ checks the validity of the block by running the $\verify$ algorithm for the blob $\bb$. If it outputs 0, they ignore the block and store the block as invalid. Otherwise, $V$ runs the $\divi(\bb)$ to obtain the erasure code pieces $\mathcal{D} = \{d_1,d_2,...,d_n\}$ of $\bb$.
    % TODO Breaks V defimition badly
    \item $V$ constructs a Merkle tree with leaves $d_1,d_2,...,d_n$. Let us denote the root of the Merkle tree by $r$. At the end, $V$ signs for each validator $V_j$ the message $(d_j, r, \mprf_i, B_{info})$ and gives each signature $\sigma_i^j$ to the distributor parachain validator. Here, $\mprf_i$ is the proof that shows that $d_i$ is one of the leaves of the Merkle tree with the root $r$ and $B_{info}$ is identifying information of $B$. We note that the distributor can be selected randomly or in order. Each parachain validator also gives the erasure code piece to at least one (trusted) collator in order to backup the piece.
\end{itemize}


 


\paragraph{Relay Chain Phase:} This phase is to make sure that a valid and available block's header is in the relay chain. 

\begin{itemize}
    \item After receiving the signatures from parachain validators in the parachain phase, the distributor sends  for each validator $V_j$ all signatures  $\{\sigma_j^k\}_{PV_k \in \pv'} $ where $\pv' \subseteq \pv$ representing the set of parachain validators who sent a signature for $V_j$. 

    \item If $V_j$ receives at least $\kappa$ signatures for $B$, it adds it to the block list which keeps the data to be included to the relay chain. Later on, if this validator is the designated validator in BABE to produce the block, it creates block headers $\bh$ (Definition \ref{def:header}) for the blocks in the block list. $\bh$ including $\{\sigma_j^k\}_{PV_k \in \pv'} $.  These signatures are necessary to show which parachain validators approve for $B$. Then, he produces the relay chain block which includes the header. See \href{http://research.web3.foundation/en/latest/polkadot/BABE/Babe/}{BABE} for details.

    \item At the same time, each validator $V_j$ examines  each block in the relay chain if it includes unavailable block headers. If $V_j$ sees a block header $\bh$ in a block of the best relay chain whose erasure code piece $d_j$ is not received by $V_j$ (unavailable code piece), then he asks for it first from parachain validators. If a parachain validator sends valid $d_j, r, \mprf_j $, then the protocol ends. Otherwise, it asks for it from collators of the corresponding parachain. If $V_j$ receives a valid $d_j, r, \mprf_j $, the protocol ends. If $V_j$ does not receive any response in $\grace$ time after seeing $\bh$, he runs the unavailability protocol in Section \ref{sec:unavail}.

\end{itemize}

\subsection{Unavailability}
\label{sec:unavail}

We integrate our availability and secondary validity check protocols directly with GRANDPA, in that an honest node $V \in \vals$ cannot vote for relay chain blocks $B$ unless
\begin{itemize}
\item $V$ possesses their own piece for all the blobs referenced by $B$, and also
\item $V$ witnessed ... 
%TODO: discuss GRANDPA vs secondary checks
\end{itemize}

We run the availability and secondary validity check protocols only for parachain blocks included in relay chain blocks, not for all parachain blocks proposed by parachain validators.  In this way, we reduce the damage done by spammy parachains, at least beyond their own assigned parachain validators.

At the same time, we avoid complex anti-spam or prioritisation logic since only GRANDPA require this prior voting restriction.  In fact, if we later require prioritisation logic then this trick isolates it inside relay chain block production.

We cannot entirely escape the availability question within BABE however:  We have several forks $C_1,...,C_k$ for which at most $f$ validators possess all their chunks, but no fork for which $f+1$ validators possess all their chunks.  Yet, each block producers $V$ possess all their chunks for at least one fork $C_i$.  If BABE were oblivious to availability, then $V$ extends $C_i$, and GRANDA stalls under this configuration. 
% TODO:  Anything about secondary validity check here ???

Instead, we define an availability grace period $\grace$ after which an unavailability subprotocol alters BABE's chain selection rule:  

If a validator $V \in \vals$ cannot obtain some piece $x$ within $\grace$ time after seeing $x$ referenced by some relay chain block header $B_{head}$, then $V$ announces via gossip the unavailability of the piece $x$. 

If any validator $V' \in \vals$ observes unavailability announcements for pieces of some blob $y$ from at least $f+1$ different validators. then $V'$ shall not propose relay chain containing $y$ in BABE.
In this situation, there might be prevotes but never any precommits in GRANDPA for chains contain in $y$, so $y$ cannot possibly be finalised by GRANDPA.  We should consider if the GHOST chain weighting rule used by BABE and by GRANDPA for prevotes should weigh unavailability announcements, but doing so should only impact performance.

We considered $V'$ abandoning any fork $C$ for which at least $f+1$ different validators gossiped unavailability announcements for possibly distinct blobs in $C$, instead of for the same blob.  Any truly unavailable pieces eventually trigger both conditions but they trigger this variant with possibly distinct blobs first, and with more false positives.  We optimise for the honest case here and caution that more false positives results in more chains, so spammy parachains create could more load on the availability system.

Any validator should revoke their past unavailability announcements for some missing piece $x$ whenever they eventually obtain $x$, again by gossiping the revocation.  We also define some super availability grace period, longer than $|\pv| \grace$, after which time, if $2 f + 1$ of the validators announced unavailability of some specific piece, then the parachain validators who signed for that block are slashed.
We revert this slash whenever revocations later reduce the unavailability announcements below $f+1$.
% TODO: Any interactions with slashing computation?

...
As soon as receiving an unavailability announcement, the assigned validator sends the erasure code piece with the proof, if he has it. 
...
The details of how the assigned validator is selected and how he obtains the erasure codes are given in Section \ref{sec:validitycheck}. 
...



\subsection{Validity Check}
\label{sec:validitycheck}
With the unavailability protocol, we can detect the block headers whose pieces are missing. Beyond this, we also need to detect whether invalid block headers are in the relay chain blocks. For example, if all parachain validators are malicious, they can sign for an invalid block. We can catch this only by reconstructing the blob with the available pieces. In the validity check protocol, we describe how extra validators (different from parachain validator) are assigned to check the validity of blocks and how malicious activities are detected.


We first give some parameters before giving the details of the protocol.
The first parameter $\vcheck$ defines the total number of validator checks needed for a block header.

$$\vcheck = |\pv| + \mu +\lceil \runav + \rinv \rceil$$

Here, $\mu$ specifies the minimum number of checks, $\runav$ is the unavailability report factor from collators and $\rinv$ is the invalidity report factor from fishermen who claim invalidity for the relevant block. 


%Here, $\mu$ specifies the minimum number of checks, $\runav$ is the unavailability report factor which is the constant $c_a$ multiplied with the rate of the unavailability report in the last 1000 blocks.
%$\rinv$ is the invalidity report factor which is the constant $c_f$ multiplied with the relative stake of fishermen who claim invalidity for the relevant block. 




There are a three-factor validity check during the availability protocol:


\begin{enumerate}

      
    \item \textbf{Fisherman Check:} All fishermen collect blobs from collators in order to check the validity all the time. Whenever they discover an invalid block, they announce it to the validators by signing their claim. These claims are added to the block list by the validators. So, these claims are going to be in the relay chain. They do not provide any proof. If their claims are wrong then they are slashed. Otherwise, they are rewarded. The correctness of the fishermen claims are determined with the validity check protocol.
    
    \item \textbf{Parachain Validator Check:} All parachain validators check the validity of the block as described in the parachain phase of the availability protocol. If one parachain validator $PV_i$ sees that an invalid block header is in a block of the best relay chain, then he announces the invalidity. If a parachain validator has never seen the blob of the block header in the relay chain or he does not sign for that block, he runs the extra validity check protocol given below.
    
    \item \textbf{Non-Parachain Validator Check:} The extra validity checks are done by the non-parachain validators. Now, we explain how these validators are selected.
    
    \paragraph{Selection:} The selection process is private and no one knows the selected ones until they make it public. Assume that there exists $m$ parachains in Polkadot. Each validator $V_j$ checks first the following if required extra check is more than $n/m$ (i.e., $\mu + \lceil r_a + r_f \rceil \geq n/m$): 
    
    \begin{equation}\label{cond:mod}
        ID_{PC} = \mathtt{VRF}_{\sk^v_{j}}(r_{t}) \mod m    
    \end{equation}
     Here, $ID_{PC} \in \mathbb{N}$ is the identifier of a parachain $PC$ and $r_{t}$ is the VRF randomness of the block producer of slot $t$ in the relay chain (See \href{http://research.web3.foundation/en/latest/polkadot/BABE/Babe/}{BABE} for the details of $r_t$). For the current time $T$ where $T > \grace$, if  less than $\vcheck$ amount of checks are completed with condition (\ref{cond:mod}) or $\mu + \lceil r_a + r_f \rceil < n/m$ , each validator $V_j$ checks the following condition:
     \begin{equation}\label{cond:time}
        H(ID_{PC} || \mathtt{VRF}_{\sk^v_j}(r_t)) < \frac{\mu + \lceil r_a + r_f \rceil - \theta}{|\V|-|\pv|}+\tau  
    \end{equation}
    
      where $\tau = T- \grace$ and $H$ is a hash function that maps a value between 0 and 1. Here, if $\mu + \lceil r_a + r_f \rceil \geq n/m$, $\theta$ is the number of validators in condition (\ref{cond:mod}) for $ID_{PC}$. Otherwise, $\theta = 0$.
    
    However, if a block producer produces multiple blocks with the same VRF randomness $r_t$ (equivocation), then a validator $V_j$ is selected for the extra check if one of the  below two conditions together with (\ref{cond:mod}) and (\ref{cond:time}) are satisfied:
    
    
    \begin{equation}\label{cond:modequiv}
         ID_{PC} = \mathtt{VRF}_{\sk^v_{j}}(r_{t}||H_B) \mod m
    \end{equation}
    
    \begin{equation}\label{cond:equiv}
         H(ID_{PC} || \mathtt{VRF}_{\sk^v_j}(H_{B})) < \frac{\vcheck - |\pv|}{|\V|-|\pv|}   
    \end{equation}
    
     where $H_B$ is the hash of the relay chain block whose VRF randomness is $r_t$. Remark that we have greater number of checks for equivocation because it is a sign of malicious activity. Malicious validators can see who is in condition (\ref{cond:mod}) or (\ref{cond:time}) and if there is  no honest validator with the conditions (\ref{cond:mod}) or (\ref{cond:time}), they can equivocate which includes an invalid block. Therefore, we need different conditions than the condition (\ref{cond:mod}) and the condition (\ref{cond:time}).
    
    When $V_j$ satisfies the condition (\ref{cond:mod}), (\ref{cond:time}), (\ref{cond:modequiv}) or (\ref{cond:equiv}), $V_j$ has a right to check the validity. For this, it follows the extra validity check protocol.
    
    We note that if number of $\vcheck$ validators satisfy the condition (\ref{cond:mod}) or (\ref{cond:time} ) for a parachain with the identifier $ID_{PC}$ and no equivocation exist, then satisfying condition (\ref{cond:modequiv}) or (\ref{cond:equiv}) does not get rewarded.
    
\end{enumerate}


\paragraph{Extra Validity Check:}
An assigned validator $V_j$  asks for the blob for the extra validity check or the erasure code pieces in the following order until receiving all information to check the validity of the block header:
\begin{enumerate}
    \item Parachain validators who signed for it: asked for the blob
    \item Other parachain validators who did not sign for it: asked for the blob
    \item Validators: asked for the erasure code piece with the proof. Assuming that $V_j$ has a set $\mathcal{D}'$ of a correct erasure codes where $|\mathcal{D}'|<k$. He asks for at least number of $k-|\mathcal{D}'|$ pieces in $\mathcal{D}\setminus \mathcal{D}'$ from the validators.
    
    \item Collators: asked for the erasure codes or the blob. We note that $V_j$ can ask for the blob or the erasure code, if he is connected to the collators network.
    
\end{enumerate}

In the end,  if $V_j$'s attempt in option 1 and 2 above is unsuccessful and  obtains all missing pieces instead of blob $\bb$, he adds them to the set $\mathcal{D}'$ and reconstructs  $\bb = (B,\pi,M)$ with the algorithm $\recons(\mathcal{D}')$. 

In any case, either obtaining $\bb$ from $PV$s or the reconstruction algorithm, $V_j$ checks the validity of the block by  running  $\verify(\bb)\rightarrow b$.
\begin{itemize}
    \item if $b = 0$ (the blob is invalid), $V_j$ signs that the block is invalid. Whenever a validator receives a valid signature from another validator saying that the block is invalid, this validator obtains the blob  the same way that extra validators obtain and follow the same process.
    \item if $b = 1$ (the blob is valid), $V_j$ signs that the block is valid. $V_j$ also  runs the $\divi((B,\pi,M))$ to obtain the erasure codes $\mathcal{D} = \{d_1,d_2,...,d_n\}$ and constructs a Merkle tree  whose leaves are $d_1,d_2,...,d_n$. In the end, he gives away the missing piece of the validator with the Merkle tree proof whenever a validator announces unavailability. 
\end{itemize}


In the end of the validity protocol, if at least $f+1$ signatures are given by the validators saying that the block is invalid, then it is considered as invalid.
\subsection{Slashing and Rewarding}

\underline{\textbf{Fisherman}} 

\emph{Slashing:} If at least $\vcheck$ validators sign saying that the block is valid, then fishermen with the claim saying the block is invalid are slashed. All the claims from these fishermen are ignored in future.

\emph{Rewarding:} If $f+1$ validators sign to say the block is invalid, the  fishermen with the claim saying that the block is invalid are rewarded.

%TODO MORE DETAILS

\noindent\underline{\textbf{Validators}}

\emph{Slashing:} If at least $f+1$ validators sign for the invalidity of a block and less than $f+1$ validators sign for the validity of the block, then we slash all validators who signed for the validity.

If at least $f+1$ validators sign for the validity of a block and less than $f+1$ validator sign for invalidity, we slash the ones who signs for the invalidity.

We note that $f+1$ validity signatures and $f+1$ invalidity signatures are not possible given that at most $f$ validators are malicious and proof of block algorithms  are correct (See Definition \ref{def:pob}).

\emph{Rewarding:} All parachain validators and extra validators who signed for the final decision (either valid or invalid) of a block receive a reward. 

Eligibility of the reward can be easily checked for the parachain validators and extra validators who satisfies the conditions when $\tau = 0$. Latter, we just need to verify the VRF proof.
However, it is not easy to verify the condition for $\tau > 0$, because this value is subjective. Therefore, we may not need to be very precise for this $\tau$ value, if the signature of the validator is correct.
