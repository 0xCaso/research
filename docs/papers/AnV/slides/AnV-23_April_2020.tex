

\documentclass{beamer}
% \documentclass{article}

% \usefonttheme[onlymath]{serif}

\usepackage{amsmath}
\usepackage{amsfonts}
\usepackage{amssymb}

\usepackage{graphicx} % http://ctan.org/pkg/graphicx
\usepackage{booktabs} % http://ctan.org/pkg/booktabs
\usepackage{xparse}   % http://ctan.org/pkg/xparse
% Rotation: \rot[<angle>][<width>]{<stuff>}
\NewDocumentCommand{\rot}{O{45} O{1em} m}{\makebox[#2][l]{\rotatebox{#1}{#3}}}%
% \renewcommand{\arraystretch}{1.2} % (or 1.3)

\usepackage{fancyvrb}
\usepackage{stackengine}

\usepackage{tikz}
\usetikzlibrary{arrows}
% \usetikzlibrary{matrix,positioning}

\tikzstyle{every picture}+=[remember picture]

% Define block styles
\tikzstyle{module} = [rounded rectangle, fill=blue!10, node distance=3cm, minimum height=2em]


\begin{document}


\begin{frame}[t]

{\bf Sharding:}  Convince $n$ parties some state evolved correctly, while revealing each update to only $k<n$ checking parties.

\bigskip\bigskip
These $k$ checkers are trusted authorities in more centralised designs, but..

\medskip

% As power corrupts
We keep ``authority'' more narrowly prescribed, ala slsahable stake,
by treat everyone relatively adversarially. 

\pause
\bigskip\bigskip

{\bf Naive Answer:}   
Select the $k$ checkers randomly, but make $k$ large
\hspace*{3pt} ByzCoin ($k=300$ or $500$?), ETH2 ($k=100$?)

\medskip

Those $k$ checkers know themselves in advance, which increases their authority.
It still works for $k$ large enough but..

\end{frame}


\begin{frame}[t]

{\bf Sharding:}  Convince $n$ parties some state evolved correctly, while revealing each update to only $k<n$ checking parties.

\bigskip\bigskip

{\bf Shared Randomness Problem:}  \\
Can anyone manipulate who becomes a checker?

\bigskip\bigskip

{\bf Adaptivity Problem:}  \\
Can anyone interfear with the checkers operation?

\bigskip
We cannot observe when the network behaves adversarially, so one critical case is:

{\bf Availability Problem:}  \\
Can honest checkers obtain the update/block?  

\pause
\bigskip\bigskip

Could we avoid the $k$ checkers even knowing themelves in advance? 

\end{frame}


\begin{frame}

{\bf Problem:}  \\
How would you convince $n$ people that you would give them data without actually sending them the data?

\bigskip

{\bf Answer:}  \\
Reed-Solomon erasure codes represent your data by the coefficients of a polynomial $p(x)$ of degree $d$ over some finite field and produces code symbols by evaluating $p(0),p(1),\ldots,p(d n / f)$.  You give symbols $p(i d/f),\ldots,p((i+1) d/f - 1)$ to the $i$th person. 

Any $f$ people could now recompute $p(x)$, and hence your data, by doing Lagrange interpolation with their combined pieces, so they can merely vote.

\end{frame}


% 0. Collation 
% 1. Backing
% 2. Availability
% 3. Approval
% 4. Finality 
%
%
% 1. Role assignment
% 2. Transport & Topology
% 3. Attestation


\begin{frame}[shrink=15]

\vspace*{1cm}
% ===== DEFAULT ROTATION PARAMETERS
\begin{tabular}{ccrccc}
 \tikz\node (C) { \stackanchor{Collation}{\vphantom{\Big(} (Cumulus)} }; & &
 %   $\overset{\textrm{\normalsize }}{\scriptscriptstyle \vdots}$
    & \rot{Role assignment}
    & \rot{Transport \& Topology}
    & \rot{Attestation} \\
    \cmidrule{4-6}
 \multicolumn{1}{l}{ \tikz\node (CB) {\small \textcolor{brown}{\texttt{CandidateBacked}}}; }
 & & \tikz\node (B) {Backing};      &\tikz\node (epoch) {epoch};        &  \tikz\node {C$\to$V};  &  \tikz\node {Check};  \\
 \multicolumn{1}{l}{ \tikz\node (CA) {\small \textcolor{brown}{\texttt{CandidateAvailable}}}; }
 & & \tikz\node (A) {Availability}; &\tikz\node {*};            &  \tikz\node {$\mu$TP};  &  \tikz\node {RS};  \\
 & & \tikz\node (V) {Approval};     &\tikz\node (VRF) {VRF$(\cdot)$}; &  \tikz\node {$\mu$TP}; &  \tikz\node {Check};  \\

    \cmidrule{4-6} 
 & \\
 \tikz\node (F) { \stackanchor{Finality}{(GRANDPA)} }; \\
 & & \multicolumn{2}{c}{ \tikz\node (BP) { \stackanchor{ Block production }{ (BABE/Sassafras) } }; }  \\
 %    $\overset{\scriptscriptstyle \vdots}{\textrm{\normalsize }}$ & & & \\ 
     % & \multicolumn{3}{c}{--- GRANDPA ---} \\
\begin{tikzpicture}[overlay] % remember picture
        \path[->,blue] (C) edge [bend left] node [above] {$B,\pi$} (B);
        \path[->,green,dashed,thick] (B) edge [bend right=20]  (CB);
        \path[->,green,dashed,thick] (CB) edge [bend left=5] (A);
        \path[-,white,opacity=0.8] (B) edge node {\textcolor{blue}{$\swarrow \downarrow \searrow$}} (A); 
        \path[-,white,opacity=0.8] (A) edge node {\textcolor{blue}{$\searrow \downarrow \swarrow$}} (V); 
        \path[->,green,dashed,thick] (A) edge [bend left=5] (CA);
        \path[->,green,dashed,thick] (CA) edge [bend right=15] (V);
        \path[->,green,dashed,thick] (V) edge [bend left] (F);

        \path[->,red,dashed,thick] (BP) edge [bend left=15] (epoch);
        \path[->,red,dashed,thick] (BP) edge [bend right] (VRF);

\end{tikzpicture}
\end{tabular}

\end{frame}

%\begin{tabular}{rccc}
%     & \rot[50]{Role assignment}
%     & \rot[50]{Transport \& Topology}
%     & \rot[50]{Attestation} \\
%     \cmidrule{2-4}
%  {Backing}      & {epoch}       & {C$\to$V}  & {Check}  \\
%  {Availability} & {*}            & {$\mu$TP}  & {Piece}  \\
%  {Approval}     & {VRF$(\cdot)$} & {$\mu$TP} & {Check}  \\
%     \cmidrule{2-4}
%\end{tabular}

\begin{frame}[t,fragile]{Backing Checks}

Roughly like ByzCoin or ETH2, except small groups.

\bigskip

{\bf Group assignment:} \\ \smallskip
% Use {\tt rand\_chacha} plus {\tt rand::seq::SliceRandom::shuffle}
% \vspace{-3pt} % \vspace{-10pt}
\begin{Verbatim}[commandchars=\\\{\}]
use rand::\{SeedableRng, seq::SliceRandom\};
let seed = \dots \textcolor{red}{epoch.randomness} \dots;
let mut indexes = (0..num_validators).collect::<Vec<_>>();
indexes.shuffle(rand_chacha::ChaChaRng::from_seed(seed));
indexes.chunks(num_backing_per_parachain)
\end{Verbatim}

\bigskip

{\bf Transport \& Topology:} \\ \smallskip
Collator selects relay chain parent, makes the PoV block $B$ by attaching witness data, and sends it to assigned validators.
% validators assigned to that parachain or parathread slot.  

\bigskip


\end{frame}


\begin{frame}[t]{Backing Checks}

% {\em Secret Reed-Solomon Sause:}
Assume $3f+1$ validators.  Reed-Solomon erasure code the PoV block $B$ into $3f+1$ pieces aka ``symbols'' so $f+1$ pieces suffices for reconstruction.  Compute the Merklee root $m_B$ of the pieces and make authenticated pieces with attached witness for $m_B$.

\bigskip

{\bf Attestation:} \\ \smallskip
Assigned validators check.  If okay, they create and sign the candidate receipt, which contains $m_B$, and gossip everything among themselves.  \\ \smallskip
If enough backers, they gossip {\em only} the candidate receipt to all validators.

\bigskip

We'd love the relay chain to avoid working on conflicting candidates, so any further work depends on the attested candidate receipt being included in a \texttt{CandidateBacked} transaction.

\end{frame}


\begin{frame}[t]{Availability}

{\bf Assignment:}  \\ \smallskip
Send for blocks you backed.  Receive for everything else.

\bigskip

{\bf Transport \& Topology:} \\ \smallskip
We've one unique erasure coded piece for each validator, so send the $i$th piece to the $i$th validator, assuming a \texttt{CandidateBacked}.  \\ \smallskip

% Actually doing this requires reinventing
\hspace*{10pt} Reinvent BitTorrent but sans tracker!  

\bigskip

{\bf Attestation:}  \\ \smallskip
Attest on-chain with \texttt{HaveCandidatePiece} for pieces you receive:  Sign $(H(R),X)$ where $X$ is a bitfield over \texttt{CandidateBacked} in $R$. 

\bigskip

We declare \texttt{CandidateAvailable} once $2f+1$ validators say so. \\
In fact, parachain candidates ``run'' in the block $R$ where they reach this goal, so they receive and send XCMP messages here.

\end{frame}


\begin{frame}[t]{Approval}

{\bf Assignment:}  \\ \smallskip
All validators assign themselves with \textcolor{violet}{\em VRFs conditions}, which
only they know until they announce themselves.
% necessarily announced before checking.

\bigskip

{\bf Transport \& Topology:} \\ \smallskip
If assigned, you fetch erasure coded pieces of $f$ish other validators. \\ \smallskip

\hspace*{5pt} Increases our BitTorrent resemblance, but still without tracker.

\bigskip

{\bf Attestation:}  \\ \smallskip
If block passes then gossip attestation for inclusion on on-chain.

Grandpa only votes for forks with \textcolor{violet}{\em enough} attestations in children.

\bigskip\bigskip

\textcolor{red}{\bf Rule:}  If validators disagree on validity then all validators check it.


\end{frame}


\begin{frame}[t]{Approval assignment VRF conditions and sufficiency}

Consider the \texttt{CandidateAvailable} $\mathcal{B} = \{ B_1,\ldots,B_n \}$ announced in a relay chain block $R$.  \\ \smallskip

We gossip VRF signatures $\pi := \mathsf{VRF}_{V.\mathsf{sk}}(\mathtt{input})$ with three different inputs types that each determine an assignment $\mathcal{V} \times \mathcal{B} \to \mathbb{N}$ of a delay for which a validator $V$ should wait before checking $B_i$. 
\vspace{-2pt}
\begin{align*} 
\mathtt{relay\_input}  &:= R.\mathtt{vrf\_inout}.\mathtt{make\_bytes}() \\
\mathtt{para\_input}_i &:= \mathtt{input\_seed} +\!\!\!\!+\, B_i.\mathtt{paraid} \\
\mathtt{equiv\_input}_i &:= H(B_i) 
\end{align*}
Although $\mathtt{equiv\_input}_i$ only become valid if $R$ has an equivocation $R'$ with $R.\mathtt{vrf\_inout} = R'.\mathtt{vrf\_inout}$

$V$ has delay 0 for $B_i$ if $i = \pi.\mathtt{make\_bytes}() \bmod n$, but the other two sample the delay uniformly using $\pi.\mathtt{make\_bytes}() \bmod \mathtt{max\_delay}$.

Avoid gossiping these before its delay makes an assignment valid.


\end{frame}


\end{document}
\endinput


\bigskip\bigskip



 $\pi$ associate a delay $\pi.\mathtt{delay}$ with a validator $V$

Any  associates a delay $t_$ $B_i$ 

At any time $t$ after $R$, 

We track these values associated to each
Associated to each $B_i$ 
\begin{itemize}
\item target number of checkers $\ell_{B_i}$,
\item announced checker assignments $A_{B_i,t}$,
\item attestations $S_{B_i}$ 
\end{itemize}

assignment seed



% All validators track the set $A_z$ of valid announced checker claims, as well as the subset $S_z$ of all validity claims $S_B$ for $B$ generated by nodes in $A_z$.  We identify here the generic one-shot protocol run by $z$ because multiple $z$ may imply the same $B$, except we emphasise that messages exist only for $S_B$ not $S_z$, and that our protocol run ends by considering $S_z$ sufficient.  

% We uniformly sample time delays from the interval $[0,\netdelay {n' \over \ell}]$, so that an expected $\ell$ land in $[0,\netdelay]$.  We use verifiable sampling here of course, meaning we compute this delay $d_{V,z}$ from $\tilde{\omega}_{V,z}$.  







\begin{frame}[t]{}

\textcolor{red}{\bf Anti-DoS Rule:}  Anyone who announces their assignment by some VRF condition must attest or be replaced by someone unforeseeable.

\end{frame}








\begin{frame}{}
.
\end{frame}




% $\mathtt{let\ } 
\hspace*{5pt}  
\hspace*{5pt}  $\mathtt{check\_paraid} := \mathsf{VRF}_{\mathsf{sk}}(\mathtt{seed}) \,\bmod\, \mathtt{num\_parachains}$




\begin{Verbatim}[commandchars=\\\{\}]

VRF

use rand::{SeedableRng, seq::SliceRandom};
let seed = .. \textcolor{red}{epoch.randomness} ..;
let mut indexes = (0..num_validators).collect::<Vec<_>>();
indexes.shuffle(rand_chacha::ChaChaRng::from_seed(seed));
indexes.chunks(num_backing_per_parachain)
\end{Verbatim}











\documentclass{beamer}

\usepackage[utf8]{inputenc}

\usepackage{amsmath}
\usepackage{graphicx} % http://ctan.org/pkg/graphicx
\usepackage{stackengine}

\usepackage{booktabs} % http://ctan.org/pkg/booktabs
\usepackage{xparse}   % http://ctan.org/pkg/xparse
% Rotation: \rot[<angle>][<width>]{<stuff>}
\NewDocumentCommand{\rot}{O{45} O{1em} m}{\makebox[#2][l]{\rotatebox{#1}{#3}}}%
% \renewcommand{\arraystretch}{1.2} % (or 1.3)


\usepackage{tikz}
\usetikzlibrary{shapes,arrows,matrix}

% \tikzstyle[every matrix/.style]+={ampersand replacement=\&,column sep=2cm,row sep=2cm}

% Define block styles
\tikzstyle{module} = [rounded rectangle, fill=blue!10, node distance=3cm, minimum height=2em]

\begin{document}

% 0. Collation 
% 1. Backing
% 2. Availability
% 3. Approval
% 4. Finality 
%
%
% 1. Role assignment
% 2. Transport & Topology
% 3. Attestation

\begin{frame}[fragile,shrink=15]
\vspace*{1cm}
\begin{tikzpicture}[auto,node distance=7cm] % row sep=3em,column sep=4em,minimum width=2em
    % node distance = 1cm
     % every matrix/.style={column sep=2cm,row sep=2cm}]
 \node [module] (C) { \stackanchor{Collation}{(Cumulus)} }; 
 \node [module,below of=C] (BP) { \stackanchor{Block production\vphantom{)}}{(BABE/Sassafras)} };
 \node [module,below of=BP] (F) { \stackanchor{Finality}{(GRANDPA)} };
 
 \node [right of=BP] {
\begin{tabular}{rccc}
     & \rot[50]{Role assignment}
     & \rot[50]{Transport \& Topology}
     & \rot[50]{Attestation} \\
     \cmidrule{2-4}
   \tikz\node (B) {Backing};      &\tikz\node {epoch};        &  \tikz\node {C$\to$V};  &  \tikz\node {Check};  \\
   \tikz\node (A) {Availability}; &\tikz\node {*};            &  \tikz\node {$\mu$TP};  &  \tikz\node {Piece};  \\
   \tikz\node (V) {Approval};     &\tikz\node {VRF$(\cdot)$}; &  \tikz\node {$\mu$TP}; &  \tikz\node {Check};  \\
     \cmidrule{2-4}
   \vphantom{\rot[50]{Transport \& Topology}} & & & \\
\end{tabular}
 };

 % \path[->] (C) edge [bend left] (B);
 % \path[->] (B) edge [bend right] (BP);
     %   \path[->] (BP) edge [out=0, in=-90] (A);

\end{tikzpicture}

\end{frame}

\end{document}




\endinput







\makeatletter
\newcommand{\superimpose}[2]{%
  {\ooalign{$#1\@firstoftwo#2$\cr\hfil$#1\@secondoftwo#2$\hfil\cr}}}
\makeatother

\newcommand{\veewedge}{\mathpalette\superimpose{{\swarrow}{\searrow}}}
