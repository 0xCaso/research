
\section{Protocol} % Availability and Validity
\label{sec:protocol}

We now describe and instantiate our {\em availability and validity protocol} that provides efficient sharding.  It consists of 
\begin{itemize}
\item a parachain phase that prepares the candidate block and performs preliminary validity checks,
\item a relay chain submission phase that distributes candidate parachain blocks and produces relay chain blocks,
\item availability and unavailability subprotocols that enforce availability in GRANDPA and BABE respectively,
\item fuller secondary validity checks for GRANDPA,
\item objection procedures for fishermen, and
\item invokation of a Byzantine fault tolerant ``finality gadget'' that gives us finality.
\end{itemize}

We do require that secondary checks complete before finality.  We cannot however require that all validator checks conclude before finality, or even ask fishermen to begin checks before finality, so invalidity can be detected after finality.

% TODO: Anything more worth saying here?  Maybe extract from:  The parachain phase is executed between collators and parachain validators. In the end of this phase, the parachain validators validate the block and provide its erasure code pieces to the validators. Then, the relay chain phase begins. If the parachain phase is executed correctly, then the relay chain phase includes extra validation of a parachain block, adding the block header to the relay chain and finalizing that relay chain block. Otherwise, unavailability protocol is run between validators. The details are as follows: 


\subsection{Parachain phase} 
\label{sec:parachain}

We first describe the protocol by which collators of a parachain $\para$ submit a candidate block to the parachain validators assigned to $\vals_\para$.

\smallskip
% \paragraph{Collator subphase:} 

Initially, a {\bf collator} $C$ of a parachain $\para$ must propose some candidate block $B$ for $\para$.  We let $B'$ denotes the parent of $B$.  As above, let $\rin$ and $\rout$ denote state root before and after executing $B$.  

In practice, we want shared security so that parachains can communicate, so $B$ should reference some relay chain block(s) $R^0_B$ that distinguish any state $\rho$ maintains on the relay chain, such as the incoming messages accumulated for $\rho$ to be processed by $B$.  We let $q$ denote the Merkle root of this state $\rho$ maintains on the relay chain in $R^0_B$.
 
First, $C$ constructs the witness data $\pi$ by evaluating the block with $\prove_{\hat{\para}}(B,M)$, so they can build the {\em candidate proof-of-validity blob} $\blobB = (B,\pi,M)$, and also obtain the block metadata $(H(B'),\rin,\rout,H(R^0_B),q,\ldots)$. 

As $\prove$ is a randomized algorithm, $C$ must next reevaluate the block with $\verify_{\hat{\para}}(\blobB)$.  We shall assume verification succeeds, but if this verification fails then $C$ reports invalid parachain code for $\para$, and discards $B$ or possibly shuts down.  Assuming no errors, $C$ sends the candidate blob $\blobB$ to the corresponding parachain validators $\vals_\para$, along with any block metadata $(H(B'),\rin,\rout,H(R^0_B),q,\ldots)$. 

\smallskip
% \paragraph{Parachain validator:}

We shall expect the parachain validator $V$ that processes candidate blob $\blobB$ to update $\blobB$'s associated metadata to reference the most recent relay chain block $R^1_B > R^0_B$, so long as $R^1_B$ still contains the same state root $q$ for $\rho$,
i.e.\ $(H(B'),\rin,\rout,H(R^1_B),$\vect{R^1_B q}$,\ldots)$.

Next each {\bf parachain validator} $V \in \vals_\para$ checks the validity of the block by evaluating the block with $\verify_\para(\blobB)$.  We refer to these as the {\em preliminary validity checks} as well.  If verification succeeds, then $V$ gossips $\blobB$ among the parachain validators $\vals_\para$ after checking $\blobB$ itself, and we continue below.

If however verification fails, the parachain validator rejects the candidate $\blobB$ and report it as invalid.  We abandon $B$ if no validators sign it, but invalidity claims cannot necessarily result in penalties for either $\para$ or $C$.  

At any time, if any two validators disagree about a parachain block's validity then all validators shall check the block.  In this case, we accumulate votes until $f+1$ claim validity or invalidity, and then slash the loosing side.  We cannot slash if neither side reaches $f+1$, but we still declare the block invalid in that case.  We expect governance to identify software faults and manually revert slashes they cause, but governance can also manually institute slashes in this second case, or manually slash $\para$ for offenses like malicious code or improper non-determinism. 


\subsection{Relay chain authorship} % Relay chain phase I: Block production 

% \smallskip
% \paragraph{Parachain validator:}

Now any {\bf parachain validator} $V \in \vals_\para$ runs $\encode_{f+1,\nvals}(\blobB)$ to obtain the {\em prepieces} list $\prepieces_B$ of $\nvals$ erasure code symbols aka pieces of $\blobB$.  Next $V$ computes a Merkle root $\merkleroot_B$ for the Merkle tree with leaves $\prepieces_B$.  $V$ constructs the signed candidate receipt $\reciept_{B,\{V\}} := (\reciept_{B},\{V\})$ for $B$ by signing an inner candidate receipt $\reciept_B = (\para.\mathsf{id},H(\reciept_{B'}),\merkleroot_B,H(B),\rin,\rout)$ where $B'$ denotes the parent of $B$, and attaching its signature\handan{}{ and where id is ...}.  
% TODO: Improve explicit state root commitments maybe?
% Also Joe asks if we should talk about parachain block headers
% Anything else?

We gossip these candidate receipts $\reciept_{B,S}$ among the parachain validators $\vals_\para$.  In doing so, we improve them by further aggregating the signature set $S$.\footnote{We envision $\vals_\para$ being small enough that BLS signatures do not improve verification time over Schnorr signatures, although BLS might reduce the candidate receipt's signature from 640 bytes down to 50 bytes.}  We publish $\reciept_{B,S}$ for relay chain block producers using relay chain gossip (mempool) whenever $S \ge \kappa_\para \npvals$, assuming this happens eventually.  We think $\kappa_\para = {1\over2}$ gives a reasonable choice, but our security analysis below shows that $\kappa_\para \npvals = 1$ suffices.  We archive $B$ if another conflicting blocks gets finalised by GRANDPA, but maybe eventually delete it.  We archive but probably do not delete $B$ if GRANDPA is stalled but the fork choice rule clearly favours other forks.  

\smallskip
% \paragraph{Relay chain block producer:}

Any upcoming {\bf relay chain block producer} $U \in \vals$ enqueues any candidate receipts $\reciept_{B,S}$ received for possible inclusion in some future relay chain block that $U$ creates (\handan{Definition \ref{def:header}}{not exists}), which we denote $R$.  We of course need $R$ to have an ancestor $R'$ that includes the candidate receipts $\reciept_{B'}$ for the parent parachain block $B'$ of our candidate $\reciept_{B,S}$, and that no $R''$ between $R$ and $R'$ include any block from $\para$.  Ideally $U$ continues aggregating the signatures $S$ on $\reciept_{B,S}$ while waiting its turn too, but $R$ must satisfy $|S| \ge \kappa_\para \npvals$.  See \href{http://research.web3.foundation/en/latest/polkadot/BABE/Babe/}{BABE} for more details on block production.
% TODO: Any specific comments on relay chain block headers $\bh$

Associated to each prepiece $d \in \prepieces_B$, we define a {\em candidate piece} $(d,\merkleroot_B,\vect{\merkleroot_B d})$ by attaching the Merkle root $\merkleroot_B$ and an inclusion proof $\vect{rd}$, which authenticates $d$ as being committed to by the Merkle root $\merkleroot_B$.  Of course this expansion commutes with expanding the signer set $S$ on the candidate receipt $\reciept_{B,S}$.

We handle only the list $\pieces_B$ of these authenticated pieces for the remainder of the protocol. 
$$ \pieces_B = Listst{ (d,\reciept_{B,S},\vect{\merkleroot_B d}) }{ d \in \prepieces_B } $$

We must distribute $\pieces_B$ among the full relay chain validator set $\vals$ with $\pieces_B[i]$ going to $\vals[i]$ for $i = 1,\ldots,\nvals$.  In so doing. we force the signer set $S$ into making $\blobB$ available for testing by random secondary checkers without revealing those secondary checkers.  This trick provides the core scalability advantage of Polkadot.

We might however see many competing parachain candidate blocks at this point, so we delay this distribution process until some relay chain block $R$ contains $\reciept_{B,S}$.  We assume such an $R$ containing $\reciept_{B,S}$ exists throughout the remainder of this section.


\subsection{Topology}
%TODO: "Piece distribution topology" is too long

We find that scalability actually depends heavily upon the topology and routing used to distribute data, but that specifics depend upon the scale.  We briefly explain the specialised topology and routing requirements for our two phases of parachain data distribution.  We caution however that routing almost always requires some capacity for multi-hop forwarding because otherwise risks excluding some elected validators.  

\smallskip
\paragraph{Candidate blocks:}

We need our network topology to permit one parachain $\para$ to distribute the erasure coded candidate pieces $\pieces_B$s relatively quickly, which amounts to a graph expansion property.  We want a reasonable connectivity property too because too many parachain validators going down requires expensive reconstructions by any unreachable parachain validators (see \handan{\S\ref{sec:reconstructions}}{not defined} below).  We might also ask good collators to send the same block to several well chosen parachain validators as well.  As an example, if our parachain validators $\vals_\para$ form a cycle then $B$ reaches all parachain validators in two hops if the collators send $B$ to one third of $\vals_\para$, but adding well chosen chords reduces our connections with collators and improves our connectivity.  

\smallskip
\paragraph{Candidate pieces:}

All parachain validators in $\vals_\para$ must compute all of $\prepieces_B$ to compute $\merkleroot_B$, from which computing $\pieces_B$ too costs nothing.  We should therefore divide the distribution burden as equally as possible among parachain validators in $\vals_\para$.  We also prefer if the topology is symmetric in the sense that the links over which $\para_1$ sends to validators in $\para_2$ are the same as the links over which $\para_2$ sends to validators in $\para_1$.  We now outline an extremely simple topology that satisfies these requirement:

We recall any parachain $\para_i$ is equipped with a somewhat ephemeral value $\para_i.\mathsf{seed}$ that depends upon its parachain validator assignment and some on-chain randomness $r$.
$$ \para_i.\mathsf{seed} := H\left( r, \mathsf{sort} \setst{ V.\mathsf{pk} }{ V \in \vals_{\para_i} } \right) $$

We begin by assuming some abstract symmetric topology $\mathcal{T}$ on the set parachains, preferably a complete graph, i.e.\ diameter one.  Consider two parachains $\para_1$ and $\para_2$ with disjoint validator sets that are connected by an edge in $\mathcal{T}$.  We let $\mathsf{parashuffle}(\para_i,\para_{2-i})$ denote the Fisher-Yates shuffle of $\vals_{\para_i}$ seeded by $H( \para_i.\mathsf{seed}, \para_{2-i}.\mathsf{seed} )$.  We now define a topology $\mathcal{T}_e$ on the $\vals$ by connecting $\mathsf{parashuffle}(\para_1,\para_2)[j]$ to $\mathsf{parashuffle}(\para_2,\para_1)[j]$ for $j = 1,\ldots,\npvals$.

We can adapt this scheme to varying $|\vals_{\para_i}| \ge \npvals$ quite easily if not all parachains have the same number of assigned validators, i.e.\ if $\npvals$ is not tight.  We can also do additional shuffles if more than one link is desired.  
% 
If assigned two nodes cannot connect to one another, then any still online attempt connections with some random other nodes from the other parachain, or perhaps use some smarter scheme. 

As an example, assume $\mathcal{T}$ is a complete graph:  After our parachain validator $V$ of $\para$ observes some relay chain block $R$ containing $\reciept_{B,S}$ then, for all other parachains $\para' \ne \para$, $V$ computes the $i_{\para'}$s such that $V = \mathsf{parashuffle}(\para,\para')[j]$ and $\vals[i_{\para'}] = \mathsf{parashuffle}(\para',\para)[j]$ for some $j \leq \npvals$, and $V$ send $\pieces_B[i_{\para'}]$ to $\vals[i_{\para'}]$ directly using QUIC.  We expect $\vals[i_{\para'}]$ might have some piece from $\para'$ for $V$ too, thanks to the symmetry of our $\mathsf{parashuffle}$ criteria.  We expect this symmetry to reduce the required connections by almost a factor of two.

We have described this as $V$ initiating the connection, but a similar procedure works for $V$ requesting its piece for some $\para'$ block, or symmetrically $\vals[i_{\para'}]$ requesting $\pieces_B[i_{\para'}]$.  In fact, an initial implementation should focus upon requests because as noted above we shall request from other validators when our first choice fails. 

If $\vals[i_{\para'}]$ cannot reach $V$ then $\vals[i_{\para'}]$ must select some backup node to replace $V$.  Assuming $\mathcal{T}$ is complete, we should distribute these evenly among $\vals_\para \setminus \{V\}$, so as one option $\vals[i_{\para'}]$ could perform a Fisher-Yates shuffle of $\vals_\para \setminus \{V\}$, seeded by its own identity $\vals[i_{\para'}].\mathsf{pk}$ and $\para.\mathsf{seed}$, and then contact those remaining parachain validators in the resulting order.  We caution this option breaks the topology's symmetry, so as noted above nodes might exploit whatever links work first, and only take guidance from this shuffle when creating new links.  If $\mathcal{T}$ is not complete then alternative approaches that choose another parachain work too.  

If $\mathcal{T}$ is not complete then intermediate nodes must forward pieces for other nodes.  In fact, the diameter of $\mathcal{T}$ equals the maximum number of hops required for $\mathcal{T}_e$ to distribute each piece.  
In practice, these edges in $\mathcal{T}_e$ ro be the two nodes
% $\mathsf{parashuffle}(\para_1,\para_2)[j]$ to $\mathsf{parashuffle}(\para_2,\para_1)[j]$
maintaining a UDP protocol like QUIC connection because UDP should permit higher valency than TCP and hence permit a lower diameter $\mathcal{T}$.  
We should evaluate other topologies besides $\mathcal{T}_e$ before going beyond complete $\mathcal{T}$ proves necessary, but our symmetry property provided by $\mathcal{T}_e$ remains important. 

We admitted adversarial manipulation of our network topology here, but it sounds acceptable for our availability scheme, at least with $\mathcal{T}$ complete.  We shall consider whether this impacts gossip protocols in future work. 
% TODO:  Future work?  Here below?


\subsection{Availability} % GRANDPA
\label{sec:availability}

We integrate our availability and secondary validity check protocols directly with GRANDPA, in that an honest node $U \in \vals$ should not vote in GRANDPA for some relay chain block $R$ unless for all candidate receipts $\reciept_{B,S}$ in $R$,
\begin{itemize}
\item $|S| \geq \kappa_\para \npvals$,
\item $U$ possesses their own piece from $\pieces_B$,
\item number of unavailability reports are less than  $f+1$ and also
\item $U$ witnessed ``enough'' secondary checks for $\reciept_{B,S}$,
 as discussed blow in \S\ref{sec:approval}.
\end{itemize}

We run these availability and secondary validity check protocols only for parachain blocks included in relay chain blocks, not for all parachain blocks proposed by parachain validators.  In this way, we reduce the damage done by spammy parachains, at least beyond their own assigned parachain validators.

At the same time, we avoid complex anti-spam or Q logic since only GRANDPA requires this prior voting restriction.  In fact, if we later require prioritisation logic then this trick isolates it inside relay chain block production.

We need pieces to be distributed before secondary checkers announce themselves or begin their checks.  We therefore ask that validators gossip availability announcements for a relay chain block $R$ whenever they receive their piece for each parachain candidate receipt included in $R$.  As a higher bandwidth but more asynchronous alternative, we could ask that validators gossip availability announcements for candidate receipts, in which case a validator $V$ considers a relay chain block $R$ to be {\em available} once it observes $f+1$ availability announcements for each parachain candidate receipt included in $R$.

In either case, all validators discover the claimed availability of relay chain blocks long before any GRANDPA votes.  In fact, this claimed availability triggers the secondary approval checks discussed blow in \S\ref{sec:approval}.  
% TODO:  Any more details on GRANDPA integration?


\subsection{Unavailability} % BABE
\label{sec:unavailability}

We cannot entirely escape the availability question within BABE however:  Imagine we have several forks $C_1,\ldots,C_k$ for which at most $f$ validators possess all their chunks, but no fork for which $f+1$ validators possess all their chunks.  Yet, each block producers $U$ possess all their chunks for at least one fork $C_i$.  If BABE were oblivious to availability, then $U$ extends $C_i$, and GRANDPA stalls under this configuration. 
% TODO:  Anything about secondary validity check here ???

Instead, we define an availability grace period $\grace$ after which an unavailability subprotocol alters BABE's chain selection rule:  

If a validator $U \in \vals$ cannot obtain some piece $d \in \pieces_B$ within $\grace$ time after seeing the candidate receipt $\reciept_{B,\cdot}$ included in some relay chain block $R$, then $U$ announces via gossip the unavailability of the piece $d$. 
% TODO: Was $R_{head}$ but really?

We now explain the validators' responses to these unavailability announcements:

First, suppose some validator $V \in \vals$ observes an unavailability announcement for some candidate receipt $\reciept_{B,S}$ from some validator $\vals[i]$.  If $V$ possesses the block $\blobB$ then $V$ has computed $\pieces_B$ to check $\reciept_{B,S}$, so $V$ already possesses $\pieces_B[i]$ and should offer it to $\vals[i]$.  Any $V \in S$ satisfies this, but so shall any validators who signed off on the secondary checks discussed below in \S\ref{sec:validity}. 

Second, if any validator $U \in \vals$ observes unavailability announcements for pieces of some candidate receipt $\reciept_{B,\cdot}$ from at least $f+1$ different validators, then in BABE a block producer $U$ shall not propose a relay chain block containing any $\reciept_{B',\cdot}$ for which $B'$ is a descendent of $B$.
In this situation, there might be prevotes but never any precommits in GRANDPA for chains containing  $\reciept_{B,\cdot}$, so $\reciept_{B,\cdot}$ cannot possibly be finalised by GRANDPA.  We should consider if the GHOST chain weighting rule used by BABE and by GRANDPA for prevotes should weigh unavailability announcements, but doing so should only impact performance.

As an aside, we also considered $U$ abandoning any fork $C$ for which at least $f+1$ different validators gossiped unavailability announcements for possibly distinct blobs in $C$, instead of for the same blob.  Any truly unavailable pieces eventually trigger both conditions but they trigger this variant with possibly distinct blobs first, and with more false positives.  We optimise for the honest case here and caution that more false positives results in more chains, so spammy parachains could create more load on the availability system.

Any validator should revoke their past unavailability announcements for some piece $d \in \pieces_B$ whenever they eventually obtain $d$, again by gossiping the revocation.  We also define some super availability grace period, longer than $\npvals \grace$, after which time, if $2f+1$ of the validators announced unavailability of some specific piece, then the parachain validators who signed for that block are slashed.
We revert this slash whenever revocations later reduce the unavailability announcements below $f+1$.
% TODO: Any interactions with slashing computation?


\subsection{Approval} % phase
\label{sec:approval}

As mentioned above in \S\ref{sec:availability}, any validator $V$ accumulates availability announcements for each relay chain block $R$.  After $V$ observe $2f+1$ such availability claims, then $V$ considers $R$ {\em available} and enters the approval phase for $R$.  We assume that $V$ considers $R$ available and that $V$ has public-secret key pair $(\pk,\sk)$ throughout this subsection.

The approval phase for $R$ has two components, secondary checker assignments and actually performing the checks.  We have three flavours of secondary checker assignments, which all roughly follow the pattern of
\begin{itemize}
\item first VRF announcements via gossip, and
\item then computation of secondary checker assignments.
\end{itemize}
\noindent 
All secondary checker assignments trigger the same secondary check protocol, which consists of
\begin{itemize}
\item retrieval and reconstruction of full candidate proof-of-validity blobs, 
\item running secondary checks on the candidate proof-of-validity blobs, and 
\item announcing these candidates validity or invalidity.
\end{itemize}
We describe this second component first because 

...

\subsubsection{Checks} 

% In this subsection, 
We now temporarily suppose that $V$ completed some secondary checker assignment criteria subphase:  First, $V$ has constructed and announced via gossip some secondary checker VRF signature $\pi_{V,\cdot}$.  Second, there exists a parachain $\rho$ determined by the specific secondary checker assignment criteria applied to $\VRF.\Out(\pi)$.  We let $B$ denote the parachain candidate $B$ for $\rho$ in $R$ and $\reciept_{B,\cdot}$ its candidate receipt.  

We caution that $\pi_{V,\cdot}$ always depends upon the verifiers' ambiant knowledge, but we optionally package some extra knowledge with $\pi_{V,\cdot}$, like equivocation proofs (see below).  We prevent late stage adversarial influence from shifting secondary checker assignment criteria, but verifiers' ambiant knowledge also impacts if and when $V$ checks $B$.  We cannot however package this extra knowledge with $\pi_{V,\cdot}$, so validators can disagree for a limited time over whether $V$ checks $B$.

\smallskip
\paragraph{Retrieval:}

We never consider substructure of $B$ to be meaningful, so $V$ must {\em retrieve} the full {\em candidate proof-of-validity blob} $\blobB$ before checking.  Now $V$ knows which which nodes have their individual pieces, thanks to their availability announcements.  It thus follows from our 2/3rd honest assumption that $V$ could always reconstruct $\blobB$ by obtaining enough pieces $\pieces_B$ from nodes known to posses them.  

We note however that $V$ also knows that all pieces are known by the primary checkers aka parachain validators who approved $\blobB$, as well as secondary checkers who already approved $\blobB$.  So $V$ could first contact some node that possesses all of $\blobB$, and only then begin a full reconstruction process. 

In both cases, $V$ must recompute $\pieces_B$ to verify $\reciept_{B,\cdot}$.  We therefore cannot see much computational difference between $V$ reconstructing $\pieces_B$ from arbitrary pieces or from $\blobB$ itself.  It remains plausible $V$ avoids some networking overhead by asking for $\blobB$ though.  We think a first implementation could reasonably target reconstructing $\pieces_B$ from arbitrary pieces, while leaving requests for the full $\blobB$ to future optimisations. 

Ideally $V$ might retrieve the pieces in $\pieces_B$ only using its existing connections in our topology specified above, except these intentionally do not include 1/3rd of validators.  Also, $V$ need not connect to any node with all of $\pieces_B$.  Yet, $V$ should connect to at least one prachain validators in $\vals_\rho$ who ideally should check $B$ first.  

We strongly caution against abandoning secondary checkers over topology concerns because then adversarial influence over the topology could wreck our assignment criteria below.

In fact, our retrieval component could be engineered to avoid requests entirely:  After obtaining $\pi_{V,\cdot}$, another validator $V'$ could simply compute its own priority for sending its piece from $\pieces_B$ to $V$.  We caution that doing do might become inefficient, either because $V$ winds up rejecting sends, or when many nodes go offline.  

\smallskip
\paragraph{Announcement:}

After $V$ obtains enough pieces to recompute and verify $\reciept_{B,\cdot}$ from $\blobB$, then $V$ verifies $\blobB$ itself by running the execute block function of $\rho$.  If both these checks pass, then $V$ signs $\reciept_{B,\cdot}$ and announces its signature via gossip.  If either check fails, then $V$ claims invalidity for $\reciept_{B,\cdot}$, which results in all validators checking $B$.

\subsubsection{Assignments}

\newcommand\netdelay{\ensuremath{\Delta_{\mathsf{net}}}}
\newcommand\checkdelay{\ensuremath{\Delta_{\mathsf{check}}}}

We consider self-assignment schemes determined by several paramaters:
\begin{itemize}
\item our target number $\ell$ of eventual checkers,
\item our total number $n'$ of candidate checkers, normally $n' = \nvals - \npvals$,
\item the weight $\theta$ of a no-show checker with $1 \leq \theta \leq 2$,
\item a network delay bound $\netdelay$ for gossip messages, and
\item a block checking delay bound $\checkdelay$ that covers both retrieval and validation.
\end{itemize}
We never split the check bound $\checkdelay$ into separate retrieval and validation bounds, because doing so could only improves latency in extremely niche situations, and normally the extra gossip costs far more. 

\paragraph{Verifiable random sampling:}

We propose self-assignment schemes based on employ {\em verifiable random sampling} using a VRF:  

Any protocol run requires some seed $z$ and some associated time $T_0$ that marks the protocols' start time.  As a rule, we select $z$ to satisfy some freshness assumption, meaning our adversary learns $z$ before some time $T_0$ with odds less than our ambiant security treshold $f/n$.

Any validator $V$ with key pair $(\pk,\sk)$ computes its VRF signature $\tilde{\pi}_{V,z} := \VRF.\Sign_\sk(z)$, which it announces when specified by the specific protocol, but not before $T_0$.  Anyone else verifies $\tilde{\pi}_{V,z}$ with $\pk$.  All participants seed their sampling procedure for $V$'s assignment with the verifiable randomness $\tilde{\omega}_{V,z} := \VRF.\Out(\pi_{V,z})$.

We prefer $z$ already be known to verifiers, but our protocols run far too fast to enforce this.  We expect premature gossip messages that arrive before the verifier knows $z$ require subtle caching and expiration policies, but some scenarios provide a short proof for $z$, which may or may not be worth including in gossip messages.

We remark that gossip messages could be signed with $\pi$ itself by incorporating the output into the full gossip message, which gets signed as extra message data.  If done, this reduces the computational cost for signature verification by a third to a half, but may violate protocol layering and complicate validating protocol security and correctness, audits, etc. 

We always need target number $\ell$ of eventual checkers when sampling assignments based on $\tilde{\omega}_{V,z}$.  We shall often alter $\ell$ during protocol runs though, so assignments must be covariant in $\ell$, meaning increasing $\ell$ should not invalidate assignments made with smaller $\ell$.  We avoid rejection sampling after $\ell$ enters the sampling computation for this reason.

\paragraph{Generic one-shot delay:}

We first define a generic ``one-shot'' self-assignment subprotocol in which the seed $z$ uniquely identifies a parachain candidate $B$ on some parachain $\rho$.  We term this a ``one-shot'' protocol in that validators reveal their assignment at the last possible moment before checking, which costs us efficiency but gives code simplicity and completeness.  It also gives us a fig leaf of defense against some adaptive adversary classes.  We describe this subprotocol generically because it acts as a basis for almost all our self-assignment subprotocols, thanks to being one-shot, and should enable considerable code reuse.  

All validators track the set $A_z$ of valid announced checker claims, as well as the subset $S_z$ of all validity claims $S_B$ for $B$ generated by nodes in $A_z$.  We identify here the generic one-shot protocol run by $z$ because multiple $z$ may imply the same $B$, except we emphasise that messages exist only for $S_B$ not $S_z$, and that our protocol run ends by considering $S_z$ sufficient.  

We uniformly sample time delays from the interval $[0,\netdelay {n' \over \ell}]$, so that an expected $\ell$ land in $[0,\netdelay]$.  We use verifiable sampling here of course, meaning we compute this delay $d_{V,z}$ from $\tilde{\omega}_{V,z}$.  
%
As noted above, we shall alter $\ell$ during protocol runs, which prevents rejection sampling.  We do not however require perfect uniformity, so simple integer division suffices here $d_{V,z} = {\netdelay n' \omega_{V,z} - \netdelay \over \ell m}$ where $m$ is the upper bound on $\omega_{V,z}$, i.e.\ $\omega_{V,z} \in [0,m)$.  
%
In this, we subtract $\netdelay$ to condense the first $[0,\netdelay)$ time interval at zero to make all initial nodes start at time $T_0$ and prevent powerful adversaries from forcing nodes before them. 
%
In Rust, we might sample with code resembling
\begin{code}
fn generic_oneshot_assigned_delay(
    ctx: &'static [u8],
    network_delay: u32, 
    candidates: u32,
    target: u32,
    omega: ::schnorrkel::vrf::VRFInOut,\
) -> u32 {
    assert!(network_delay > 0); // Unrealistic network otherwise
    assert!(candidates > 0 && target > 0); // No candidates, not even us.
    let mut s = u32::from_le_bytes( omega.make_bytes(ctx) ) as u64;
    s *= (candidates as u64) * (network_delay as u64);
    s /= (u32::value_max() as u64) + 1;
    s.saturating_sub(network_delay as u64)
    .try_into::<u32>().expect("candidates * network_delay > u32::value_max()")
    / target
}
\end{code}

Consider a validator $V$ with current time $t$.  We describe how $V$ tracks the announcements $A_z$ and their answering validity claims $S_z$, and deals with no-shows.

$V$ records the time $t_a$ it receives each checker announcement $a \in A_z$.  Also $V$ defines its no-show count $\ell'$ to be the number of $a \in A_z$ for which both $t_a + \checkdelay < t$ and the owner $V'\in \vals$ of $a$ does not yet appear in $S_z$.

All validator $V$ announces via gossip their $\tilde{\pi}_{V,z}$, and begins retrieving and checking $B$, whenever
\begin{enumerate}
\item $T_0 + d_{V,z} + \netdelay < t$,
\item $|A_z| < \ell + \theta \ell'$, and
\item $V \notin \vals \setminus \vals_\rho$.
\begin{enumerate}
In some protocols, we do increase $\ell$ to compensate for absent parachain validators in $\vals_\rho$, so relieving a validity claim from a $V \in \vals_\rho$ may reduce $\ell$, but such a claim never counts towards $S_z$.

Also $V$ considers the block $B$ valid whenever $|S_z| \ge \ell + \theta \ell'$, but one-shot subprotocol cannot terminiate themselves because their caller might later increase $\ell$.  We ask the caller manage prioritisation for work on $z$ and $B$.

\subsubsection{Criteria}

We 
$\equivocationchecks$
$\basesecondarychecks$

\smallskip
\paragraph{Initial one-shot:}

% \subparagraph{Announcement:}

We recall that a relay chain block $R$ always has an associated VRF signature $\hat{\pi}_R$ whose output $\hat{\omega}_R := \VRF.\Out_(\hat{\pi}_R)$ gets recycled for the random beacon, and perhaps seals the block $R$ itself.  In Praos \cite{Praos} or BABE \cite{BABE}, $\hat{\omega}_R$ equals the VRF output $\omega_R$ that justifies creating $R$, but secret single leader elections like Sassafras \cite{Sassafras} might reveal $\omega_R$ early.  We want $\hat{\omega}_R$ to be the random beacon source that only the block producer of $R$ knows in advance, and that even they can only influence by choosing wether or not to make $R$.
% TODO: Should this be in AnV-1-setting.tex and shoud that say more about VRFs? 

We always run initial secondary checker assignments based on $\hat{\omega}_R$ and the parachain $\rho$ of $B$ to minimize an adversaries influence to their choice to use their own $\hat{\pi}_R$.  Implementations have two options for these initial secondary checker assignments, either repeatedly employ the one-shot scheme described above, or else reduce variance among validator workloads with a packed scheme described below.

We describe using the one-shot self-assignment subprotocol:  All validators run the one-shot self-assignment subprotocol for each parachain $\rho$ represented in $R$ using the seed $z_B := \hat{\omega}_R || \rho.\mathsf{seed}$.\footnote{We write $z_B$ instead of $z_{R,\rho}$ because parachains have only one block per relay chain block.}  We outline computing the associated target number $\ell_B$ of eventual checkers for the one-shot assignment subprotocol: 

As discussed previously, our approval phase begins only after acquiring enough preliminary backing parachain validators from $\rho$.  We suppose here that $\npvals$ preliminary backing checks to always be realistic, making fewer than $\npvals$ preliminary backing checks suspicious.  We set $\ell'_B$ to be the number $|\vals_\rho|$ of parachain validators for $\rho$ minus $|S \cap \vals_\rho|$. 
% TODO: \npvals vs |\vals_\rho|
We also define $\theta'$ to be a suspicion factor with $0 \le \theta' \le 2$, which we multiply by $\ell'_B$ below.

We accumulate several dynamic report factors, like an unavailability report factors $\rfuv$ from validators and $\runav$ from collators, and an invalidity report factor $\rinv$ from both collators and untrusted fishermen.  We set the target number $\ell$ of secondary checkers to be the sum of these, along with any missing preliminary backers $\npvals-\primarychecks$.  So
$$ \ell_B := \basesecondarychecks + \lceil \theta' \ell'_B + \rfuv + \runav + \rinv \rceil \mathoperiod $$

Implementations might slightly delay accounting for $\theta' \ell_\rho$, and possibly other factors, to prevent $\ell$ from being overly large initially. 

\smallskip
\paragraph{Initial packed:}

We observe one shortcoming of the generic one-shot self-assignment scheme:  All validators have separate $z_B$ for each parachain $\rho$ represented in $R$, which results in statistically independent VRF signatures $\hat{\pi}_{V,B}$ for each $\rho$.  

We normally love independence under attack scenarios, but here it frequently results in validators either not doing any secondary check or else doing far too many.  In principle, we could dampen the excess checks this with additional prioritisation rules for the generic scheme, except such rules allow adversaries to pack in their own nodes with high priorities.

Instead, we shall sacrifice some unneeded independence across parachains for a more balanced distribution.
%
We shall reuse our target number $\ell_B$ of secondary checkers as defined for the initial one-shot variant, which depends upon the parachain $\rho$ of $B$ as well as the relay chain block $R$.
%
Instead of distinct $z_B$ though, we ask that each $V$ announce one $\tilde{\pi}_{V,R} := \VRF.\Sign_\sk(\hat{\omega}_R)$ via gossip for $R$ as soon as it considers $R$ to be available, making $\hat{\omega}_R$ alone the seed.

% \subparagraph{Sampling:}

Let $P$ denote the full list of parachains, possibly padded with some null entries.
Also, for any validator $V$, let $\rho_V$ denote the parachain assignment of $V$.
We define a permutation $P_{V,R}$ of $P \setminut \{ \rho_V \}$ with a Fisher-Yates shuffle seeded by $\omega_{V,R} := \VRF.\Out(\tilde{\pi}_{V,R})$.  We note $P_{V,R}$ is entirely determined by $\hat{\omega}_R$ and the key $\sk$ of $V$, since $P$ itself remains constant and does not change with $R$.

We now assign $V$ to check the parachains listed earliest in its $P_{V,R}$ that require additional validators.
%
In concrete terms, anyone observing the gossip messages has the pre-assignment level sets
$$ \mathsf{pal}(R,\rho,i) := \Setst{ V }{ P_{V,R}[j] = \rho \quad\mathrm{for some $j < i$} } $$
and then defines $\mathsf{assignees}(R,\rho)$ to be $\mathsf{pal}(R,\rho,i)$ for the minimal $i$ such that $\mathsf{assignees}(R,\rho)$ has at least the desired number $\ell_B$ of validators.\footnote{We could leave $\ell_B$ as a paramater here for a more generic construction, but this sampling procedure neer gets reused elsewhere.}

% \smallskip
% \subparagraph{Assigning:}

Assuming perfect information, it follows that $\mathsf{assignees}(R,\rho)$ gives our secondary checkers assigned to a parachain candidate $B$ on $\rho$ in $R$. 
%
We caution however that $\omega_{V,R}$ might remain unknown for numerous $V$, due to delays in availability or gossip, either here or earlier.  As this artificially enlarges $\mathsf{assignees}(R,\rho)$, we suggest using $\mathsf{assignees}(R,\rho) \cap \mathsf{pal}(R,\rho,\lfloor t/\netdelay \rfloor)$ for some time delayed index paramater $t$ so as to prevent $\mathsf{assignees}(R,\rho)$ from growing too fast and then shrinking.

% \smallskip
% \subparagraph{Delays:}

In this way, we still avoid relay chain block producers holding excessive power over the assignments, much like in the generic one-shot scheme.  We do however sacrifice some independence of delays across parachains so that our shuffle $P_{V,R}$ can distribute these initial secondary checker assignments more evenly among the validators.  

Our shuffle has running time linear in the number of parachains, with each iteration sampling from say $\mathsf{ChaChaRng}(\omega_{V,R})$.  We could produce $P_{V,R}$ lazily with a lazy Fisher-Yates shuffle, which sounds like premature optimisation, but gives another advantage over sampling delays.

If larger delays are desired, then we could either pad $P$ with null entries, or else sample extra delays between the levels, probably from a geometric distribution.  We favor sampling extra delays because if delays are large then padding $P$ slows computing $P_{V,R}$ more.

\smallskip
\paragraph{Initial no-shows:}

We address no-shows carefully in the one-shot scheme, but ignored them in the initial packed discussion since no-shows defenses largely help against adaptive attacker classes.  We suggest the initial packed scheme fall back onto the one-shot scheme for no-shows.  Intuitively, we run the initial one-shot scheme in parallel with ``$\ell = 0$'' or more precisely with the generic one-shot scheme's second announcement condition replaced by $|A_z| < \theta \ell'$. 

We should strongly consider moving some or all of the report factor from $\ell_B$ into this condition too.  ...
% TODO:  Alistair, Jeff, and Handan must discudd this part carefully .

At the extreme, our shuffled parachain list $P_{V,R}$ could be replaced with a single check on the parachain $\rho$ whose index is $\tilde{\pi}_{V,R}$ taken modulo the number of parachains.  All these checks could replace the one-shot winners with no delay.

As an aside, we considered using the no-show validators' VRF outputs $\omega_{V,R}$ here, but judge such designs to be unnecessary code complexity. 

\smallskip
\paragraph{Equivocation:}

% TODO:  Alistair, Jeff, and Handan must discudd this part carefully to set \equivocationchecks.  I preserved that equivocation checks use parachain block hashes, not relay chain block hashes, from Handan's write up, which make sense because more equivocations does not require more checks.  It makes (3) from her write up make no sense.  Also her (4) used #Vcehck in a strange way. 

We say $R$ has equivocations if there exists another relay chain block $R'$ distinct from $R$ but using the same VRF output $\hat{\omega}_{R'} = \hat{\omega}_R$ for the relay chain randomness.  We distrust relay chain blocks with equivocations of course, and must mark them as having equivocations, but we sadly cannot invalidate them because doing so creates attacks on finality. 
% TODO: Explain?

We therefore fear an adversary might create an $R'$ with an innocent $B'$ on $\rho$ merely to learn the secondary checkers determined by $\hat{\omega}_R$ for $\rho$, but then equivocate with another $R$ that containing a malicious $B$ on $\rho$.  In this scenario, our equivocation $R$ triggers exactly the same secondary checks for $B$ as $R'$ did for $B'$, which wrecks our advantage over the adversary.

We desire additional checks that remain unforeseeable to our adversary here, but without triggering excessive checks against innocent parachain candidates under repeated equivocations.     

We find this compromise by instantiating the generic one-shot self-assignment subprotocol with seed $z_B = H(B)$, or perhaps with $H(B)$ replaced by the Merkle root $\merkleroot_B$, so that each parachain candidate gets assigned only an extra checkers group only once.  In other words, we work with VRF signatures $\tilde{\pi}_{V,B} := \VRF.\Sign_\sk(H(B))$ and do verifiable sampling with the VRF outputs $\omega_{V,B} := \VRF.\Out(\tilde{\pi}_{V,B})$, as opposed to the $\tilde{\pi}_{V,R,\rho}$ or $\tilde{\pi}_{V,R}$ from our initial checks.

In this one-shot application, we could choose our target number $\ell$ of secondary checkers to be a constant $\equivocationchecks$, perhaps equal to $\basesecondarychecks$, or at the other extreme reuse our target number $\ell_B$ of secondary checkers as defined for the initial one-shot variant.  We suggest however that report factor from $\ell_B$ by reconsidered under the equivocations attack scenario.

In this way, an adversary who equivocates always faces $\ell$ extra checks against their new parachain block $B$, without impacting the innocent parachains too much when faced with numerous equivocations.

Implementers might suppress this check if parachains always carry less value than gets slashed when relay chain block producers equivocate.  Attacks could impact multiple parachains simultaneously but doing so requires significantly more resources.


\subsection{Fishermen}
\label{sec:fishermen}

...




