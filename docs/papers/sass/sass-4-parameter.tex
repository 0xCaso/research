\section{Parameter Selection}

The parameters of Sassafras should be selected with respect to the results in Section 
\ref{sec:analysis}.

According to Theorem \ref{th:sec}, $ p_\mathsf{hSlot} $ must be greater than 0.5. Therefore,
\begin{itemize}
	\item if $ \nweak = 0 $, $ p_{\mathsf{pSlot}} + p_\mathsf{mSlot} = 1- \alpha^2  \leq 0.5 $ implies that $ \alpha \geq \sqrt{0.5} \approx 0.71  $
	
	\item if  if $ \nweak > 0 $, we should also consider random weak slots. So, we need $ p_{\mathsf{pSlot}} + p_\mathsf{mSlot} + p_\mathsf{rSlot} \leq 0.5$. Since $ p_\mathsf{rSlot} $ depends on $ \eplen , \vrfaattemptsbound $ and $ \alpha $, we should decide alpha after fixing $ \eplen  $. 
\end{itemize}


\paragraph{Finding Epoch Length with respect to ECQ, CP and CG parameters:}

We first need to decide our security bound e.g., $ p_{attack} = 0.005 $ and a life time $ L $ in terms of epoch. Below, we choose the parameters for an adversary who can break the Sassafras in $ L $ epochs with at most the probability $ p_{attack} $.	

\begin{enumerate}
	\item Decide $ \eplen $
	\item Set $ s_{ecq} = s_{hcg}  = \eplen /3 $ (since $ \eplen  $ should be at least $ 2s_{ecq} + s_{hcg} $)
	\item $ \tau_{hcg} = p_{\mathsf{hSlot}} * (1 - \delta_{\mathsf{hSlot}}) $ (from Lemma \ref{lem:hcg})
	\item $ \tau = \tau_{hcg} \frac{s_{hcg}}{2s_{ecq} + s_{hcg}} $ (from Lemma \ref{lem:cg})
	\item $ k_{ecq} = \tau s_{ecq}  $ and $ k = \tau \eplen  $ to satisfy the Theorem \ref{th:sec}
	\item Compute $ p_{sass} $ as in Theorem \ref{th:sec} based on the parameters above. If it is less than $ p_{attack} $, then we find the smallest possible $ \eplen $.
\end{enumerate}

According to above algorithm, we find that the epoch length $ \eplen = 1599 $ slots and $ k = 134 $.

Now, we set $ \eplen $. Based on this we should find $ \vrfaattemptsbound $ such that $ p_{\mathsf{pSlot}} + p_\mathsf{mSlot} + p_\mathsf{rSlot} \leq 0.5$. In Tables,  we find the minimum $ \vrfaattemptsbound $, $ c $ and $ \alpha $ based on $ \nweak $ which is the number of weak corruption that an adversary can execute simultaneously for $ n = 100 $, $ n = 200 $, $ n = 500 $ and $ n =1000 $, respectively. As it can be seen from these tables, when we have less validators, we need more $ \vrfaattemptsbound $ to deanonymise the slots.



	\begin{table}
	\parbox{.45\linewidth}{
		\centering
		\begin{tabular}{|c|c|c|c|}
			\hline
			$ \nweak $ & $ \alpha $ & $ \vrfaattemptsbound $& $ c $  \\\hline
			1&0.75&64&0.81
			\\\hline
			2&0.77&54&0.95
			\\\hline
			
			3&-&-&-
			\\\hline
			
			4&-&-&-
			\\\hline
			
			5&-&-&-
			\\\hline
			
			6&-&-&-
			\\\hline
			7&-&-&-
			\\\hline
		\end{tabular}
		\caption{Parameters when $ n = 100 $}}
		\label{tb:param100}
\hfill
		\parbox{.45\linewidth}{
		\centering
		\begin{tabular}{|c|c|c|c|}
			\hline
			$ \nweak $ & $ \alpha $ & $ \vrfaattemptsbound $& $ c $  \\\hline
			1&0.73&63&0.62
			\\\hline
			2&0.74&63&0.72
			\\\hline
			3&0.75&59&0.77
			\\\hline
			4&0.76&45&0.86
			\\\hline
			5&0.76&62&0.9
			\\\hline
			6&0.77&47&0.95
			\\\hline
			7&0.77&64&0.97
			\\\hline
		\end{tabular}
		\caption{Parameters when $ n = 200 $}
		\label{tb:param200}}
	\end{table}



\begin{table}
\parbox{.45\linewidth}{
	\centering
	\begin{tabular}{|c|c|c|c|}
		\hline
		$ \nweak $ & $ \alpha $ & $ \vrfaattemptsbound $& $ c $  \\\hline
1&0.72&35&0.5
\\\hline
2&0.73&23&0.59
\\\hline
3&0.73&45&0.59
\\\hline
4&0.74&22&0.68
\\\hline
5&0.74&31&0.68
\\\hline
6&0.74&57&0.68
\\\hline
7&0.74&60&0.77
\\\hline
	\end{tabular}
\caption{Parameters when $ n = 500 $}
\label{tb:param500}
}
\hfill
\parbox{.45\linewidth}{
	\centering
	\begin{tabular}{|c|c|c|c|}
		\hline
		$ \nweak $ & $ \alpha $ & $ \vrfaattemptsbound $& $ c $  \\\hline
	1&0.72&10&0.5
	\\\hline
	2&0.72&18&0.5
	\\\hline
	3&0.72&56&0.52
	\\\hline
	4&0.73&12&0.59
	\\\hline
	5&0.73&16&0.59
	\\\hline
	6&0.73&23&0.59
	\\\hline
	7&0.73&45&0.59
	\\\hline
	\end{tabular}
	\caption{Parameters when $ n = 1000 $}
	\label{tb:param1000}
}
\end{table}





 
 
 
 