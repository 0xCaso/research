\section{Definitions}

\paragraph{Notations} $\V= \{V_1,V_2,...,V_n\}$ is the validator set consisting of validators $V_i$. $\pv$ is a set of parachain validators and $\pv \subset \V$. For the sake of distinction, we denote parachain validators by $PV_i \equiv V_i$.

\begin{definition}[Blob]
A blob is a tuple $(B, \pi, M)$ where $B$ is the parachain block, $\pi$ is the validity proof of $B$ and $M$ is the set of outgoing messages from that parachain.
\end{definition}


\begin{definition}[Block Header]\label{def:header}
Block header is a summary of the block $B$ which links to $B\in PC$ and includes all the signatures by parachain validators who claims its validity. There exists a function $\recons$ which takes the block header $B_{head}$ and block data $B_{data}$ as an input and outputs $B$ which $B_{head}$ is linked to. 
\end{definition}



\begin{definition}[Erasure Code]\label{def:erasure}
Erasure code consists of two algorithms: $\divi_k$ and $\recons$ Given data $D$, the algorithm $\divi$ outputs $n$ pieces $d_1,d_2,...,d_n$ and given $k$ out of $n$ pieces, $\recons$ outputs $D$.  

\end{definition}

In other words, erasure code divides data into $n$ pieces and at least $k$ out of $n$ pieces are enough to construct the data.

In the availability and validity protocol in Section \ref{sec:protocol}, parachain validators divide a blob into $n$ pieces. These $n$ pieces corresponds to the $B_{data}$ in Definition \ref{def:header} and so the $\recons$ algorithm in Definition \ref{def:header}  corresponds to the $\recons$ algorithm of the erasure code in Definition \ref{def:erasure}.








\subsection{Security Model}


%Think it again
%\begin{definition}[Risk Value]
%Given a total stake $S$ and slashing value $x$, the risk value is defined as $p\frac{x}{S}$ where $p$ is the probability of getting caught.
%\end{definition}

The active parties in the availability and validity scheme are validators, parachain validators, fishermen and collators. 




\begin{itemize}

\item Parachain validators are responsible to validate a blob. They can be malicious.
\item Collators are the full nodes of parachain and provide a PoV blob to the responsible parachain validators. They can be malicious and collude with parachain validators.
\item Validators are responsible of maintaining relay chain and finalizing the blocks in the relay chain. A validator can be also a parachain validator. 
\item Fishermen inspects any malicious activity. If any exists they announce and prove it.  



\end{itemize}

In our security model, we assume that all parties except fishermen can be malicious. Given $|V| = n = 3f + 1$, at most number of $f$ validators can be malicious.


\begin{definition}[Proof of Validity (PoV)]\label{def:pob}
PoV consists of two algorithms $\prove$ and $\verify$:
\begin{itemize}
    \item $\prove$  takes the block $B$, the parachain $PC$ and the outgoing messages from $PC$ as an input and outputs the proof $\pi$.
    \item $\verify$  takes the blob $(B,\pi,M)$ as an input and outputs 1 (valid) or 0 (invalid).
\end{itemize}
\end{definition}
\paragraph{Correctness of PoV:} PoV is correct if for all $B \in PC$ and $M$, $\verify(B,\pi,M)\rightarrow 1$ where $\prove(B, PC,M)\rightarrow \pi$.

\paragraph{Security of PoV:} PoV is secure if an adversary generates a blob $(B,\pi,M)$ for a parachain $PC$ where  $\verify(B,\pi,M) \rightarrow 1$ and $B \notin PC$ with a negligible probability.


We note that $B\in PC$ (resp. $B \notin PC$) means that the block $B$ is a valid (resp. invalid) block for a parachain $PC$.


\begin{definition}[Unavailable Block]\label{def:unavail}
A block is unavailable if less than $k$ erasure code of its blob obtained by honest validators.
\end{definition}


\begin{definition}[Security of Availability]
Assume that at most number of $f$ validators are corrupted. If the probability of having a finalized block in the relay chain which includes an unavailable block is negligible, then the availability protocol is secure against malicious validators.
\end{definition}

%More formal definition
\begin{definition}[Security of Validity]
Assume that  at most number of $f$ validators are corrupted. If the probability of having a finalized block in the relay chain which includes the header of an invalid block is less than a risk value, then the validity protocol is secure against malicious validators.
\end{definition}






